%% Generated by Sphinx.
\def\sphinxdocclass{report}
\documentclass[a4paper,10pt,english]{sphinxmanual}
\ifdefined\pdfpxdimen
   \let\sphinxpxdimen\pdfpxdimen\else\newdimen\sphinxpxdimen
\fi \sphinxpxdimen=.75bp\relax
\ifdefined\pdfimageresolution
    \pdfimageresolution= \numexpr \dimexpr1in\relax/\sphinxpxdimen\relax
\fi
%% let collapsable pdf bookmarks panel have high depth per default
\PassOptionsToPackage{bookmarksdepth=5}{hyperref}

\PassOptionsToPackage{warn}{textcomp}
\usepackage[utf8]{inputenc}
\ifdefined\DeclareUnicodeCharacter
% support both utf8 and utf8x syntaxes
  \ifdefined\DeclareUnicodeCharacterAsOptional
    \def\sphinxDUC#1{\DeclareUnicodeCharacter{"#1}}
  \else
    \let\sphinxDUC\DeclareUnicodeCharacter
  \fi
  \sphinxDUC{00A0}{\nobreakspace}
  \sphinxDUC{2500}{\sphinxunichar{2500}}
  \sphinxDUC{2502}{\sphinxunichar{2502}}
  \sphinxDUC{2514}{\sphinxunichar{2514}}
  \sphinxDUC{251C}{\sphinxunichar{251C}}
  \sphinxDUC{2572}{\textbackslash}
\fi
\usepackage{cmap}
\usepackage[T1]{fontenc}
\usepackage{amsmath,amssymb,amstext}
\usepackage{babel}


\usepackage{amsmath,amsfonts,amssymb,amsthm}


\usepackage{fncychap}
\usepackage{sphinx}

\fvset{fontsize=auto}
\usepackage{geometry}


% Include hyperref last.
\usepackage{hyperref}
% Fix anchor placement for figures with captions.
\usepackage{hypcap}% it must be loaded after hyperref.
% Set up styles of URL: it should be placed after hyperref.
\urlstyle{same}

\addto\captionsenglish{\renewcommand{\contentsname}{User Guide:}}

\usepackage{sphinxmessages}
\setcounter{tocdepth}{1}


% Jupyter Notebook code cell colors
\definecolor{nbsphinxin}{HTML}{307FC1}
\definecolor{nbsphinxout}{HTML}{BF5B3D}
\definecolor{nbsphinx-code-bg}{HTML}{F5F5F5}
\definecolor{nbsphinx-code-border}{HTML}{E0E0E0}
\definecolor{nbsphinx-stderr}{HTML}{FFDDDD}
% ANSI colors for output streams and traceback highlighting
\definecolor{ansi-black}{HTML}{3E424D}
\definecolor{ansi-black-intense}{HTML}{282C36}
\definecolor{ansi-red}{HTML}{E75C58}
\definecolor{ansi-red-intense}{HTML}{B22B31}
\definecolor{ansi-green}{HTML}{00A250}
\definecolor{ansi-green-intense}{HTML}{007427}
\definecolor{ansi-yellow}{HTML}{DDB62B}
\definecolor{ansi-yellow-intense}{HTML}{B27D12}
\definecolor{ansi-blue}{HTML}{208FFB}
\definecolor{ansi-blue-intense}{HTML}{0065CA}
\definecolor{ansi-magenta}{HTML}{D160C4}
\definecolor{ansi-magenta-intense}{HTML}{A03196}
\definecolor{ansi-cyan}{HTML}{60C6C8}
\definecolor{ansi-cyan-intense}{HTML}{258F8F}
\definecolor{ansi-white}{HTML}{C5C1B4}
\definecolor{ansi-white-intense}{HTML}{A1A6B2}
\definecolor{ansi-default-inverse-fg}{HTML}{FFFFFF}
\definecolor{ansi-default-inverse-bg}{HTML}{000000}

% Define an environment for non-plain-text code cell outputs (e.g. images)
\makeatletter
\newenvironment{nbsphinxfancyoutput}{%
    % Avoid fatal error with framed.sty if graphics too long to fit on one page
    \let\sphinxincludegraphics\nbsphinxincludegraphics
    \nbsphinx@image@maxheight\textheight
    \advance\nbsphinx@image@maxheight -2\fboxsep   % default \fboxsep 3pt
    \advance\nbsphinx@image@maxheight -2\fboxrule  % default \fboxrule 0.4pt
    \advance\nbsphinx@image@maxheight -\baselineskip
\def\nbsphinxfcolorbox{\spx@fcolorbox{nbsphinx-code-border}{white}}%
\def\FrameCommand{\nbsphinxfcolorbox\nbsphinxfancyaddprompt\@empty}%
\def\FirstFrameCommand{\nbsphinxfcolorbox\nbsphinxfancyaddprompt\sphinxVerbatim@Continues}%
\def\MidFrameCommand{\nbsphinxfcolorbox\sphinxVerbatim@Continued\sphinxVerbatim@Continues}%
\def\LastFrameCommand{\nbsphinxfcolorbox\sphinxVerbatim@Continued\@empty}%
\MakeFramed{\advance\hsize-\width\@totalleftmargin\z@\linewidth\hsize\@setminipage}%
\lineskip=1ex\lineskiplimit=1ex\raggedright%
}{\par\unskip\@minipagefalse\endMakeFramed}
\makeatother
\newbox\nbsphinxpromptbox
\def\nbsphinxfancyaddprompt{\ifvoid\nbsphinxpromptbox\else
    \kern\fboxrule\kern\fboxsep
    \copy\nbsphinxpromptbox
    \kern-\ht\nbsphinxpromptbox\kern-\dp\nbsphinxpromptbox
    \kern-\fboxsep\kern-\fboxrule\nointerlineskip
    \fi}
\newlength\nbsphinxcodecellspacing
\setlength{\nbsphinxcodecellspacing}{0pt}

% Define support macros for attaching opening and closing lines to notebooks
\newsavebox\nbsphinxbox
\makeatletter
\newcommand{\nbsphinxstartnotebook}[1]{%
    \par
    % measure needed space
    \setbox\nbsphinxbox\vtop{{#1\par}}
    % reserve some space at bottom of page, else start new page
    \needspace{\dimexpr2.5\baselineskip+\ht\nbsphinxbox+\dp\nbsphinxbox}
    % mimick vertical spacing from \section command
      \addpenalty\@secpenalty
      \@tempskipa 3.5ex \@plus 1ex \@minus .2ex\relax
      \addvspace\@tempskipa
      {\Large\@tempskipa\baselineskip
             \advance\@tempskipa-\prevdepth
             \advance\@tempskipa-\ht\nbsphinxbox
             \ifdim\@tempskipa>\z@
               \vskip \@tempskipa
             \fi}
    \unvbox\nbsphinxbox
    % if notebook starts with a \section, prevent it from adding extra space
    \@nobreaktrue\everypar{\@nobreakfalse\everypar{}}%
    % compensate the parskip which will get inserted by next paragraph
    \nobreak\vskip-\parskip
    % do not break here
    \nobreak
}% end of \nbsphinxstartnotebook

\newcommand{\nbsphinxstopnotebook}[1]{%
    \par
    % measure needed space
    \setbox\nbsphinxbox\vbox{{#1\par}}
    \nobreak % it updates page totals
    \dimen@\pagegoal
    \advance\dimen@-\pagetotal \advance\dimen@-\pagedepth
    \advance\dimen@-\ht\nbsphinxbox \advance\dimen@-\dp\nbsphinxbox
    \ifdim\dimen@<\z@
      % little space left
      \unvbox\nbsphinxbox
      \kern-.8\baselineskip
      \nobreak\vskip\z@\@plus1fil
      \penalty100
      \vskip\z@\@plus-1fil
      \kern.8\baselineskip
    \else
      \unvbox\nbsphinxbox
    \fi
}% end of \nbsphinxstopnotebook

% Ensure height of an included graphics fits in nbsphinxfancyoutput frame
\newdimen\nbsphinx@image@maxheight % set in nbsphinxfancyoutput environment
\newcommand*{\nbsphinxincludegraphics}[2][]{%
    \gdef\spx@includegraphics@options{#1}%
    \setbox\spx@image@box\hbox{\includegraphics[#1,draft]{#2}}%
    \in@false
    \ifdim \wd\spx@image@box>\linewidth
      \g@addto@macro\spx@includegraphics@options{,width=\linewidth}%
      \in@true
    \fi
    % no rotation, no need to worry about depth
    \ifdim \ht\spx@image@box>\nbsphinx@image@maxheight
      \g@addto@macro\spx@includegraphics@options{,height=\nbsphinx@image@maxheight}%
      \in@true
    \fi
    \ifin@
      \g@addto@macro\spx@includegraphics@options{,keepaspectratio}%
    \fi
    \setbox\spx@image@box\box\voidb@x % clear memory
    \expandafter\includegraphics\expandafter[\spx@includegraphics@options]{#2}%
}% end of "\MakeFrame"-safe variant of \sphinxincludegraphics
\makeatother

\makeatletter
\renewcommand*\sphinx@verbatim@nolig@list{\do\'\do\`}
\begingroup
\catcode`'=\active
\let\nbsphinx@noligs\@noligs
\g@addto@macro\nbsphinx@noligs{\let'\PYGZsq}
\endgroup
\makeatother
\renewcommand*\sphinxbreaksbeforeactivelist{\do\<\do\"\do\'}
\renewcommand*\sphinxbreaksafteractivelist{\do\.\do\,\do\:\do\;\do\?\do\!\do\/\do\>\do\-}
\makeatletter
\fvset{codes*=\sphinxbreaksattexescapedchars\do\^\^\let\@noligs\nbsphinx@noligs}
\makeatother



\title{Anaerobic Digestion Model \#1 Fast (ADM1F)}
\date{Aug 26, 2021}
\release{0.1}
\author{Elchin Jafarov, Satish Karra,\and Kuang Zhu,\and Wenjuan Zhang, Steven Skerlos, Kurt Solander, Lutgarde Raskin, and Meltem Urgun Demirtas}
\newcommand{\sphinxlogo}{\vbox{}}
\renewcommand{\releasename}{ }
\makeindex
\begin{document}

\pagestyle{empty}
\sphinxmaketitle
\pagestyle{plain}
\sphinxtableofcontents
\pagestyle{normal}
\phantomsection\label{\detokenize{index::doc}}


\sphinxAtStartPar
Anaerobic digestion (AD) process converts organic wastes into biogas. Biogas can generate heat and electricity through a cascade of biochemical reactions and has been adapted by various facilities and industries to treat and recover energy from high\sphinxhyphen{}strength liquid or solid waste streams. Anaerobic Digestion Model 1 (ADM1) is a mathematical model that describes the stoichiometry and kinetics of the essential biochemical reactions in AD. This repository includes C++ version of the Matlab/Simulink %
\begin{footnote}[1]\sphinxAtStartFootnote
\sphinxhref{https://pdfs.semanticscholar.org/9f84/13e7bb8ec49b3d0eb321e9d54720f117a527.pdf}{Rosen C., Vrecko D., Gernaey K.V., Pons M.\sphinxhyphen{}N. and Jeppsson U. (2006). Implementing ADM1 for plant\sphinxhyphen{}wide benchmark simulations in Matlab/Simulink. Water Sci. Technol., 54(4), 11\sphinxhyphen{}19.}
%
\end{footnote} version of the ADM1 model and the solid retention time (SRT) version %
\begin{footnote}[2]\sphinxAtStartFootnote
\sphinxtitleref{Zhu et al, in prep. A Novel Core\sphinxhyphen{}shell ADM1 model allows rapid optimization of membrane anaerobic digestion processes}
%
\end{footnote}. The C++ version of the model is computationally more efficient than its Matlab/Simulink predecessor. We called this version of the model Anaerobic Digestion Model 1 Fast (ADM1F).

\noindent\sphinxincludegraphics[width=1216\sphinxpxdimen,height=685\sphinxpxdimen]{{digester_m}.png}


\chapter{Compile ADM1F}
\label{\detokenize{compile:compile-adm1f}}\label{\detokenize{compile::doc}}\begin{enumerate}
\sphinxsetlistlabels{\arabic}{enumi}{enumii}{}{.}%
\setcounter{enumi}{-1}
\item {} 
\sphinxAtStartPar
Download or clone the ADM1F model to your local computer:

\begin{sphinxVerbatim}[commandchars=\\\{\}]
\PYGZdl{} git clone https://github.com/wwtmodels/Anaerobic\PYGZhy{}Digestion\PYGZhy{}Models.git
\end{sphinxVerbatim}

\begin{sphinxVerbatim}[commandchars=\\\{\}]
\PYGZdl{} cd ADM1F
\end{sphinxVerbatim}

\item {} 
\sphinxAtStartPar
ADM1F uses external numerical library package PETSc. First download PETSc:

\begin{sphinxVerbatim}[commandchars=\\\{\}]
\PYGZdl{} cd build; git clone \PYGZhy{}b release https://gitlab.com/petsc/petsc.git petsc
\end{sphinxVerbatim}

\begin{sphinxVerbatim}[commandchars=\\\{\}]
\PYGZdl{} cd petsc; git checkout v3.14
\end{sphinxVerbatim}

\item {} 
\sphinxAtStartPar
Set \sphinxstylestrong{PETSC\_DIR} and \sphinxstylestrong{PETSC\_ARCH} in your environmental variables. We suggest to put these lines in your \sphinxtitleref{\textasciitilde{}/.bashrc} or similar files (\sphinxtitleref{\textasciitilde{}/.bash\_profile} on Mac OS X). Once you add it into the bash file, run \sphinxtitleref{source  \textasciitilde{}/.bash\_profile}:

\begin{sphinxVerbatim}[commandchars=\\\{\}]
\PYGZdl{} export PETSC\PYGZus{}DIR=/path\PYGZhy{}to\PYGZhy{}my\PYGZhy{}ADM1F\PYGZhy{}folder/build/petsc
\end{sphinxVerbatim}

\sphinxAtStartPar
and:

\begin{sphinxVerbatim}[commandchars=\\\{\}]
\PYGZdl{} export PETSC\PYGZus{}ARCH=macx\PYGZhy{}debug
\end{sphinxVerbatim}

\sphinxAtStartPar
\sphinxstylestrong{Make sure} that ‘adolc\sphinxhyphen{}utils’ folder is in the ‘build’ folder.

\item {} 
\sphinxAtStartPar
Configure PETSC:

\begin{sphinxVerbatim}[commandchars=\\\{\}]
\PYGZdl{} ./configure \PYGZhy{}\PYGZhy{}download\PYGZhy{}mpich \PYGZhy{}\PYGZhy{}with\PYGZhy{}cc=clang \PYGZhy{}\PYGZhy{}with\PYGZhy{}fc=gfortran \PYGZhy{}\PYGZhy{}with\PYGZhy{}debugging=0 \PYGZhy{}\PYGZhy{}download\PYGZhy{}adolc PETSC\PYGZus{}ARCH=macx\PYGZhy{}debug \PYGZhy{}\PYGZhy{}with\PYGZhy{}cxx\PYGZhy{}dialect=C++11 \PYGZhy{}\PYGZhy{}download\PYGZhy{}colpack
\end{sphinxVerbatim}

\end{enumerate}

\sphinxAtStartPar
\sphinxstylestrong{NOTE}: that these are for Mac OSX. If you are installing on a linux machine, then replace \sphinxstylestrong{clang} with \sphinxstylestrong{gcc}. Also, sometimes turning off \sphinxtitleref{\textendash{}with\sphinxhyphen{}fc=0} could help with compilation. This step will take awhile.
\begin{enumerate}
\sphinxsetlistlabels{\arabic}{enumi}{enumii}{}{.}%
\setcounter{enumi}{3}
\item {} 
\sphinxAtStartPar
If configuration goes well, you can then compile. This step will take awhile too.:

\begin{sphinxVerbatim}[commandchars=\\\{\}]
\PYGZdl{} make PETSC\PYGZus{}DIR=/path\PYGZhy{}to\PYGZhy{}my\PYGZhy{}ADM1F\PYGZhy{}folder/build/petsc PETSC\PYGZus{}ARCH=macx\PYGZhy{}debug all
\end{sphinxVerbatim}

\item {} 
\sphinxAtStartPar
After compilation, PETSc will show you how to test your installation (testing is optional).

\item {} 
\sphinxAtStartPar
Navigate back to the \sphinxtitleref{build} folder (\sphinxtitleref{cd ../}) and compile adm1f:

\begin{sphinxVerbatim}[commandchars=\\\{\}]
\PYGZdl{} make adm1f
\end{sphinxVerbatim}

\sphinxAtStartPar
or:

\begin{sphinxVerbatim}[commandchars=\\\{\}]
\PYGZdl{} make
\end{sphinxVerbatim}

\item {} 
\sphinxAtStartPar
Set \sphinxstylestrong{ADM1F\_EXE} in your environmental variable. Add this line in your \sphinxtitleref{\textasciitilde{}/.bashrc} or similar files (\sphinxtitleref{\textasciitilde{}/.bash\_profile} on Mac OS X).  Once you add it into the bash file, do not forget to \sphinxtitleref{source  \textasciitilde{}/.bash\_profile}:

\begin{sphinxVerbatim}[commandchars=\\\{\}]
\PYGZdl{} export ADM1F\PYGZus{}EXE=path\PYGZhy{}to\PYGZhy{}my\PYGZhy{}ADM1F\PYGZhy{}folder/build/adm1f
\end{sphinxVerbatim}

\item {} 
\sphinxAtStartPar
\sphinxstylestrong{NOTE}: There are two versions of the ADMF1: the original version  (adm1f.cxx), and the modified version of the model (adm1f\_srt.cxx, see {\hyperref[\detokenize{compile:adm1f-srt}]{\sphinxcrossref{\DUrole{std,std-ref}{ADM1F SRT}}}}).

\end{enumerate}


\chapter{Running ADM1F}
\label{\detokenize{compile:running-adm1f}}\begin{enumerate}
\sphinxsetlistlabels{\arabic}{enumi}{enumii}{}{.}%
\item {} 
\sphinxAtStartPar
Make sure that \sphinxstylestrong{ADM1F\_EXE:} is not empty (see step 7 from the previous section).:

\begin{sphinxVerbatim}[commandchars=\\\{\}]
\PYGZdl{} echo \PYGZdl{}ADM1F\PYGZus{}EXE
\end{sphinxVerbatim}

\item {} 
\sphinxAtStartPar
Navigate to the \sphinxtitleref{simulations} folder and run the model:

\begin{sphinxVerbatim}[commandchars=\\\{\}]
\PYGZdl{} \PYGZdl{}ADM1F\PYGZus{}EXE

or using command\PYGZhy{}line options (see 4 and 5):

\PYGZdl{} \PYGZdl{}ADM1F\PYGZus{}EXE \PYGZhy{}ts\PYGZus{}monitor \PYGZhy{}steady
\end{sphinxVerbatim}

\item {} 
\sphinxAtStartPar
Note that adm1f will look for three files \sphinxtitleref{ic.dat}, \sphinxtitleref{params.dat}, and \sphinxtitleref{influent.dat}, which contain the initial conditions (45 values), parameters (100 values), and influent values (28 values), see {\hyperref[\detokenize{inouts:inouts-label}]{\sphinxcrossref{\DUrole{std,std-ref}{Inputs/Outputs}}}}.

\item {} 
\sphinxAtStartPar
The command\sphinxhyphen{}line options are:
\begin{itemize}
\item {} 
\sphinxAtStartPar
\sphinxhyphen{}Cat {[}val{]} \sphinxhyphen{} mass of Cat+ added {[}kmol/m3{]}

\item {} 
\sphinxAtStartPar
\sphinxhyphen{}Vliq {[}val{]} \sphinxhyphen{} volume of liquid {[}m3{]}

\item {} 
\sphinxAtStartPar
\sphinxhyphen{}Vgas {[}val{]} \sphinxhyphen{} volume of liquid {[}m3{]}

\item {} 
\sphinxAtStartPar
\sphinxhyphen{}t\_resx {[}val{]} \sphinxhyphen{}SRT adjustment: t\_resx = SRT\sphinxhyphen{}HRT, {[}d{]} (works only for adm1f\_srt.cxx)

\item {} 
\sphinxAtStartPar
\sphinxhyphen{}params\_file {[}filename{]} \sphinxhyphen{} specify params filename (default is params.dat)

\item {} 
\sphinxAtStartPar
\sphinxhyphen{}ic\_file {[}filename{]} \sphinxhyphen{} specify initial conditions filename (default is ic.dat)

\item {} 
\sphinxAtStartPar
\sphinxhyphen{}influent\_file {[}filename{]} \sphinxhyphen{} specify influent filename (default is influent.dat)

\item {} 
\sphinxAtStartPar
\sphinxhyphen{}ts\_monitor \sphinxhyphen{} shows the timestep and time information on screen

\item {} 
\sphinxAtStartPar
\sphinxhyphen{}steady \sphinxhyphen{} run as steady state else runs as transient

\item {} 
\sphinxAtStartPar
\sphinxhyphen{}debug \sphinxhyphen{} gives out more details on the screen

\end{itemize}

\item {} 
\sphinxAtStartPar
More command\sphinxhyphen{}line options can be found \sphinxhref{https://www.mcs.anl.gov/petsc/petsc-current/docs/manualpages/TS/TSSetFromOptions.html}{here}.

\end{enumerate}


\chapter{ADM1F SRT}
\label{\detokenize{compile:adm1f-srt}}\label{\detokenize{compile:id1}}
\sphinxAtStartPar
The adm1f\_srt.cxx version includes solid retention time (SRT) and other modifications described below. To switch to the SRT version of the model change ‘EXAMPLESC  = adm1f\_srt.cxx’ , ‘OBJECTS\_PF = adm1f\_srt.o’, and ‘adm1f: adm1f\_srt.o’ in the \sphinxtitleref{build/makefile}. Then recompile the model (Compile ADM1F, step 6).
\begin{itemize}
\item {} 
\sphinxAtStartPar
Includes a term (T\_resx) in the mass balance to separate the solids retention time from hydraulic retention time.

\item {} 
\sphinxAtStartPar
Uses the empirical Hill function that describes the inhibition of acetogenesis and hydrogenotrophic methanogenesis by acetic acid with the noncompetitive inhibition model %
\begin{footnote}[1]\sphinxAtStartFootnote
Love, N. G., R. J. Smith, K. R. Gilmore, and C. W. Randall. 1999. Oxime inhibition of nitrification during treatment of an ammonia\sphinxhyphen{}containing industrial waste. Water Environment Research 71:418\textendash{}26.
%
\end{footnote} %
\begin{footnote}[2]\sphinxAtStartFootnote
Oslislo, A., and Z. Lewandowski. 1985. Inhibition of nitrification in the packed bed reactors by selected organic compounds. Water Research 19:423\textendash{}26.
%
\end{footnote}.

\item {} 
\sphinxAtStartPar
Describes the inhibition of acetic acid on acetoclastic methanogenesis with the Haldane equation %
\begin{footnote}[3]\sphinxAtStartFootnote
Haldane, J. B. S. 1930. Enzymes. London: Longmans.
%
\end{footnote} %
\begin{footnote}[4]\sphinxAtStartFootnote
Andrews, J. F. 1968. A mathematical model for the continuous culture of microorganisms utilizing inhibitory substrates. Biotechnology and Bioengineering 10:707\textendash{}23.
%
\end{footnote}.

\item {} 
\sphinxAtStartPar
Includes a adsorption\sphinxhyphen{}inhibition term describing the long\sphinxhyphen{}chain fatty acid %
\begin{footnote}[6]\sphinxAtStartFootnote
Palatsi, J., Illa, J., Prenafeta\sphinxhyphen{}Boldú, F.X., Laureni, M., Fernandez, B., Angelidaki, I., Flotats. X. 2010. Long\sphinxhyphen{}chain fatty acids inhibition and adaptation process in anaerobic thermophilic digestion: Batch tests, microbial community structure and mathematical modelling. Bioresource Technology. 101, 7, 2243\sphinxhyphen{}2251.
%
\end{footnote} (LCFA) inhibition of LCFA degradation and methanogenesis.

\item {} 
\sphinxAtStartPar
Includes Arrhenius equations describing the effect of temperature on bioreaction kinetics %
\begin{footnote}[5]\sphinxAtStartFootnote
Novak, J. T. 1974. Temperature\sphinxhyphen{}substrate interactions in biological treatment. Journal, Water Pollution Control Federation 46:1984\textendash{}94.
%
\end{footnote}.

\item {} 
\sphinxAtStartPar
Includes a cation term to simulate the addition of NaOH for pH adjustment.

\end{itemize}
\subsubsection*{References}


\chapter{Inputs/Outputs}
\label{\detokenize{inouts:inputs-outputs}}\label{\detokenize{inouts:inouts-label}}\label{\detokenize{inouts::doc}}

\section{Initial Conditions {[}ic.dat{]}}
\label{\detokenize{inouts:initial-conditions-ic-dat}}

\begin{savenotes}\sphinxatlongtablestart\begin{longtable}[c]{|\X{10}{120}|\X{20}{120}|\X{30}{120}|\X{60}{120}|}
\hline
\sphinxstyletheadfamily 
\sphinxAtStartPar
Index
&\sphinxstyletheadfamily 
\sphinxAtStartPar
Notation
&\sphinxstyletheadfamily 
\sphinxAtStartPar
Unit
&\sphinxstyletheadfamily 
\sphinxAtStartPar
Description
\\
\hline
\endfirsthead

\multicolumn{4}{c}%
{\makebox[0pt]{\sphinxtablecontinued{\tablename\ \thetable{} \textendash{} continued from previous page}}}\\
\hline
\sphinxstyletheadfamily 
\sphinxAtStartPar
Index
&\sphinxstyletheadfamily 
\sphinxAtStartPar
Notation
&\sphinxstyletheadfamily 
\sphinxAtStartPar
Unit
&\sphinxstyletheadfamily 
\sphinxAtStartPar
Description
\\
\hline
\endhead

\hline
\multicolumn{4}{r}{\makebox[0pt][r]{\sphinxtablecontinued{continues on next page}}}\\
\endfoot

\endlastfoot

\sphinxAtStartPar
1
&
\sphinxAtStartPar
S\_su
&
\sphinxAtStartPar
kgCOD/m3
&
\sphinxAtStartPar
soluble  monosaccharides
\\
\hline
\sphinxAtStartPar
2
&
\sphinxAtStartPar
S\_aa
&
\sphinxAtStartPar
kgCOD/m3
&
\sphinxAtStartPar
soluble  amino acids
\\
\hline
\sphinxAtStartPar
3
&
\sphinxAtStartPar
S\_fa
&
\sphinxAtStartPar
kgCOD/m3
&
\sphinxAtStartPar
soluble  total LCFA
\\
\hline
\sphinxAtStartPar
4
&
\sphinxAtStartPar
S\_va
&
\sphinxAtStartPar
kgCOD/m3
&
\sphinxAtStartPar
soluble  total valerate
\\
\hline
\sphinxAtStartPar
5
&
\sphinxAtStartPar
S\_bu
&
\sphinxAtStartPar
kgCOD/m3
&
\sphinxAtStartPar
soluble  total butyrate
\\
\hline
\sphinxAtStartPar
6
&
\sphinxAtStartPar
S\_pro
&
\sphinxAtStartPar
kgCOD/m3
&
\sphinxAtStartPar
soluble  total propionate
\\
\hline
\sphinxAtStartPar
7
&
\sphinxAtStartPar
S\_ac
&
\sphinxAtStartPar
kgCOD/m3
&
\sphinxAtStartPar
soluble  acetate
\\
\hline
\sphinxAtStartPar
8
&
\sphinxAtStartPar
S\_h2
&
\sphinxAtStartPar
kgCOD/m3
&
\sphinxAtStartPar
hydrogen gas
\\
\hline
\sphinxAtStartPar
9
&
\sphinxAtStartPar
S\_ch4
&
\sphinxAtStartPar
kgCOD/m3
&
\sphinxAtStartPar
methane gas
\\
\hline
\sphinxAtStartPar
10
&
\sphinxAtStartPar
S\_IC
&
\sphinxAtStartPar
kmoleC/m3
&
\sphinxAtStartPar
soluble inorganic carbon
\\
\hline
\sphinxAtStartPar
11
&
\sphinxAtStartPar
S\_IN
&
\sphinxAtStartPar
kmoleC/m3
&
\sphinxAtStartPar
soluble inorganic nitrogen
\\
\hline
\sphinxAtStartPar
12
&
\sphinxAtStartPar
S\_I
&
\sphinxAtStartPar
kgCOD/m3
&
\sphinxAtStartPar
soluble inert materials
\\
\hline
\sphinxAtStartPar
13
&
\sphinxAtStartPar
X\_c\_biom
&
\sphinxAtStartPar
kgCOD/m3
&
\sphinxAtStartPar
particulate  of composites
\\
\hline
\sphinxAtStartPar
14
&
\sphinxAtStartPar
X\_ch\_biom
&
\sphinxAtStartPar
kgCOD/m3
&
\sphinxAtStartPar
particulate  of carbohydrate
\\
\hline
\sphinxAtStartPar
15
&
\sphinxAtStartPar
X\_pr\_biom
&
\sphinxAtStartPar
kgCOD/m3
&
\sphinxAtStartPar
particulate  of proteins
\\
\hline
\sphinxAtStartPar
16
&
\sphinxAtStartPar
X\_li\_biom
&
\sphinxAtStartPar
kgCOD/m3
&
\sphinxAtStartPar
particulate  of lipids
\\
\hline
\sphinxAtStartPar
17
&
\sphinxAtStartPar
X\_su
&
\sphinxAtStartPar
kgCOD/m3
&
\sphinxAtStartPar
monosaccharides degraders (microorganisms)
\\
\hline
\sphinxAtStartPar
18
&
\sphinxAtStartPar
X\_aa
&
\sphinxAtStartPar
kgCOD/m3
&
\sphinxAtStartPar
amino acids degraders (microorganisms)
\\
\hline
\sphinxAtStartPar
19
&
\sphinxAtStartPar
X\_fa
&
\sphinxAtStartPar
kgCOD/m3
&
\sphinxAtStartPar
LCFA degraders (microorganisms)
\\
\hline
\sphinxAtStartPar
20
&
\sphinxAtStartPar
X\_c4
&
\sphinxAtStartPar
kgCOD/m3
&
\sphinxAtStartPar
valerate and butyrate degraders (microorganisms)
\\
\hline
\sphinxAtStartPar
21
&
\sphinxAtStartPar
X\_pro
&
\sphinxAtStartPar
kgCOD/m3
&
\sphinxAtStartPar
propionate degraders (microorganisms)
\\
\hline
\sphinxAtStartPar
22
&
\sphinxAtStartPar
X\_ac
&
\sphinxAtStartPar
kgCOD/m3
&
\sphinxAtStartPar
acetate degraders (microorganisms)
\\
\hline
\sphinxAtStartPar
23
&
\sphinxAtStartPar
X\_h2
&
\sphinxAtStartPar
kgCOD/m3
&
\sphinxAtStartPar
hydrogen degraders (microorganisms)
\\
\hline
\sphinxAtStartPar
24
&
\sphinxAtStartPar
X\_I
&
\sphinxAtStartPar
kgCOD/m3
&
\sphinxAtStartPar
particulate  of inerts
\\
\hline
\sphinxAtStartPar
25
&
\sphinxAtStartPar
S\_cation
&
\sphinxAtStartPar
kmole/m3
&
\sphinxAtStartPar
cations (strong base)
\\
\hline
\sphinxAtStartPar
26
&
\sphinxAtStartPar
S\_anion
&
\sphinxAtStartPar
kmole/m3
&
\sphinxAtStartPar
anions (strong acid)
\\
\hline
\sphinxAtStartPar
27
&
\sphinxAtStartPar
S\_hva
&
\sphinxAtStartPar
kgCOD/m3
&
\sphinxAtStartPar
soluble  valerate acid
\\
\hline
\sphinxAtStartPar
28
&
\sphinxAtStartPar
S\_hbu
&
\sphinxAtStartPar
kgCOD/m3
&
\sphinxAtStartPar
soluble  butyratic acid
\\
\hline
\sphinxAtStartPar
29
&
\sphinxAtStartPar
S\_hpro
&
\sphinxAtStartPar
kgCOD/m3
&
\sphinxAtStartPar
soluble  propionatic acid
\\
\hline
\sphinxAtStartPar
30
&
\sphinxAtStartPar
S\_hac
&
\sphinxAtStartPar
kgCOD/m3
&
\sphinxAtStartPar
soluble  acetatic acid
\\
\hline
\sphinxAtStartPar
31
&
\sphinxAtStartPar
S\_hco3
&
\sphinxAtStartPar
kmole/m3
&
\sphinxAtStartPar
soluble  bicarbonate
\\
\hline
\sphinxAtStartPar
32
&
\sphinxAtStartPar
S\_nh3
&
\sphinxAtStartPar
kmole/m3
&
\sphinxAtStartPar
soluble  ammonia
\\
\hline
\sphinxAtStartPar
33
&
\sphinxAtStartPar
S\_gas\_h2
&
\sphinxAtStartPar
kgCOD/m3
&
\sphinxAtStartPar
soluble  hydrogen gas
\\
\hline
\sphinxAtStartPar
34
&
\sphinxAtStartPar
S\_gas\_ch4
&
\sphinxAtStartPar
kgCOD/m3
&
\sphinxAtStartPar
soluble  methane gas
\\
\hline
\sphinxAtStartPar
35
&
\sphinxAtStartPar
S\_gas\_co2
&
\sphinxAtStartPar
kmole/m3
&
\sphinxAtStartPar
soluble  carbon dioxide gas
\\
\hline
\sphinxAtStartPar
36
&
\sphinxAtStartPar
Q
&
\sphinxAtStartPar
m3/d
&
\sphinxAtStartPar
flow rate
\\
\hline
\sphinxAtStartPar
37
&
\sphinxAtStartPar
Temp
&
\sphinxAtStartPar
°C
&
\sphinxAtStartPar
temperature
\\
\hline
\sphinxAtStartPar
38
&
\sphinxAtStartPar
S\_D1\_D
&
\sphinxAtStartPar
unitless
&
\sphinxAtStartPar
Dummy
\\
\hline
\sphinxAtStartPar
39
&
\sphinxAtStartPar
S\_D2\_D
&
\sphinxAtStartPar
unitless
&
\sphinxAtStartPar
Dummy
\\
\hline
\sphinxAtStartPar
40
&
\sphinxAtStartPar
S\_D3\_D
&
\sphinxAtStartPar
unitless
&
\sphinxAtStartPar
Dummy
\\
\hline
\sphinxAtStartPar
41
&
\sphinxAtStartPar
X\_D4\_D
&
\sphinxAtStartPar
unitless
&
\sphinxAtStartPar
Dummy
\\
\hline
\sphinxAtStartPar
42
&
\sphinxAtStartPar
X\_D5\_D
&
\sphinxAtStartPar
unitless
&
\sphinxAtStartPar
Dummy
\\
\hline
\sphinxAtStartPar
43
&
\sphinxAtStartPar
S\_H\_ion
&
\sphinxAtStartPar
kmoleH+/m3
&
\sphinxAtStartPar
soluble  hydrogen ion
\\
\hline
\sphinxAtStartPar
44
&
\sphinxAtStartPar
S\_co2
&
\sphinxAtStartPar
kmoleC/m3
&
\sphinxAtStartPar
soluble  carbon dioxide
\\
\hline
\sphinxAtStartPar
45
&
\sphinxAtStartPar
S\_nh4
&
\sphinxAtStartPar
kmoleN/m3
&
\sphinxAtStartPar
soluble  ammonium
\\
\hline
\end{longtable}\sphinxatlongtableend\end{savenotes}


\section{Influent Data {[}influent.dat{]}}
\label{\detokenize{inouts:influent-data-influent-dat}}

\begin{savenotes}\sphinxattablestart
\centering
\begin{tabular}[t]{|\X{10}{120}|\X{20}{120}|\X{30}{120}|\X{60}{120}|}
\hline
\sphinxstyletheadfamily 
\sphinxAtStartPar
Index
&\sphinxstyletheadfamily 
\sphinxAtStartPar
Notation
&\sphinxstyletheadfamily 
\sphinxAtStartPar
Unit
&\sphinxstyletheadfamily 
\sphinxAtStartPar
Description
\\
\hline
\sphinxAtStartPar
1
&
\sphinxAtStartPar
S\_su\_in
&
\sphinxAtStartPar
kgCOD/m3
&
\sphinxAtStartPar
soluble  input monosaccharides
\\
\hline
\sphinxAtStartPar
2
&
\sphinxAtStartPar
S\_aa\_in
&
\sphinxAtStartPar
kgCOD/m3
&
\sphinxAtStartPar
soluble  input amino acids
\\
\hline
\sphinxAtStartPar
3
&
\sphinxAtStartPar
S\_fa\_in
&
\sphinxAtStartPar
kgCOD/m3
&
\sphinxAtStartPar
soluble  input total LCFA
\\
\hline
\sphinxAtStartPar
4
&
\sphinxAtStartPar
S\_va\_in
&
\sphinxAtStartPar
kgCOD/m3
&
\sphinxAtStartPar
soluble  input total valerate
\\
\hline
\sphinxAtStartPar
5
&
\sphinxAtStartPar
S\_bu\_in
&
\sphinxAtStartPar
kgCOD/m3
&
\sphinxAtStartPar
soluble  input total butyrate
\\
\hline
\sphinxAtStartPar
6
&
\sphinxAtStartPar
S\_pro\_in
&
\sphinxAtStartPar
kgCOD/m3
&
\sphinxAtStartPar
soluble  input total propionate
\\
\hline
\sphinxAtStartPar
7
&
\sphinxAtStartPar
S\_ac\_in
&
\sphinxAtStartPar
kgCOD/m3
&
\sphinxAtStartPar
soluble input acetate
\\
\hline
\sphinxAtStartPar
8
&
\sphinxAtStartPar
S\_h2\_in
&
\sphinxAtStartPar
kgCOD/m3
&
\sphinxAtStartPar
hydrogen gas
\\
\hline
\sphinxAtStartPar
9
&
\sphinxAtStartPar
S\_ch4\_in
&
\sphinxAtStartPar
kgCOD/m3
&
\sphinxAtStartPar
methane gas
\\
\hline
\sphinxAtStartPar
10
&
\sphinxAtStartPar
S\_IC\_in
&
\sphinxAtStartPar
kmoleC/m3
&
\sphinxAtStartPar
soluble input inorganic carbon
\\
\hline
\sphinxAtStartPar
11
&
\sphinxAtStartPar
S\_IN\_in
&
\sphinxAtStartPar
kmoleC/m3
&
\sphinxAtStartPar
soluble input inorganic nitrogen
\\
\hline
\sphinxAtStartPar
12
&
\sphinxAtStartPar
S\_I \_in
&
\sphinxAtStartPar
kgCOD/m3
&
\sphinxAtStartPar
soluble input inert materials
\\
\hline
\sphinxAtStartPar
13
&
\sphinxAtStartPar
X\_c\_biom\_in
&
\sphinxAtStartPar
kgCOD/m3
&
\sphinxAtStartPar
particulate input of composites
\\
\hline
\sphinxAtStartPar
14
&
\sphinxAtStartPar
X\_ch\_biom\_in
&
\sphinxAtStartPar
kgCOD/m3
&
\sphinxAtStartPar
particulate input of carbohydrate
\\
\hline
\sphinxAtStartPar
15
&
\sphinxAtStartPar
X\_pr\_biom\_in
&
\sphinxAtStartPar
kgCOD/m3
&
\sphinxAtStartPar
particulate input of proteins
\\
\hline
\sphinxAtStartPar
16
&
\sphinxAtStartPar
X\_li\_biom\_in
&
\sphinxAtStartPar
kgCOD/m3
&
\sphinxAtStartPar
particulate input of lipids
\\
\hline
\sphinxAtStartPar
17
&
\sphinxAtStartPar
X\_su\_in
&
\sphinxAtStartPar
kgCOD/m3
&
\sphinxAtStartPar
monosaccharides degraders (microorganisms)
\\
\hline
\sphinxAtStartPar
18
&
\sphinxAtStartPar
X\_aa\_in
&
\sphinxAtStartPar
kgCOD/m3
&
\sphinxAtStartPar
amino acids degraders (microorganisms)
\\
\hline
\sphinxAtStartPar
19
&
\sphinxAtStartPar
X\_fa\_in
&
\sphinxAtStartPar
kgCOD/m3
&
\sphinxAtStartPar
LCFA degraders (microorganisms)
\\
\hline
\sphinxAtStartPar
20
&
\sphinxAtStartPar
X\_c4\_in
&
\sphinxAtStartPar
kgCOD/m3
&
\sphinxAtStartPar
valerate and butyrate degraders (microorganisms)
\\
\hline
\sphinxAtStartPar
21
&
\sphinxAtStartPar
X\_pro\_in
&
\sphinxAtStartPar
kgCOD/m3
&
\sphinxAtStartPar
propionate degraders (microorganisms)
\\
\hline
\sphinxAtStartPar
22
&
\sphinxAtStartPar
X\_ac\_in
&
\sphinxAtStartPar
kgCOD/m3
&
\sphinxAtStartPar
acetate degraders (microorganisms)
\\
\hline
\sphinxAtStartPar
23
&
\sphinxAtStartPar
X\_h2\_in
&
\sphinxAtStartPar
kgCOD/m3
&
\sphinxAtStartPar
hydrogen degraders (microorganisms)
\\
\hline
\sphinxAtStartPar
24
&
\sphinxAtStartPar
X\_I\_in
&
\sphinxAtStartPar
kgCOD/m3
&
\sphinxAtStartPar
particulate input of inerts
\\
\hline
\sphinxAtStartPar
25
&
\sphinxAtStartPar
S\_cation\_in
&
\sphinxAtStartPar
kmole/m3
&
\sphinxAtStartPar
input cations
\\
\hline
\sphinxAtStartPar
26
&
\sphinxAtStartPar
S\_anion\_in
&
\sphinxAtStartPar
kmole/m3
&
\sphinxAtStartPar
input anions
\\
\hline
\sphinxAtStartPar
27
&
\sphinxAtStartPar
Q
&
\sphinxAtStartPar
m3/d
&
\sphinxAtStartPar
flow rate
\\
\hline
\sphinxAtStartPar
28
&
\sphinxAtStartPar
Temp
&
\sphinxAtStartPar
°C
&
\sphinxAtStartPar
temperature
\\
\hline
\end{tabular}
\par
\sphinxattableend\end{savenotes}


\section{Parameters {[}params.dat{]}}
\label{\detokenize{inouts:parameters-params-dat}}

\begin{savenotes}\sphinxatlongtablestart\begin{longtable}[c]{|\X{10}{120}|\X{20}{120}|\X{30}{120}|\X{60}{120}|}
\hline
\sphinxstyletheadfamily 
\sphinxAtStartPar
Index
&\sphinxstyletheadfamily 
\sphinxAtStartPar
Notation
&\sphinxstyletheadfamily 
\sphinxAtStartPar
Unit
&\sphinxstyletheadfamily 
\sphinxAtStartPar
Description
\\
\hline
\endfirsthead

\multicolumn{4}{c}%
{\makebox[0pt]{\sphinxtablecontinued{\tablename\ \thetable{} \textendash{} continued from previous page}}}\\
\hline
\sphinxstyletheadfamily 
\sphinxAtStartPar
Index
&\sphinxstyletheadfamily 
\sphinxAtStartPar
Notation
&\sphinxstyletheadfamily 
\sphinxAtStartPar
Unit
&\sphinxstyletheadfamily 
\sphinxAtStartPar
Description
\\
\hline
\endhead

\hline
\multicolumn{4}{r}{\makebox[0pt][r]{\sphinxtablecontinued{continues on next page}}}\\
\endfoot

\endlastfoot

\sphinxAtStartPar
1
&
\sphinxAtStartPar
f\_sI\_xc
&
\sphinxAtStartPar
kgCOD/kg COD
&
\sphinxAtStartPar
fraction of composites (substrate) disintegrate to soluble inerts (product)
\\
\hline
\sphinxAtStartPar
2
&
\sphinxAtStartPar
f\_xI\_xc
&
\sphinxAtStartPar
kgCOD/kg COD
&
\sphinxAtStartPar
fraction of composites (substrate) disintegrate to particulate inerts (product)
\\
\hline
\sphinxAtStartPar
3
&
\sphinxAtStartPar
f\_ch\_xc
&
\sphinxAtStartPar
kgCOD/kg COD
&
\sphinxAtStartPar
fraction of composites (substrate) disintegrate to carbohydrates (product)
\\
\hline
\sphinxAtStartPar
4
&
\sphinxAtStartPar
f\_pr\_xc
&
\sphinxAtStartPar
kgCOD/kg COD
&
\sphinxAtStartPar
fraction of composites (substrate) disintegrate to proteins (product)
\\
\hline
\sphinxAtStartPar
5
&
\sphinxAtStartPar
f\_li\_xc
&
\sphinxAtStartPar
kgCOD/kg COD
&
\sphinxAtStartPar
fraction of composites (substrate) disintegrate to lipids (product)
\\
\hline
\sphinxAtStartPar
6
&
\sphinxAtStartPar
N\_xc
&
\sphinxAtStartPar
kmole N/(kg COD)
&
\sphinxAtStartPar
nitrogen content of  composites
\\
\hline
\sphinxAtStartPar
7
&
\sphinxAtStartPar
N\_I
&
\sphinxAtStartPar
kmole N/(kg COD)
&
\sphinxAtStartPar
nitrogen content of  inerts
\\
\hline
\sphinxAtStartPar
8
&
\sphinxAtStartPar
N\_aa
&
\sphinxAtStartPar
kmole N/(kg COD)
&
\sphinxAtStartPar
nitrogen content of  amino acids
\\
\hline
\sphinxAtStartPar
9
&
\sphinxAtStartPar
C\_xc
&
\sphinxAtStartPar
kmole C/(kg COD)
&
\sphinxAtStartPar
carbon content of  composites
\\
\hline
\sphinxAtStartPar
10
&
\sphinxAtStartPar
C\_sI
&
\sphinxAtStartPar
kmole C/(kg COD)
&
\sphinxAtStartPar
carbon content of  soluble inerts
\\
\hline
\sphinxAtStartPar
11
&
\sphinxAtStartPar
C\_ch
&
\sphinxAtStartPar
kmole C/(kg COD)
&
\sphinxAtStartPar
carbon content of  carbohydrates
\\
\hline
\sphinxAtStartPar
12
&
\sphinxAtStartPar
C\_pr
&
\sphinxAtStartPar
kmole C/(kg COD)
&
\sphinxAtStartPar
carbon content of  proteins
\\
\hline
\sphinxAtStartPar
13
&
\sphinxAtStartPar
C\_li
&
\sphinxAtStartPar
kmole C/(kg COD)
&
\sphinxAtStartPar
carbon content of  lipids
\\
\hline
\sphinxAtStartPar
14
&
\sphinxAtStartPar
C\_xI
&
\sphinxAtStartPar
kmole C/(kg COD)
&
\sphinxAtStartPar
carbon content of  particulate inerts
\\
\hline
\sphinxAtStartPar
15
&
\sphinxAtStartPar
C\_su
&
\sphinxAtStartPar
kmole C/(kg COD)
&
\sphinxAtStartPar
carbon content of  monosaccharides
\\
\hline
\sphinxAtStartPar
16
&
\sphinxAtStartPar
C\_aa
&
\sphinxAtStartPar
kmole C/(kg COD)
&
\sphinxAtStartPar
carbon content of  amino acids
\\
\hline
\sphinxAtStartPar
17
&
\sphinxAtStartPar
f\_fa\_li
&
\sphinxAtStartPar
kgCOD/kg COD
&
\sphinxAtStartPar
fraction of lipids (substrate) degrade to LCFA (product)
\\
\hline
\sphinxAtStartPar
18
&
\sphinxAtStartPar
C\_fa
&
\sphinxAtStartPar
kmole C/(kg COD)
&
\sphinxAtStartPar
carbon content of  total LCFA
\\
\hline
\sphinxAtStartPar
19
&
\sphinxAtStartPar
f\_h2\_su
&
\sphinxAtStartPar
kgCOD/kg COD
&
\sphinxAtStartPar
fraction of monosaccharides (substrate) degrade to hydrogen (product)
\\
\hline
\sphinxAtStartPar
20
&
\sphinxAtStartPar
f\_bu\_su
&
\sphinxAtStartPar
kgCOD/kg COD
&
\sphinxAtStartPar
fraction of monosaccharides (substrate) degrade to butyrate (product)
\\
\hline
\sphinxAtStartPar
21
&
\sphinxAtStartPar
f\_pro\_su
&
\sphinxAtStartPar
kgCOD/kg COD
&
\sphinxAtStartPar
fraction of monosaccharides (substrate) degrade to propionate (product)
\\
\hline
\sphinxAtStartPar
22
&
\sphinxAtStartPar
f\_ac\_su
&
\sphinxAtStartPar
kgCOD/kg COD
&
\sphinxAtStartPar
fraction of monosaccharides (substrate) degrade to acetate (product)
\\
\hline
\sphinxAtStartPar
23
&
\sphinxAtStartPar
N\_bac
&
\sphinxAtStartPar
kmole N/(kg COD)
&
\sphinxAtStartPar
nitrogen content of synthesized into bacteria
\\
\hline
\sphinxAtStartPar
24
&
\sphinxAtStartPar
C\_bu
&
\sphinxAtStartPar
kmole C/(kg COD)
&
\sphinxAtStartPar
carbon content of  total butyrate
\\
\hline
\sphinxAtStartPar
25
&
\sphinxAtStartPar
C\_pro
&
\sphinxAtStartPar
kmole C/(kg COD)
&
\sphinxAtStartPar
carbon content of  total propionate
\\
\hline
\sphinxAtStartPar
26
&
\sphinxAtStartPar
C\_ac
&
\sphinxAtStartPar
kmole C/(kg COD)
&
\sphinxAtStartPar
carbon content of  total acetate
\\
\hline
\sphinxAtStartPar
27
&
\sphinxAtStartPar
C\_bac
&
\sphinxAtStartPar
kmole C/(kg COD)
&
\sphinxAtStartPar
carbon content of  synthesized into bacteria
\\
\hline
\sphinxAtStartPar
28
&
\sphinxAtStartPar
Y\_su
&
\sphinxAtStartPar
kgCOD\_X/kg COD\_S
&
\sphinxAtStartPar
yield of biomass on monosaccharides (substrate)
\\
\hline
\sphinxAtStartPar
29
&
\sphinxAtStartPar
f\_h2\_aa
&
\sphinxAtStartPar
kgCOD/kg COD
&
\sphinxAtStartPar
fraction of amino acids (substrate) degrade to hydrogen (product)
\\
\hline
\sphinxAtStartPar
30
&
\sphinxAtStartPar
f\_va\_aa
&
\sphinxAtStartPar
kgCOD/kg COD
&
\sphinxAtStartPar
fraction of amino acids (substrate) degrade to valerate (product)
\\
\hline
\sphinxAtStartPar
31
&
\sphinxAtStartPar
f\_bu\_aa
&
\sphinxAtStartPar
kgCOD/kg COD
&
\sphinxAtStartPar
fraction of amino acids (substrate) degrade to butyrate (product)
\\
\hline
\sphinxAtStartPar
32
&
\sphinxAtStartPar
f\_pro\_aa
&
\sphinxAtStartPar
kgCOD/kg COD
&
\sphinxAtStartPar
fraction of amino acids (substrate) degrade to propionate (product)
\\
\hline
\sphinxAtStartPar
33
&
\sphinxAtStartPar
f\_ac\_aa
&
\sphinxAtStartPar
kgCOD/kg COD
&
\sphinxAtStartPar
fraction of amino acids (substrate) degrade to acetate (product)
\\
\hline
\sphinxAtStartPar
34
&
\sphinxAtStartPar
C\_va
&
\sphinxAtStartPar
kmole C/(kg COD)
&
\sphinxAtStartPar
carbon content of  total valerate
\\
\hline
\sphinxAtStartPar
35
&
\sphinxAtStartPar
Y\_aa
&
\sphinxAtStartPar
kgCOD\_X/kg COD\_S
&
\sphinxAtStartPar
yield of biomass on amino acids (substrate)
\\
\hline
\sphinxAtStartPar
36
&
\sphinxAtStartPar
Y\_fa
&
\sphinxAtStartPar
kgCOD\_X/kg COD\_S
&
\sphinxAtStartPar
yield of biomass on total LCFA (substrate)
\\
\hline
\sphinxAtStartPar
37
&
\sphinxAtStartPar
Y\_c4
&
\sphinxAtStartPar
kgCOD\_X/kg COD\_S
&
\sphinxAtStartPar
yield of biomass on butyrate (substrate)
\\
\hline
\sphinxAtStartPar
38
&
\sphinxAtStartPar
Y\_pro
&
\sphinxAtStartPar
kgCOD\_X/kg COD\_S
&
\sphinxAtStartPar
yield of biomass on total propionate (substrate)
\\
\hline
\sphinxAtStartPar
39
&
\sphinxAtStartPar
C\_ch4
&
\sphinxAtStartPar
kmole C/(kg COD)
&
\sphinxAtStartPar
carbon content of methane
\\
\hline
\sphinxAtStartPar
40
&
\sphinxAtStartPar
Y\_ac
&
\sphinxAtStartPar
kgCOD\_X/kg COD\_S
&
\sphinxAtStartPar
yield of biomass on total acetate (substrate)
\\
\hline
\sphinxAtStartPar
41
&
\sphinxAtStartPar
Y\_h2
&
\sphinxAtStartPar
kgCOD\_X/kg COD\_S
&
\sphinxAtStartPar
yield of biomass on hydrogen (substrate)
\\
\hline
\sphinxAtStartPar
42
&
\sphinxAtStartPar
k\_dis
&
\sphinxAtStartPar
1/d
&
\sphinxAtStartPar
disintegration rate
\\
\hline
\sphinxAtStartPar
43
&
\sphinxAtStartPar
k\_hyd\_ch
&
\sphinxAtStartPar
1/d
&
\sphinxAtStartPar
hydrolysis rate of carbohydrates (carbs to simple sugars)
\\
\hline
\sphinxAtStartPar
44
&
\sphinxAtStartPar
k\_hyd\_pr
&
\sphinxAtStartPar
1/d
&
\sphinxAtStartPar
hydrolysis rate of proteins
\\
\hline
\sphinxAtStartPar
45
&
\sphinxAtStartPar
k\_hyd\_li
&
\sphinxAtStartPar
1/d
&
\sphinxAtStartPar
hydrolysis rate of lipids
\\
\hline
\sphinxAtStartPar
46
&
\sphinxAtStartPar
K\_S\_IN
&
\sphinxAtStartPar
kgCOD\_S/m3
&
\sphinxAtStartPar
half saturation value of inorganic nitrogen
\\
\hline
\sphinxAtStartPar
47
&
\sphinxAtStartPar
k\_m\_su
&
\sphinxAtStartPar
kgCOD\_S/kgCOD\_X/d
&
\sphinxAtStartPar
Monod maximum specific uptake rate for monosaccharides
\\
\hline
\sphinxAtStartPar
48
&
\sphinxAtStartPar
K\_S\_su
&
\sphinxAtStartPar
kgCOD\_S/m3
&
\sphinxAtStartPar
half saturation value of monosaccharides
\\
\hline
\sphinxAtStartPar
49
&
\sphinxAtStartPar
pH\_UL\_acidacet
&
\sphinxAtStartPar
unitless
&
\sphinxAtStartPar
upper pH limit of acidic acetate
\\
\hline
\sphinxAtStartPar
50
&
\sphinxAtStartPar
pH\_LL\_acidacet
&
\sphinxAtStartPar
unitless
&
\sphinxAtStartPar
lower pH limit of acidic acetate
\\
\hline
\sphinxAtStartPar
51
&
\sphinxAtStartPar
k\_m\_aa
&
\sphinxAtStartPar
kgCOD\_S/kgCOD\_X/d
&
\sphinxAtStartPar
Monod maximum specific uptake rate for amino acids
\\
\hline
\sphinxAtStartPar
52
&
\sphinxAtStartPar
K\_S\_aa
&
\sphinxAtStartPar
kgCOD\_S/m3
&
\sphinxAtStartPar
half saturation value of amino acids
\\
\hline
\sphinxAtStartPar
53
&
\sphinxAtStartPar
k\_m\_fa
&
\sphinxAtStartPar
kgCOD\_S/kgCOD\_X/d
&
\sphinxAtStartPar
Monod maximum specific uptake rate for total LCFA
\\
\hline
\sphinxAtStartPar
54
&
\sphinxAtStartPar
K\_S\_fa
&
\sphinxAtStartPar
kgCOD\_S/m3
&
\sphinxAtStartPar
half saturation value of total LCFA
\\
\hline
\sphinxAtStartPar
55
&
\sphinxAtStartPar
K\_Ih2\_fa
&
\sphinxAtStartPar
kgCOD/m3
&
\sphinxAtStartPar
inhibition constant LCFA (substrate) degradation by hydrogen (inhibitor)
\\
\hline
\sphinxAtStartPar
56
&
\sphinxAtStartPar
k\_m\_c4
&
\sphinxAtStartPar
kgCOD\_S/kgCOD\_X/d
&
\sphinxAtStartPar
Monod maximum specific uptake rate for butyrate and valerate
\\
\hline
\sphinxAtStartPar
57
&
\sphinxAtStartPar
K\_S\_c4
&
\sphinxAtStartPar
kgCOD\_S/m3
&
\sphinxAtStartPar
half saturation value of butyrate
\\
\hline
\sphinxAtStartPar
58
&
\sphinxAtStartPar
K\_Ih2\_c4
&
\sphinxAtStartPar
kgCOD/m3
&
\sphinxAtStartPar
inhibition constant for butyrate and valerate (substrate) degradation by hydrogen (inhibitor)
\\
\hline
\sphinxAtStartPar
59
&
\sphinxAtStartPar
k\_m\_pro
&
\sphinxAtStartPar
kgCOD\_S/kgCOD\_X/d
&
\sphinxAtStartPar
Monod maximum specific uptake rate for total propionate
\\
\hline
\sphinxAtStartPar
60
&
\sphinxAtStartPar
K\_S\_pro
&
\sphinxAtStartPar
kgCOD\_S/m3
&
\sphinxAtStartPar
half saturation value of total propionate
\\
\hline
\sphinxAtStartPar
61
&
\sphinxAtStartPar
K\_Ih2\_pro
&
\sphinxAtStartPar
kgCOD/m3
&
\sphinxAtStartPar
inhibition constant for propionate (substrate) degradation by hydrogen (inhibitor)
\\
\hline
\sphinxAtStartPar
62
&
\sphinxAtStartPar
k\_m\_ac
&
\sphinxAtStartPar
kgCOD\_S/kgCOD\_X/d
&
\sphinxAtStartPar
Monod maximum specific uptake rate for total acetate
\\
\hline
\sphinxAtStartPar
63
&
\sphinxAtStartPar
K\_S\_ac
&
\sphinxAtStartPar
kgCOD\_S/m3
&
\sphinxAtStartPar
half saturation value of total acetate
\\
\hline
\sphinxAtStartPar
64
&
\sphinxAtStartPar
K\_I\_nh3
&
\sphinxAtStartPar
kmol N/m3
&
\sphinxAtStartPar
inhibition constant by ammonia (inhibitor)
\\
\hline
\sphinxAtStartPar
65
&
\sphinxAtStartPar
pH\_UL\_ac
&
\sphinxAtStartPar
unitless
&
\sphinxAtStartPar
upper pH limit for total acetate
\\
\hline
\sphinxAtStartPar
66
&
\sphinxAtStartPar
pH\_LL\_ac
&
\sphinxAtStartPar
unitless
&
\sphinxAtStartPar
lower pH limit for total acetate
\\
\hline
\sphinxAtStartPar
67
&
\sphinxAtStartPar
k\_m\_h2
&
\sphinxAtStartPar
kgCOD\_S/kgCOD\_X/d
&
\sphinxAtStartPar
Monod maximum specific uptake rate for hydrogen
\\
\hline
\sphinxAtStartPar
68
&
\sphinxAtStartPar
K\_S\_h2
&
\sphinxAtStartPar
kgCOD\_S/m3
&
\sphinxAtStartPar
half saturation value of hydrogen
\\
\hline
\sphinxAtStartPar
69
&
\sphinxAtStartPar
pH\_UL\_h2
&
\sphinxAtStartPar
unitless
&
\sphinxAtStartPar
upper pH limit for hydrogen
\\
\hline
\sphinxAtStartPar
70
&
\sphinxAtStartPar
pH\_LL\_h2
&
\sphinxAtStartPar
unitless
&
\sphinxAtStartPar
lower pH limit for hydrogen
\\
\hline
\sphinxAtStartPar
71
&
\sphinxAtStartPar
k\_dec\_Xsu
&
\sphinxAtStartPar
1/d
&
\sphinxAtStartPar
first order decay rate for the monosaccharide degraders
\\
\hline
\sphinxAtStartPar
72
&
\sphinxAtStartPar
k\_dec\_Xaa
&
\sphinxAtStartPar
1/d
&
\sphinxAtStartPar
first order decay rate for the amino acids degraders
\\
\hline
\sphinxAtStartPar
73
&
\sphinxAtStartPar
k\_dec\_Xfa
&
\sphinxAtStartPar
1/d
&
\sphinxAtStartPar
first order decay rate for the LCFA degraders
\\
\hline
\sphinxAtStartPar
74
&
\sphinxAtStartPar
k\_dec\_Xc4
&
\sphinxAtStartPar
1/d
&
\sphinxAtStartPar
first order decay rate for the butyrate and valerate
\\
\hline
\sphinxAtStartPar
75
&
\sphinxAtStartPar
k\_dec\_Xpro
&
\sphinxAtStartPar
1/d
&
\sphinxAtStartPar
first order decay rate for the propionate degraders
\\
\hline
\sphinxAtStartPar
76
&
\sphinxAtStartPar
k\_dec\_Xac
&
\sphinxAtStartPar
1/d
&
\sphinxAtStartPar
first order decay rate for the acetate degraders
\\
\hline
\sphinxAtStartPar
77
&
\sphinxAtStartPar
k\_dec\_Xh2
&
\sphinxAtStartPar
1/d
&
\sphinxAtStartPar
first order decay rate for the hydrogen degraders
\\
\hline
\sphinxAtStartPar
78
&
\sphinxAtStartPar
R
&
\sphinxAtStartPar
bar m3 kmole\sphinxhyphen{}1 K\sphinxhyphen{}1
&
\sphinxAtStartPar
gas law constant (8.314e\sphinxhyphen{}2)
\\
\hline
\sphinxAtStartPar
79
&
\sphinxAtStartPar
T\_base
&
\sphinxAtStartPar
°C
&
\sphinxAtStartPar
base temperature
\\
\hline
\sphinxAtStartPar
80
&
\sphinxAtStartPar
T\_op
&
\sphinxAtStartPar
°C
&
\sphinxAtStartPar
operating temperature
\\
\hline
\sphinxAtStartPar
81
&
\sphinxAtStartPar
pK\_w\_base
&
\sphinxAtStartPar
unitless
&
\sphinxAtStartPar
pKa of water
\\
\hline
\sphinxAtStartPar
82
&
\sphinxAtStartPar
pK\_a\_va\_base
&
\sphinxAtStartPar
unitless
&
\sphinxAtStartPar
pKa of total valerate
\\
\hline
\sphinxAtStartPar
83
&
\sphinxAtStartPar
pK\_a\_bu\_base
&
\sphinxAtStartPar
unitless
&
\sphinxAtStartPar
pKa of total butyrate
\\
\hline
\sphinxAtStartPar
84
&
\sphinxAtStartPar
pK\_a\_pro\_base
&
\sphinxAtStartPar
unitless
&
\sphinxAtStartPar
pKa of total propionate
\\
\hline
\sphinxAtStartPar
85
&
\sphinxAtStartPar
pK\_a\_ac\_base
&
\sphinxAtStartPar
unitless
&
\sphinxAtStartPar
pKa of total acetate
\\
\hline
\sphinxAtStartPar
86
&
\sphinxAtStartPar
pK\_a\_co2\_base
&
\sphinxAtStartPar
unitless
&
\sphinxAtStartPar
pKa of carbon dioxide
\\
\hline
\sphinxAtStartPar
87
&
\sphinxAtStartPar
pK\_a\_IN\_base
&
\sphinxAtStartPar
unitless
&
\sphinxAtStartPar
pKa of inorganic nitrogen
\\
\hline
\sphinxAtStartPar
88
&
\sphinxAtStartPar
pK\_a\_hco3\_base
&
\sphinxAtStartPar
unitless
&
\sphinxAtStartPar
pKa of bicarbonate
\\
\hline
\sphinxAtStartPar
89
&
\sphinxAtStartPar
k\_A\_Bbu
&
\sphinxAtStartPar
1/M/d
&
\sphinxAtStartPar
acid base kinetic parameter for total butyrate
\\
\hline
\sphinxAtStartPar
90
&
\sphinxAtStartPar
k\_A\_Bpro
&
\sphinxAtStartPar
1/M/d
&
\sphinxAtStartPar
acid base kinetic parameter for total propionate
\\
\hline
\sphinxAtStartPar
91
&
\sphinxAtStartPar
k\_A\_Bac
&
\sphinxAtStartPar
1/M/d
&
\sphinxAtStartPar
acid base kinetic parameter for total acetate
\\
\hline
\sphinxAtStartPar
92
&
\sphinxAtStartPar
k\_A\_Bco2
&
\sphinxAtStartPar
1/M/d
&
\sphinxAtStartPar
acid base kinetic parameter for carbon dioxide
\\
\hline
\sphinxAtStartPar
93
&
\sphinxAtStartPar
k\_A\_BIN
&
\sphinxAtStartPar
1/M/d
&
\sphinxAtStartPar
acid base kinetic parameter for inhibitors
\\
\hline
\sphinxAtStartPar
94
&
\sphinxAtStartPar
P\_atm
&
\sphinxAtStartPar
bar
&
\sphinxAtStartPar
atmospheric pressure
\\
\hline
\sphinxAtStartPar
95
&
\sphinxAtStartPar
kLa
&
\sphinxAtStartPar
1/d
&
\sphinxAtStartPar
gas\sphinxhyphen{}liquid transfer coefficient
\\
\hline
\sphinxAtStartPar
96
&
\sphinxAtStartPar
K\_H\_h2o\_base
&
\sphinxAtStartPar
M(liq)/bar
&
\sphinxAtStartPar
Henry’s law coefficient of water
\\
\hline
\sphinxAtStartPar
97
&
\sphinxAtStartPar
K\_H\_co2\_base
&
\sphinxAtStartPar
M(liq)/bar
&
\sphinxAtStartPar
Henry’s law coefficient of carbon dioxide
\\
\hline
\sphinxAtStartPar
98
&
\sphinxAtStartPar
K\_H\_ch4\_base
&
\sphinxAtStartPar
M(liq)/bar
&
\sphinxAtStartPar
Henry’s law coefficient of methane
\\
\hline
\sphinxAtStartPar
99
&
\sphinxAtStartPar
K\_H\_h2\_base
&
\sphinxAtStartPar
M(liq)/bar
&
\sphinxAtStartPar
Henry’s law coefficient of hydrogen
\\
\hline
\sphinxAtStartPar
100
&
\sphinxAtStartPar
k\_P
&
\sphinxAtStartPar
m2/d/bar
&
\sphinxAtStartPar
proportional gain
\\
\hline
\end{longtable}\sphinxatlongtableend\end{savenotes}


\section{Outputs {[}indicator***.out{]}}
\label{\detokenize{inouts:outputs-indicator-out}}

\begin{savenotes}\sphinxatlongtablestart\begin{longtable}[c]{|\X{10}{120}|\X{20}{120}|\X{30}{120}|\X{60}{120}|}
\hline
\sphinxstyletheadfamily 
\sphinxAtStartPar
Index
&\sphinxstyletheadfamily 
\sphinxAtStartPar
Notation
&\sphinxstyletheadfamily 
\sphinxAtStartPar
Unit
&\sphinxstyletheadfamily 
\sphinxAtStartPar
Description
\\
\hline
\endfirsthead

\multicolumn{4}{c}%
{\makebox[0pt]{\sphinxtablecontinued{\tablename\ \thetable{} \textendash{} continued from previous page}}}\\
\hline
\sphinxstyletheadfamily 
\sphinxAtStartPar
Index
&\sphinxstyletheadfamily 
\sphinxAtStartPar
Notation
&\sphinxstyletheadfamily 
\sphinxAtStartPar
Unit
&\sphinxstyletheadfamily 
\sphinxAtStartPar
Description
\\
\hline
\endhead

\hline
\multicolumn{4}{r}{\makebox[0pt][r]{\sphinxtablecontinued{continues on next page}}}\\
\endfoot

\endlastfoot

\sphinxAtStartPar
1
&
\sphinxAtStartPar
Ssu
&
\sphinxAtStartPar
mg COD/L
&
\sphinxAtStartPar
soluble  monosaccharides
\\
\hline
\sphinxAtStartPar
2
&
\sphinxAtStartPar
Saa
&
\sphinxAtStartPar
mg COD/L
&
\sphinxAtStartPar
soluble  amino acids
\\
\hline
\sphinxAtStartPar
3
&
\sphinxAtStartPar
Sfa
&
\sphinxAtStartPar
mg COD/L
&
\sphinxAtStartPar
soluble  total LCFA
\\
\hline
\sphinxAtStartPar
4
&
\sphinxAtStartPar
Sva
&
\sphinxAtStartPar
mg COD/L
&
\sphinxAtStartPar
soluble  total valerate
\\
\hline
\sphinxAtStartPar
5
&
\sphinxAtStartPar
Sbu
&
\sphinxAtStartPar
mg COD/L
&
\sphinxAtStartPar
soluble  total butyrate
\\
\hline
\sphinxAtStartPar
6
&
\sphinxAtStartPar
Spro
&
\sphinxAtStartPar
mg COD/L
&
\sphinxAtStartPar
soluble  total propionate
\\
\hline
\sphinxAtStartPar
7
&
\sphinxAtStartPar
Sac
&
\sphinxAtStartPar
mg COD/L
&
\sphinxAtStartPar
soluble  total acetate
\\
\hline
\sphinxAtStartPar
8
&
\sphinxAtStartPar
Sh2
&
\sphinxAtStartPar
mg COD/L
&
\sphinxAtStartPar
soluble  hydrogen
\\
\hline
\sphinxAtStartPar
9
&
\sphinxAtStartPar
Sch4
&
\sphinxAtStartPar
mg COD/L
&
\sphinxAtStartPar
soluble  methane
\\
\hline
\sphinxAtStartPar
10
&
\sphinxAtStartPar
Sic
&
\sphinxAtStartPar
mg C/L
&
\sphinxAtStartPar
soluble  inorganic carbon
\\
\hline
\sphinxAtStartPar
11
&
\sphinxAtStartPar
Sin
&
\sphinxAtStartPar
mg N/L
&
\sphinxAtStartPar
soluble  inorganic nitrogen
\\
\hline
\sphinxAtStartPar
12
&
\sphinxAtStartPar
Si
&
\sphinxAtStartPar
mg COD/L
&
\sphinxAtStartPar
soluble  inerts
\\
\hline
\sphinxAtStartPar
13
&
\sphinxAtStartPar
Xc
&
\sphinxAtStartPar
mg COD/L
&
\sphinxAtStartPar
particulate  composites
\\
\hline
\sphinxAtStartPar
14
&
\sphinxAtStartPar
Xch
&
\sphinxAtStartPar
mg COD/L
&
\sphinxAtStartPar
particulate  carbohydrates
\\
\hline
\sphinxAtStartPar
15
&
\sphinxAtStartPar
Xpr
&
\sphinxAtStartPar
mg COD/L
&
\sphinxAtStartPar
particulate  proteins
\\
\hline
\sphinxAtStartPar
16
&
\sphinxAtStartPar
Xli
&
\sphinxAtStartPar
mg COD/L
&
\sphinxAtStartPar
particulate  lipids
\\
\hline
\sphinxAtStartPar
17
&
\sphinxAtStartPar
Xsu
&
\sphinxAtStartPar
mg COD/L
&
\sphinxAtStartPar
monosaccharides degraders (microorganisms)
\\
\hline
\sphinxAtStartPar
18
&
\sphinxAtStartPar
Xaa
&
\sphinxAtStartPar
mg COD/L
&
\sphinxAtStartPar
amino acids degraders (microorganisms)
\\
\hline
\sphinxAtStartPar
19
&
\sphinxAtStartPar
Xfa
&
\sphinxAtStartPar
mg COD/L
&
\sphinxAtStartPar
LCFA degraders (microorganisms)
\\
\hline
\sphinxAtStartPar
20
&
\sphinxAtStartPar
Xc4
&
\sphinxAtStartPar
mg COD/L
&
\sphinxAtStartPar
butyrate and valerate  degraders (microorganisms)
\\
\hline
\sphinxAtStartPar
21
&
\sphinxAtStartPar
Xpro
&
\sphinxAtStartPar
mg COD/L
&
\sphinxAtStartPar
propionate degraders (microorganisms)
\\
\hline
\sphinxAtStartPar
22
&
\sphinxAtStartPar
Xac
&
\sphinxAtStartPar
mg COD/L
&
\sphinxAtStartPar
acetate degraders (microorganisms)
\\
\hline
\sphinxAtStartPar
23
&
\sphinxAtStartPar
Xh2
&
\sphinxAtStartPar
mg COD/L
&
\sphinxAtStartPar
hydrogen degraders (microorganisms)
\\
\hline
\sphinxAtStartPar
24
&
\sphinxAtStartPar
Xi
&
\sphinxAtStartPar
mg COD/L
&
\sphinxAtStartPar
particulate  inerts
\\
\hline
\sphinxAtStartPar
25
&
\sphinxAtStartPar
scat+
&
\sphinxAtStartPar
mmol/L
&
\sphinxAtStartPar
cations
\\
\hline
\sphinxAtStartPar
26
&
\sphinxAtStartPar
san\sphinxhyphen{}
&
\sphinxAtStartPar
mmol/L
&
\sphinxAtStartPar
anions
\\
\hline
\sphinxAtStartPar
27
&
\sphinxAtStartPar
pH
&
\sphinxAtStartPar
unitless
&
\sphinxAtStartPar
a scale used to specify how acidic or basic a water\sphinxhyphen{}based solution is
\\
\hline
\sphinxAtStartPar
28
&
\sphinxAtStartPar
S\_H+
&
\sphinxAtStartPar
mol/L
&
\sphinxAtStartPar
soluble hydrogen cation
\\
\hline
\sphinxAtStartPar
29
&
\sphinxAtStartPar
Sva\sphinxhyphen{}
&
\sphinxAtStartPar
mg COD/L
&
\sphinxAtStartPar
soluble total valerate anion
\\
\hline
\sphinxAtStartPar
30
&
\sphinxAtStartPar
Sbu\sphinxhyphen{}
&
\sphinxAtStartPar
mg COD/L
&
\sphinxAtStartPar
soluble total butyrate anion
\\
\hline
\sphinxAtStartPar
31
&
\sphinxAtStartPar
Spro\sphinxhyphen{}
&
\sphinxAtStartPar
mg COD/L
&
\sphinxAtStartPar
soluble total propionate anion
\\
\hline
\sphinxAtStartPar
32
&
\sphinxAtStartPar
Sac\sphinxhyphen{}
&
\sphinxAtStartPar
mg COD/L
&
\sphinxAtStartPar
soluble total acetate anion
\\
\hline
\sphinxAtStartPar
33
&
\sphinxAtStartPar
Shco3\sphinxhyphen{}
&
\sphinxAtStartPar
mg C/L
&
\sphinxAtStartPar
soluble bicarbonate anion
\\
\hline
\sphinxAtStartPar
34
&
\sphinxAtStartPar
Sco2
&
\sphinxAtStartPar
mg C/L
&
\sphinxAtStartPar
soluble carbon dioxide
\\
\hline
\sphinxAtStartPar
35
&
\sphinxAtStartPar
Snh3
&
\sphinxAtStartPar
mg N/L
&
\sphinxAtStartPar
soluble ammonia
\\
\hline
\sphinxAtStartPar
36
&
\sphinxAtStartPar
Snh4+
&
\sphinxAtStartPar
mg N/L
&
\sphinxAtStartPar
soluble ammonia cation (ammonium)
\\
\hline
\sphinxAtStartPar
37
&
\sphinxAtStartPar
Sgas,h2
&
\sphinxAtStartPar
mg COD/L
&
\sphinxAtStartPar
soluble hydrogen gas
\\
\hline
\sphinxAtStartPar
38
&
\sphinxAtStartPar
Sgas,ch4
&
\sphinxAtStartPar
mg COD/L
&
\sphinxAtStartPar
soluble methane gas
\\
\hline
\sphinxAtStartPar
39
&
\sphinxAtStartPar
Sgas,co2
&
\sphinxAtStartPar
mg C/L
&
\sphinxAtStartPar
soluble carbon dioxide gas
\\
\hline
\sphinxAtStartPar
40
&
\sphinxAtStartPar
pgas,h2
&
\sphinxAtStartPar
atm
&
\sphinxAtStartPar
partial pressure of gas hydrogen
\\
\hline
\sphinxAtStartPar
41
&
\sphinxAtStartPar
pgas,ch4
&
\sphinxAtStartPar
atm
&
\sphinxAtStartPar
partial pressure of gas methane
\\
\hline
\sphinxAtStartPar
42
&
\sphinxAtStartPar
pgas,co2
&
\sphinxAtStartPar
atm
&
\sphinxAtStartPar
partial pressure of gas carbon dioxide
\\
\hline
\sphinxAtStartPar
43
&
\sphinxAtStartPar
pgas,total
&
\sphinxAtStartPar
atm
&
\sphinxAtStartPar
partial pressure of gas all gases
\\
\hline
\sphinxAtStartPar
44
&
\sphinxAtStartPar
pgas
&
\sphinxAtStartPar
m3/d
&
\sphinxAtStartPar
flow rate of gas
\\
\hline
\sphinxAtStartPar
45
&
\sphinxAtStartPar
Si
&
\sphinxAtStartPar
mg COD/L
&
\sphinxAtStartPar
soluble inert organics
\\
\hline
\sphinxAtStartPar
46
&
\sphinxAtStartPar
Ss
&
\sphinxAtStartPar
mg COD/L
&
\sphinxAtStartPar
readily biodegradable substrate
\\
\hline
\sphinxAtStartPar
47
&
\sphinxAtStartPar
Xi
&
\sphinxAtStartPar
mg COD/L
&
\sphinxAtStartPar
particulate inert organics
\\
\hline
\sphinxAtStartPar
48
&
\sphinxAtStartPar
Xs
&
\sphinxAtStartPar
mg COD/L
&
\sphinxAtStartPar
slowly biodegradable substrate
\\
\hline
\sphinxAtStartPar
49
&
\sphinxAtStartPar
Xd
&
\sphinxAtStartPar
mg COD/L
&
\sphinxAtStartPar
particulate  arising from biomass decay decay
\\
\hline
\sphinxAtStartPar
50
&
\sphinxAtStartPar
Snh
&
\sphinxAtStartPar
mg N/L
&
\sphinxAtStartPar
ammonia and ammonium nitrogen, soluble  the ammonia produced during ammonification process from soluble organic nitrogen
\\
\hline
\sphinxAtStartPar
51
&
\sphinxAtStartPar
Sns
&
\sphinxAtStartPar
mg N/L
&
\sphinxAtStartPar
soluble biodegradable organic nitrogen generated during hydrolysis of particulate biodegradable organic nitrogen, suggesting it is the concentration of soluble organic hydrogen generated during hydrolysis
\\
\hline
\sphinxAtStartPar
52
&
\sphinxAtStartPar
Xns
&
\sphinxAtStartPar
mg N/L
&
\sphinxAtStartPar
particulate biodegradable organic nitrogen generated during hydrolysis of particulate biodegradable organic nitrogen, suggesting it is the concentration of soluble organic hydrogen generated during hydrolysis
\\
\hline
\sphinxAtStartPar
53
&
\sphinxAtStartPar
Salk
&
\sphinxAtStartPar
mg C/L
&
\sphinxAtStartPar
charge balance
\\
\hline
\sphinxAtStartPar
54
&
\sphinxAtStartPar
TSS
&
\sphinxAtStartPar
mg TSS/L
&
\sphinxAtStartPar
total suspended solids
\\
\hline
\sphinxAtStartPar
55
&
\sphinxAtStartPar
VFA\_C2toC5
&
\sphinxAtStartPar
mg COD/L
&
\sphinxAtStartPar
volatile fatty acid from C2 to C5
\\
\hline
\sphinxAtStartPar
56
&
\sphinxAtStartPar
mass\_Sac
&
\sphinxAtStartPar
mg Hac/L
&
\sphinxAtStartPar
acetic acid
\\
\hline
\sphinxAtStartPar
57
&
\sphinxAtStartPar
PAratio
&
\sphinxAtStartPar
kg acetate/ kg acetate equivalent of propionate
&
\sphinxAtStartPar
acetate propionate ratio
\\
\hline
\sphinxAtStartPar
58
&
\sphinxAtStartPar
Alk
&
\sphinxAtStartPar
mg/L CaCO3
&
\sphinxAtStartPar
alkalinity
\\
\hline
\sphinxAtStartPar
59
&
\sphinxAtStartPar
NH3
&
\sphinxAtStartPar
mg N/L
&
\sphinxAtStartPar
ammonia
\\
\hline
\sphinxAtStartPar
60
&
\sphinxAtStartPar
NH4
&
\sphinxAtStartPar
mg N/L
&
\sphinxAtStartPar
ammonium
\\
\hline
\sphinxAtStartPar
61
&
\sphinxAtStartPar
LCFA
&
\sphinxAtStartPar
mgCOD\_LCFA/L
&
\sphinxAtStartPar
long chain fatty acid
\\
\hline
\sphinxAtStartPar
62
&
\sphinxAtStartPar
percentch4
&
\sphinxAtStartPar
\%
&
\sphinxAtStartPar
biogas methane content, methane percentage output, percent by volume
\\
\hline
\sphinxAtStartPar
63
&
\sphinxAtStartPar
energych4
&
\sphinxAtStartPar
\%
&
\sphinxAtStartPar
energy content of CH4 gas, methane energy output, methane converted to COD, percentage of input that’s converted to CH4 energy wise
\\
\hline
\sphinxAtStartPar
64
&
\sphinxAtStartPar
efficiency
&
\sphinxAtStartPar
\%
&
\sphinxAtStartPar
COD removal
\\
\hline
\sphinxAtStartPar
65
&
\sphinxAtStartPar
VFA/ALK
&
\sphinxAtStartPar
g acetate eq./g CaCO3
&
\sphinxAtStartPar
volatile fatty acid to alkalinity ratio
\\
\hline
\sphinxAtStartPar
66
&
\sphinxAtStartPar
ACN
&
\sphinxAtStartPar
kg COD/m3/d
&
\sphinxAtStartPar
acetate capacity number, the ratio between the maximum acetate utilization rate and the average acetate production rate
\\
\hline
\sphinxAtStartPar
67
&
\sphinxAtStartPar
SampleT
&
\sphinxAtStartPar
d
&\\
\hline
\end{longtable}\sphinxatlongtableend\end{savenotes}


\chapter{Examples}
\label{\detokenize{examples:examples}}\label{\detokenize{examples::doc}}
\sphinxAtStartPar
All demo cases are documented in jupyter notebooks.


\section{ADM1F: Steady State}
\label{\detokenize{jupyter_notebook/Steady_State_Run:ADM1F:-Steady-State}}\label{\detokenize{jupyter_notebook/Steady_State_Run::doc}}
\sphinxAtStartPar
Here we run the steady state case and comparing it with the Matlab results. Make sure to compile \sphinxcode{\sphinxupquote{build/adm1f.cxx}}.

\sphinxAtStartPar
Author: Elchin Jafarov


\subsection{1. Steady State Run}
\label{\detokenize{jupyter_notebook/Steady_State_Run:1.-Steady-State-Run}}
{
\sphinxsetup{VerbatimColor={named}{nbsphinx-code-bg}}
\sphinxsetup{VerbatimBorderColor={named}{nbsphinx-code-border}}
\begin{sphinxVerbatim}[commandchars=\\\{\}]
\llap{\color{nbsphinxin}[1]:\,\hspace{\fboxrule}\hspace{\fboxsep}}\PYG{k+kn}{import} \PYG{n+nn}{os}
\PYG{k+kn}{import} \PYG{n+nn}{numpy} \PYG{k}{as} \PYG{n+nn}{np}
\PYG{k+kn}{import} \PYG{n+nn}{pandas} \PYG{k}{as} \PYG{n+nn}{pd}
\PYG{k+kn}{import} \PYG{n+nn}{subprocess}
\PYG{k+kn}{import} \PYG{n+nn}{sklearn}\PYG{n+nn}{.}\PYG{n+nn}{metrics} \PYG{k}{as} \PYG{n+nn}{sklm}
\PYG{k+kn}{import} \PYG{n+nn}{xlrd}
\PYG{k+kn}{import} \PYG{n+nn}{matplotlib}\PYG{n+nn}{.}\PYG{n+nn}{pyplot} \PYG{k}{as} \PYG{n+nn}{plt}
\PYG{o}{\PYGZpc{}}\PYG{k}{matplotlib} inline
\end{sphinxVerbatim}
}

{
\sphinxsetup{VerbatimColor={named}{nbsphinx-code-bg}}
\sphinxsetup{VerbatimBorderColor={named}{nbsphinx-code-border}}
\begin{sphinxVerbatim}[commandchars=\\\{\}]
\llap{\color{nbsphinxin}[2]:\,\hspace{\fboxrule}\hspace{\fboxsep}}\PYG{c+c1}{\PYGZsh{} navigate to simulations folder}
\PYG{n}{os}\PYG{o}{.}\PYG{n}{chdir}\PYG{p}{(}\PYG{l+s+s1}{\PYGZsq{}}\PYG{l+s+s1}{../../simulations}\PYG{l+s+s1}{\PYGZsq{}}\PYG{p}{)}
\end{sphinxVerbatim}
}

{
\sphinxsetup{VerbatimColor={named}{nbsphinx-code-bg}}
\sphinxsetup{VerbatimBorderColor={named}{nbsphinx-code-border}}
\begin{sphinxVerbatim}[commandchars=\\\{\}]
\llap{\color{nbsphinxin}[3]:\,\hspace{\fboxrule}\hspace{\fboxsep}}\PYG{c+c1}{\PYGZsh{} check the path to the executable}
\PYG{o}{!}\PYG{n+nb}{echo} \PYG{n+nv}{\PYGZdl{}ADM1F\PYGZus{}EXE}
\end{sphinxVerbatim}
}

{

\kern-\sphinxverbatimsmallskipamount\kern-\baselineskip
\kern+\FrameHeightAdjust\kern-\fboxrule
\vspace{\nbsphinxcodecellspacing}

\sphinxsetup{VerbatimColor={named}{white}}
\sphinxsetup{VerbatimBorderColor={named}{nbsphinx-code-border}}
\begin{sphinxVerbatim}[commandchars=\\\{\}]
/Users/elchin/project/ADM1F\_WM/build/adm1f
\end{sphinxVerbatim}
}

{
\sphinxsetup{VerbatimColor={named}{nbsphinx-code-bg}}
\sphinxsetup{VerbatimBorderColor={named}{nbsphinx-code-border}}
\begin{sphinxVerbatim}[commandchars=\\\{\}]
\llap{\color{nbsphinxin}[4]:\,\hspace{\fboxrule}\hspace{\fboxsep}}\PYG{c+c1}{\PYGZsh{} running the executable in the cell}
\PYG{o}{!}\PYG{n+nv}{\PYGZdl{}ADM1F\PYGZus{}EXE} \PYGZhy{}steady
\end{sphinxVerbatim}
}

{

\kern-\sphinxverbatimsmallskipamount\kern-\baselineskip
\kern+\FrameHeightAdjust\kern-\fboxrule
\vspace{\nbsphinxcodecellspacing}

\sphinxsetup{VerbatimColor={named}{white}}
\sphinxsetup{VerbatimBorderColor={named}{nbsphinx-code-border}}
\begin{sphinxVerbatim}[commandchars=\\\{\}]
Vliq [m3] is: 3400.000000
Vgas [m3] is: 300.000000
Reading parameters in file: params.dat
Reading influent values in file: influent.dat
Reading initial condition values in file: ic.dat
Running as steady state problem.
Solving.
Done!
\end{sphinxVerbatim}
}

{
\sphinxsetup{VerbatimColor={named}{nbsphinx-code-bg}}
\sphinxsetup{VerbatimBorderColor={named}{nbsphinx-code-border}}
\begin{sphinxVerbatim}[commandchars=\\\{\}]
\llap{\color{nbsphinxin}[5]:\,\hspace{\fboxrule}\hspace{\fboxsep}}\PYG{c+c1}{\PYGZsh{} remove the output files}
\PYG{o}{!}sh clean.sh
\end{sphinxVerbatim}
}

{
\sphinxsetup{VerbatimColor={named}{nbsphinx-code-bg}}
\sphinxsetup{VerbatimBorderColor={named}{nbsphinx-code-border}}
\begin{sphinxVerbatim}[commandchars=\\\{\}]
\llap{\color{nbsphinxin}[6]:\,\hspace{\fboxrule}\hspace{\fboxsep}}\PYG{c+c1}{\PYGZsh{} or run using subprocess}
\PYG{n}{subprocess}\PYG{o}{.}\PYG{n}{Popen}\PYG{p}{(}\PYG{l+s+s1}{\PYGZsq{}}\PYG{l+s+s1}{\PYGZdl{}ADM1F\PYGZus{}EXE \PYGZhy{}ts\PYGZus{}monitor \PYGZhy{}steady}\PYG{l+s+s1}{\PYGZsq{}}\PYG{p}{,} \PYG{n}{shell}\PYG{o}{=}\PYG{k+kc}{True}\PYG{p}{)}
\end{sphinxVerbatim}
}

{

\kern-\sphinxverbatimsmallskipamount\kern-\baselineskip
\kern+\FrameHeightAdjust\kern-\fboxrule
\vspace{\nbsphinxcodecellspacing}

\sphinxsetup{VerbatimColor={named}{white}}
\sphinxsetup{VerbatimBorderColor={named}{nbsphinx-code-border}}
\begin{sphinxVerbatim}[commandchars=\\\{\}]
\llap{\color{nbsphinxout}[6]:\,\hspace{\fboxrule}\hspace{\fboxsep}}<subprocess.Popen at 0x7f84a4974040>
\end{sphinxVerbatim}
}

\sphinxAtStartPar
If the run is successful then \sphinxcode{\sphinxupquote{indicator\sphinxhyphen{}***.out}} should be saved in the \sphinxcode{\sphinxupquote{simulations}} folder. Here take the last time step saved in the last indicator file (\sphinxcode{\sphinxupquote{indicator\sphinxhyphen{}062.out}}). The accending numeration of the output files corresponds to the time iterations taken towards the steady state condition.


\subsection{2. Comparison of the ADM1F (C++) with ADM1 (Matlab)}
\label{\detokenize{jupyter_notebook/Steady_State_Run:2.-Comparison-of-the-ADM1F-(C++)-with-ADM1-(Matlab)}}
\sphinxAtStartPar
The ADM1F runs much faster than the corresponding Matlab version. The main difference between C++ and the Matlab versions of the model is that ADM1F uses optimized solvers from the PETCS package to solve the corresponding mass balance equations. The ADM1F allows usage of the diffrent solvers. The ADM1(Matlab) is using ode45 nonstiff differential equation solver that cannot be changed. Below we benchmark ADM1F(C++) outputs with the ADM1(Matlab).

{
\sphinxsetup{VerbatimColor={named}{nbsphinx-code-bg}}
\sphinxsetup{VerbatimBorderColor={named}{nbsphinx-code-border}}
\begin{sphinxVerbatim}[commandchars=\\\{\}]
\llap{\color{nbsphinxin}[7]:\,\hspace{\fboxrule}\hspace{\fboxsep}}\PYG{c+c1}{\PYGZsh{} read the output produced by ADM1(Matlab) from the xls file}
\PYG{n}{wb} \PYG{o}{=} \PYG{n}{xlrd}\PYG{o}{.}\PYG{n}{open\PYGZus{}workbook}\PYG{p}{(}\PYG{l+s+s1}{\PYGZsq{}}\PYG{l+s+s1}{../docs/jupyter\PYGZus{}notebook/out\PYGZus{}sludge.xls}\PYG{l+s+s1}{\PYGZsq{}}\PYG{p}{)}
\PYG{n}{sheet} \PYG{o}{=} \PYG{n}{wb}\PYG{o}{.}\PYG{n}{sheet\PYGZus{}by\PYGZus{}index}\PYG{p}{(}\PYG{l+m+mi}{1}\PYG{p}{)}
\PYG{n}{results\PYGZus{}matlab} \PYG{o}{=} \PYG{p}{[}\PYG{n}{sheet}\PYG{o}{.}\PYG{n}{cell\PYGZus{}value}\PYG{p}{(}\PYG{l+m+mi}{4}\PYG{p}{,}\PYG{n}{i}\PYG{p}{)} \PYG{k}{for} \PYG{n}{i} \PYG{o+ow}{in} \PYG{n+nb}{range}\PYG{p}{(}\PYG{l+m+mi}{66}\PYG{p}{)}\PYG{p}{]}
\end{sphinxVerbatim}
}

\sphinxAtStartPar
load the last file from the steady runs and compare it with the Matlab output.

{
\sphinxsetup{VerbatimColor={named}{nbsphinx-code-bg}}
\sphinxsetup{VerbatimBorderColor={named}{nbsphinx-code-border}}
\begin{sphinxVerbatim}[commandchars=\\\{\}]
\llap{\color{nbsphinxin}[8]:\,\hspace{\fboxrule}\hspace{\fboxsep}}\PYG{n}{results\PYGZus{}c}\PYG{o}{=}\PYG{n}{np}\PYG{o}{.}\PYG{n}{loadtxt}\PYG{p}{(}\PYG{l+s+s1}{\PYGZsq{}}\PYG{l+s+s1}{indicator\PYGZhy{}062.out}\PYG{l+s+s1}{\PYGZsq{}}\PYG{p}{,} \PYG{n}{skiprows}\PYG{o}{=}\PYG{l+m+mi}{2}\PYG{p}{,} \PYG{n}{unpack}\PYG{o}{=}\PYG{k+kc}{True}\PYG{p}{)}
\end{sphinxVerbatim}
}

{
\sphinxsetup{VerbatimColor={named}{nbsphinx-code-bg}}
\sphinxsetup{VerbatimBorderColor={named}{nbsphinx-code-border}}
\begin{sphinxVerbatim}[commandchars=\\\{\}]
\llap{\color{nbsphinxin}[9]:\,\hspace{\fboxrule}\hspace{\fboxsep}}\PYG{n}{res} \PYG{o}{=} \PYG{n}{pd}\PYG{o}{.}\PYG{n}{DataFrame}\PYG{p}{(}\PYG{p}{\PYGZob{}}
    \PYG{l+s+s2}{\PYGZdq{}}\PYG{l+s+s2}{Matlab}\PYG{l+s+s2}{\PYGZdq{}}\PYG{p}{:} \PYG{n}{np}\PYG{o}{.}\PYG{n}{asarray}\PYG{p}{(}\PYG{n}{results\PYGZus{}matlab}\PYG{p}{)}\PYG{p}{,}
    \PYG{l+s+s2}{\PYGZdq{}}\PYG{l+s+s2}{C++}\PYG{l+s+s2}{\PYGZdq{}}\PYG{p}{:} \PYG{n}{np}\PYG{o}{.}\PYG{n}{asarray}\PYG{p}{(}\PYG{n}{results\PYGZus{}c}\PYG{p}{[}\PYG{p}{:}\PYG{o}{\PYGZhy{}}\PYG{l+m+mi}{1}\PYG{p}{]}\PYG{p}{)}\PYG{p}{\PYGZcb{}}\PYG{p}{)}
\PYG{n}{res}\PYG{o}{.}\PYG{n}{head}\PYG{p}{(}\PYG{p}{)}
\end{sphinxVerbatim}
}

{

\kern-\sphinxverbatimsmallskipamount\kern-\baselineskip
\kern+\FrameHeightAdjust\kern-\fboxrule
\vspace{\nbsphinxcodecellspacing}

\sphinxsetup{VerbatimColor={named}{white}}
\sphinxsetup{VerbatimBorderColor={named}{nbsphinx-code-border}}
\begin{sphinxVerbatim}[commandchars=\\\{\}]
\llap{\color{nbsphinxout}[9]:\,\hspace{\fboxrule}\hspace{\fboxsep}}      Matlab       C++
0  10.116210  10.11620
1   4.529384   4.52938
2  83.095031  83.09500
3   9.171951   9.17149
4  11.824647  11.82410
\end{sphinxVerbatim}
}

{
\sphinxsetup{VerbatimColor={named}{nbsphinx-code-bg}}
\sphinxsetup{VerbatimBorderColor={named}{nbsphinx-code-border}}
\begin{sphinxVerbatim}[commandchars=\\\{\}]
\llap{\color{nbsphinxin}[10]:\,\hspace{\fboxrule}\hspace{\fboxsep}}\PYG{n}{plt}\PYG{o}{.}\PYG{n}{plot}\PYG{p}{(}\PYG{n}{res}\PYG{p}{[}\PYG{l+s+s1}{\PYGZsq{}}\PYG{l+s+s1}{Matlab}\PYG{l+s+s1}{\PYGZsq{}}\PYG{p}{]}\PYG{o}{\PYGZhy{}}\PYG{n}{res}\PYG{p}{[}\PYG{l+s+s1}{\PYGZsq{}}\PYG{l+s+s1}{C++}\PYG{l+s+s1}{\PYGZsq{}}\PYG{p}{]}\PYG{p}{,}\PYG{n}{linewidth}\PYG{o}{=}\PYG{l+m+mi}{2}\PYG{p}{)}
\PYG{n}{plt}\PYG{o}{.}\PYG{n}{xlabel}\PYG{p}{(}\PYG{l+s+s1}{\PYGZsq{}}\PYG{l+s+s1}{Number of outputs}\PYG{l+s+s1}{\PYGZsq{}}\PYG{p}{,}\PYG{n}{fontsize}\PYG{o}{=}\PYG{l+m+mi}{14}\PYG{p}{)}
\PYG{n}{plt}\PYG{o}{.}\PYG{n}{ylabel}\PYG{p}{(}\PYG{l+s+s1}{\PYGZsq{}}\PYG{l+s+s1}{Difference }\PYG{l+s+s1}{\PYGZsq{}}\PYG{p}{,}\PYG{n}{fontsize}\PYG{o}{=}\PYG{l+m+mi}{14}\PYG{p}{)}\PYG{p}{;}
\PYG{n}{plt}\PYG{o}{.}\PYG{n}{title}\PYG{p}{(}\PYG{l+s+s1}{\PYGZsq{}}\PYG{l+s+s1}{res[Matlab]\PYGZhy{}res[C++]}\PYG{l+s+s1}{\PYGZsq{}}\PYG{p}{,}\PYG{n}{fontsize}\PYG{o}{=}\PYG{l+m+mi}{14}\PYG{p}{)}
\PYG{n}{plt}\PYG{o}{.}\PYG{n}{xlim}\PYG{p}{(}\PYG{p}{[}\PYG{l+m+mi}{0}\PYG{p}{,}\PYG{l+m+mi}{60}\PYG{p}{]}\PYG{p}{)}\PYG{p}{;}
\end{sphinxVerbatim}
}

\hrule height -\fboxrule\relax
\vspace{\nbsphinxcodecellspacing}

\makeatletter\setbox\nbsphinxpromptbox\box\voidb@x\makeatother

\begin{nbsphinxfancyoutput}

\noindent\sphinxincludegraphics[width=411\sphinxpxdimen,height=283\sphinxpxdimen]{{jupyter_notebook_Steady_State_Run_14_0}.png}

\end{nbsphinxfancyoutput}

{
\sphinxsetup{VerbatimColor={named}{nbsphinx-code-bg}}
\sphinxsetup{VerbatimBorderColor={named}{nbsphinx-code-border}}
\begin{sphinxVerbatim}[commandchars=\\\{\}]
\llap{\color{nbsphinxin}[11]:\,\hspace{\fboxrule}\hspace{\fboxsep}}\PYG{n}{rmse}\PYG{o}{=}\PYG{n}{sklm}\PYG{o}{.}\PYG{n}{mean\PYGZus{}squared\PYGZus{}error}\PYG{p}{(}\PYG{n}{res}\PYG{p}{[}\PYG{l+s+s1}{\PYGZsq{}}\PYG{l+s+s1}{Matlab}\PYG{l+s+s1}{\PYGZsq{}}\PYG{p}{]}\PYG{p}{,}\PYG{n}{res}\PYG{p}{[}\PYG{l+s+s1}{\PYGZsq{}}\PYG{l+s+s1}{C++}\PYG{l+s+s1}{\PYGZsq{}}\PYG{p}{]}\PYG{p}{)}
\PYG{n}{mae}\PYG{o}{=}\PYG{n}{sklm}\PYG{o}{.}\PYG{n}{mean\PYGZus{}absolute\PYGZus{}error}\PYG{p}{(}\PYG{n}{res}\PYG{p}{[}\PYG{l+s+s1}{\PYGZsq{}}\PYG{l+s+s1}{Matlab}\PYG{l+s+s1}{\PYGZsq{}}\PYG{p}{]}\PYG{p}{,}\PYG{n}{res}\PYG{p}{[}\PYG{l+s+s1}{\PYGZsq{}}\PYG{l+s+s1}{C++}\PYG{l+s+s1}{\PYGZsq{}}\PYG{p}{]}\PYG{p}{)}
\PYG{n}{r2}\PYG{o}{=}\PYG{n}{sklm}\PYG{o}{.}\PYG{n}{r2\PYGZus{}score}\PYG{p}{(}\PYG{n}{res}\PYG{p}{[}\PYG{l+s+s1}{\PYGZsq{}}\PYG{l+s+s1}{Matlab}\PYG{l+s+s1}{\PYGZsq{}}\PYG{p}{]}\PYG{p}{,}\PYG{n}{res}\PYG{p}{[}\PYG{l+s+s1}{\PYGZsq{}}\PYG{l+s+s1}{C++}\PYG{l+s+s1}{\PYGZsq{}}\PYG{p}{]}\PYG{p}{)}
\PYG{n+nb}{print}\PYG{p}{(}\PYG{l+s+s1}{\PYGZsq{}}\PYG{l+s+s1}{MAE:}\PYG{l+s+s1}{\PYGZsq{}}\PYG{p}{,}\PYG{n+nb}{round}\PYG{p}{(}\PYG{n}{mae}\PYG{p}{,}\PYG{l+m+mi}{4}\PYG{p}{)}\PYG{p}{)}
\PYG{n+nb}{print}\PYG{p}{(}\PYG{l+s+s1}{\PYGZsq{}}\PYG{l+s+s1}{RMSE:}\PYG{l+s+s1}{\PYGZsq{}}\PYG{p}{,}\PYG{n+nb}{round}\PYG{p}{(}\PYG{n}{rmse}\PYG{p}{,}\PYG{l+m+mi}{4}\PYG{p}{)}\PYG{p}{)}
\PYG{n+nb}{print}\PYG{p}{(}\PYG{l+s+s1}{\PYGZsq{}}\PYG{l+s+s1}{R2 Score:}\PYG{l+s+s1}{\PYGZsq{}}\PYG{p}{,}\PYG{n+nb}{round}\PYG{p}{(}\PYG{n}{r2}\PYG{p}{,}\PYG{l+m+mi}{4}\PYG{p}{)}\PYG{p}{)}
\end{sphinxVerbatim}
}

{

\kern-\sphinxverbatimsmallskipamount\kern-\baselineskip
\kern+\FrameHeightAdjust\kern-\fboxrule
\vspace{\nbsphinxcodecellspacing}

\sphinxsetup{VerbatimColor={named}{white}}
\sphinxsetup{VerbatimBorderColor={named}{nbsphinx-code-border}}
\begin{sphinxVerbatim}[commandchars=\\\{\}]
MAE: 0.0136
RMSE: 0.0014
R2 Score: 1.0
\end{sphinxVerbatim}
}


\section{ADM1F: Execution time}
\label{\detokenize{jupyter_notebook/execution_time:ADM1F:-Execution-time}}\label{\detokenize{jupyter_notebook/execution_time::doc}}
\sphinxAtStartPar
Here we calculate the execution time for a sample of size 100. We perturb a certain number of elements in one of the inputs files (e.g. influent.dat, ic.dat, params.dat) by some ‘percent’ value. We sample perturbed elements 100 times in a non\sphinxhyphen{}repeatable fashion using latin hypercube ‘lhs’ or ‘uniform’ sampling methods. Then we calculate the execution time 100 times. Note, if you do not have any of the packages used in this script, use \sphinxcode{\sphinxupquote{pip install package\_name}}.

\sphinxAtStartPar
Authors: Wenjuan Zhang and Elchin Jafarov

{
\sphinxsetup{VerbatimColor={named}{nbsphinx-code-bg}}
\sphinxsetup{VerbatimBorderColor={named}{nbsphinx-code-border}}
\begin{sphinxVerbatim}[commandchars=\\\{\}]
\llap{\color{nbsphinxin}[1]:\,\hspace{\fboxrule}\hspace{\fboxsep}}\PYG{k+kn}{import} \PYG{n+nn}{adm1f\PYGZus{}utils} \PYG{k}{as} \PYG{n+nn}{adm1fu}
\PYG{k+kn}{import} \PYG{n+nn}{os}
\PYG{k+kn}{import} \PYG{n+nn}{matplotlib}\PYG{n+nn}{.}\PYG{n+nn}{pyplot} \PYG{k}{as} \PYG{n+nn}{plt}
\PYG{o}{\PYGZpc{}}\PYG{k}{matplotlib} inline
\end{sphinxVerbatim}
}

{
\sphinxsetup{VerbatimColor={named}{nbsphinx-code-bg}}
\sphinxsetup{VerbatimBorderColor={named}{nbsphinx-code-border}}
\begin{sphinxVerbatim}[commandchars=\\\{\}]
\llap{\color{nbsphinxin}[2]:\,\hspace{\fboxrule}\hspace{\fboxsep}}\PYG{c+c1}{\PYGZsh{} navigate to simulations folder}
\PYG{n}{os}\PYG{o}{.}\PYG{n}{chdir}\PYG{p}{(}\PYG{l+s+s1}{\PYGZsq{}}\PYG{l+s+s1}{../../simulations}\PYG{l+s+s1}{\PYGZsq{}}\PYG{p}{)}
\end{sphinxVerbatim}
}

\sphinxAtStartPar
\sphinxstylestrong{1.} Let’s vary elements of the influent.dat

{
\sphinxsetup{VerbatimColor={named}{nbsphinx-code-bg}}
\sphinxsetup{VerbatimBorderColor={named}{nbsphinx-code-border}}
\begin{sphinxVerbatim}[commandchars=\\\{\}]
\llap{\color{nbsphinxin}[3]:\,\hspace{\fboxrule}\hspace{\fboxsep}}\PYG{c+c1}{\PYGZsh{}Set the path to the ADM1F executable}
\PYG{n}{ADM1F\PYGZus{}EXE} \PYG{o}{=} \PYG{l+s+s1}{\PYGZsq{}}\PYG{l+s+s1}{/Users/elchin/project/ADM1F\PYGZus{}WM/build/adm1f}\PYG{l+s+s1}{\PYGZsq{}}

\PYG{c+c1}{\PYGZsh{} Set the value of percentage and sample size for lhs}
\PYG{n}{percent} \PYG{o}{=} \PYG{l+m+mf}{0.1} \PYG{c+c1}{\PYGZsh{} NOTE: for params percent should be \PYGZlt{}= 0.05}
\PYG{n}{sample\PYGZus{}size} \PYG{o}{=} \PYG{l+m+mi}{100}
\PYG{n}{variable} \PYG{o}{=} \PYG{l+s+s1}{\PYGZsq{}}\PYG{l+s+s1}{influent}\PYG{l+s+s1}{\PYGZsq{}}    \PYG{c+c1}{\PYGZsh{} influent/params/ic}
\PYG{n}{method} \PYG{o}{=} \PYG{l+s+s1}{\PYGZsq{}}\PYG{l+s+s1}{lhs}\PYG{l+s+s1}{\PYGZsq{}}    \PYG{c+c1}{\PYGZsh{}\PYGZsq{}uniform\PYGZsq{} or \PYGZsq{}lhs\PYGZsq{}}
\end{sphinxVerbatim}
}

{
\sphinxsetup{VerbatimColor={named}{nbsphinx-code-bg}}
\sphinxsetup{VerbatimBorderColor={named}{nbsphinx-code-border}}
\begin{sphinxVerbatim}[commandchars=\\\{\}]
\llap{\color{nbsphinxin}[4]:\,\hspace{\fboxrule}\hspace{\fboxsep}}\PYG{c+c1}{\PYGZsh{}use help command to learn more about create\PYGZus{}a\PYGZus{}sample\PYGZus{}matrix function}
\PYG{c+c1}{\PYGZsh{}help(adm1fu.create\PYGZus{}a\PYGZus{}sample\PYGZus{}matrix)}
\end{sphinxVerbatim}
}

{
\sphinxsetup{VerbatimColor={named}{nbsphinx-code-bg}}
\sphinxsetup{VerbatimBorderColor={named}{nbsphinx-code-border}}
\begin{sphinxVerbatim}[commandchars=\\\{\}]
\llap{\color{nbsphinxin}[5]:\,\hspace{\fboxrule}\hspace{\fboxsep}}\PYG{n}{index}\PYG{o}{=}\PYG{n}{adm1fu}\PYG{o}{.}\PYG{n}{create\PYGZus{}a\PYGZus{}sample\PYGZus{}matrix}\PYG{p}{(}\PYG{n}{variable}\PYG{p}{,}\PYG{n}{method}\PYG{p}{,}\PYG{n}{percent}\PYG{p}{,}\PYG{n}{sample\PYGZus{}size}\PYG{p}{)}
\PYG{n+nb}{print} \PYG{p}{(}\PYG{p}{)}
\PYG{n+nb}{print} \PYG{p}{(}\PYG{l+s+s1}{\PYGZsq{}}\PYG{l+s+s1}{Number of elements participated in the sampling:}\PYG{l+s+s1}{\PYGZsq{}}\PYG{p}{,}\PYG{n+nb}{len}\PYG{p}{(}\PYG{n}{index}\PYG{p}{)}\PYG{p}{)}
\end{sphinxVerbatim}
}

{

\kern-\sphinxverbatimsmallskipamount\kern-\baselineskip
\kern+\FrameHeightAdjust\kern-\fboxrule
\vspace{\nbsphinxcodecellspacing}

\sphinxsetup{VerbatimColor={named}{white}}
\sphinxsetup{VerbatimBorderColor={named}{nbsphinx-code-border}}
\begin{sphinxVerbatim}[commandchars=\\\{\}]
Saves a sampling matrix [sample\_size,array\_size] into var\_influent.csv
sample\_size,array\_size:  (100, 11)
Each column of the matrix corresponds to a variable perturbed 100 times around its original value
var\_influent.csv SAVED!

Number of elements participated in the sampling: 11
\end{sphinxVerbatim}
}

{
\sphinxsetup{VerbatimColor={named}{nbsphinx-code-bg}}
\sphinxsetup{VerbatimBorderColor={named}{nbsphinx-code-border}}
\begin{sphinxVerbatim}[commandchars=\\\{\}]
\llap{\color{nbsphinxin}[6]:\,\hspace{\fboxrule}\hspace{\fboxsep}}\PYG{n}{exe\PYGZus{}time}\PYG{o}{=}\PYG{n}{adm1fu}\PYG{o}{.}\PYG{n}{adm1f\PYGZus{}output\PYGZus{}sampling}\PYG{p}{(}\PYG{n}{ADM1F\PYGZus{}EXE}\PYG{p}{,}\PYG{n}{variable}\PYG{p}{,}\PYG{n}{index}\PYG{p}{)}
\end{sphinxVerbatim}
}

{

\kern-\sphinxverbatimsmallskipamount\kern-\baselineskip
\kern+\FrameHeightAdjust\kern-\fboxrule
\vspace{\nbsphinxcodecellspacing}

\sphinxsetup{VerbatimColor={named}{white}}
\sphinxsetup{VerbatimBorderColor={named}{nbsphinx-code-border}}
\begin{sphinxVerbatim}[commandchars=\\\{\}]
All 100 runs were successfully computed
outputs\_influent.csv SAVED!
\end{sphinxVerbatim}
}

\sphinxAtStartPar
\sphinxstylestrong{Note:} Depending on the computer system configuration, the computational time might vary.

{
\sphinxsetup{VerbatimColor={named}{nbsphinx-code-bg}}
\sphinxsetup{VerbatimBorderColor={named}{nbsphinx-code-border}}
\begin{sphinxVerbatim}[commandchars=\\\{\}]
\llap{\color{nbsphinxin}[7]:\,\hspace{\fboxrule}\hspace{\fboxsep}}\PYG{k}{def} \PYG{n+nf}{plot\PYGZus{}exec\PYGZus{}time}\PYG{p}{(}\PYG{n}{exe\PYGZus{}time}\PYG{p}{)}\PYG{p}{:}
    \PYG{n}{plt}\PYG{o}{.}\PYG{n}{plot}\PYG{p}{(}\PYG{n}{exe\PYGZus{}time}\PYG{p}{,}\PYG{l+s+s1}{\PYGZsq{}}\PYG{l+s+s1}{*}\PYG{l+s+s1}{\PYGZsq{}}\PYG{p}{)}
    \PYG{n}{plt}\PYG{o}{.}\PYG{n}{axhline}\PYG{p}{(}\PYG{n}{exe\PYGZus{}time}\PYG{o}{.}\PYG{n}{mean}\PYG{p}{(}\PYG{p}{)}\PYG{p}{,}\PYG{n}{linestyle}\PYG{o}{=}\PYG{l+s+s1}{\PYGZsq{}}\PYG{l+s+s1}{\PYGZhy{}\PYGZhy{}}\PYG{l+s+s1}{\PYGZsq{}}\PYG{p}{,} \PYG{n}{alpha}\PYG{o}{=}\PYG{l+m+mf}{0.6}\PYG{p}{,}\PYG{n}{color}\PYG{o}{=}\PYG{l+s+s1}{\PYGZsq{}}\PYG{l+s+s1}{green}\PYG{l+s+s1}{\PYGZsq{}}\PYG{p}{)}
    \PYG{n}{plt}\PYG{o}{.}\PYG{n}{xlabel}\PYG{p}{(}\PYG{l+s+s1}{\PYGZsq{}}\PYG{l+s+s1}{sample size}\PYG{l+s+s1}{\PYGZsq{}}\PYG{p}{,}\PYG{n}{fontsize}\PYG{o}{=}\PYG{l+m+mi}{14}\PYG{p}{)}
    \PYG{n}{plt}\PYG{o}{.}\PYG{n}{ylabel}\PYG{p}{(}\PYG{l+s+s1}{\PYGZsq{}}\PYG{l+s+s1}{Time [secs]}\PYG{l+s+s1}{\PYGZsq{}}\PYG{p}{,}\PYG{n}{fontsize}\PYG{o}{=}\PYG{l+m+mi}{14}\PYG{p}{)}
    \PYG{n}{plt}\PYG{o}{.}\PYG{n}{legend}\PYG{p}{(}\PYG{p}{[}\PYG{l+s+s1}{\PYGZsq{}}\PYG{l+s+s1}{exec time}\PYG{l+s+s1}{\PYGZsq{}}\PYG{p}{,}\PYG{l+s+s1}{\PYGZsq{}}\PYG{l+s+s1}{mean}\PYG{l+s+s1}{\PYGZsq{}}\PYG{p}{]}\PYG{p}{)}
    \PYG{n+nb}{print}\PYG{p}{(}\PYG{l+s+s1}{\PYGZsq{}}\PYG{l+s+s1}{cumulative time:}\PYG{l+s+s1}{\PYGZsq{}}\PYG{p}{,}\PYG{n+nb}{round}\PYG{p}{(}\PYG{n}{exe\PYGZus{}time}\PYG{o}{.}\PYG{n}{sum}\PYG{p}{(}\PYG{p}{)}\PYG{p}{,}\PYG{l+m+mi}{2}\PYG{p}{)}\PYG{p}{,}\PYG{l+s+s1}{\PYGZsq{}}\PYG{l+s+s1}{seconds}\PYG{l+s+s1}{\PYGZsq{}}\PYG{p}{,}\PYG{p}{)}
    \PYG{n+nb}{print}\PYG{p}{(}\PYG{l+s+s1}{\PYGZsq{}}\PYG{l+s+s1}{mean time:}\PYG{l+s+s1}{\PYGZsq{}}\PYG{p}{,}\PYG{n+nb}{round}\PYG{p}{(}\PYG{n}{exe\PYGZus{}time}\PYG{o}{.}\PYG{n}{mean}\PYG{p}{(}\PYG{p}{)}\PYG{p}{,}\PYG{l+m+mi}{2}\PYG{p}{)}\PYG{p}{,}\PYG{l+s+s1}{\PYGZsq{}}\PYG{l+s+s1}{seconds}\PYG{l+s+s1}{\PYGZsq{}}\PYG{p}{)}
    \PYG{n+nb}{print}\PYG{p}{(}\PYG{l+s+s1}{\PYGZsq{}}\PYG{l+s+s1}{min time:}\PYG{l+s+s1}{\PYGZsq{}}\PYG{p}{,}\PYG{n+nb}{round}\PYG{p}{(}\PYG{n}{exe\PYGZus{}time}\PYG{o}{.}\PYG{n}{min}\PYG{p}{(}\PYG{p}{)}\PYG{p}{,}\PYG{l+m+mi}{2}\PYG{p}{)}\PYG{p}{,}\PYG{l+s+s1}{\PYGZsq{}}\PYG{l+s+s1}{seconds}\PYG{l+s+s1}{\PYGZsq{}}\PYG{p}{)}
    \PYG{n+nb}{print}\PYG{p}{(}\PYG{l+s+s1}{\PYGZsq{}}\PYG{l+s+s1}{max time:}\PYG{l+s+s1}{\PYGZsq{}}\PYG{p}{,}\PYG{n+nb}{round}\PYG{p}{(}\PYG{n}{exe\PYGZus{}time}\PYG{o}{.}\PYG{n}{max}\PYG{p}{(}\PYG{p}{)}\PYG{p}{,}\PYG{l+m+mi}{2}\PYG{p}{)}\PYG{p}{,}\PYG{l+s+s1}{\PYGZsq{}}\PYG{l+s+s1}{seconds}\PYG{l+s+s1}{\PYGZsq{}}\PYG{p}{)}
    \PYG{n}{ax} \PYG{o}{=} \PYG{n}{plt}\PYG{o}{.}\PYG{n}{gca}\PYG{p}{(}\PYG{p}{)}
    \PYG{n}{ax}\PYG{o}{.}\PYG{n}{tick\PYGZus{}params}\PYG{p}{(}\PYG{n}{axis} \PYG{o}{=} \PYG{l+s+s1}{\PYGZsq{}}\PYG{l+s+s1}{both}\PYG{l+s+s1}{\PYGZsq{}}\PYG{p}{,} \PYG{n}{which} \PYG{o}{=} \PYG{l+s+s1}{\PYGZsq{}}\PYG{l+s+s1}{major}\PYG{l+s+s1}{\PYGZsq{}}\PYG{p}{,} \PYG{n}{labelsize} \PYG{o}{=} \PYG{l+m+mi}{14}\PYG{p}{)}

\PYG{n}{plot\PYGZus{}exec\PYGZus{}time}\PYG{p}{(}\PYG{n}{exe\PYGZus{}time}\PYG{p}{)}
\end{sphinxVerbatim}
}

{

\kern-\sphinxverbatimsmallskipamount\kern-\baselineskip
\kern+\FrameHeightAdjust\kern-\fboxrule
\vspace{\nbsphinxcodecellspacing}

\sphinxsetup{VerbatimColor={named}{white}}
\sphinxsetup{VerbatimBorderColor={named}{nbsphinx-code-border}}
\begin{sphinxVerbatim}[commandchars=\\\{\}]
cumulative time: 23.97 seconds
mean time: 0.24 seconds
min time: 0.16 seconds
max time: 0.34 seconds
\end{sphinxVerbatim}
}

\hrule height -\fboxrule\relax
\vspace{\nbsphinxcodecellspacing}

\makeatletter\setbox\nbsphinxpromptbox\box\voidb@x\makeatother

\begin{nbsphinxfancyoutput}

\noindent\sphinxincludegraphics[width=406\sphinxpxdimen,height=270\sphinxpxdimen]{{jupyter_notebook_execution_time_9_1}.png}

\end{nbsphinxfancyoutput}

\sphinxAtStartPar
\sphinxstylestrong{2.} Let’s vary the param.dat elements and compute the execution time.

{
\sphinxsetup{VerbatimColor={named}{nbsphinx-code-bg}}
\sphinxsetup{VerbatimBorderColor={named}{nbsphinx-code-border}}
\begin{sphinxVerbatim}[commandchars=\\\{\}]
\llap{\color{nbsphinxin}[8]:\,\hspace{\fboxrule}\hspace{\fboxsep}}\PYG{c+c1}{\PYGZsh{} Set the value of percentage and sample size for lhs}
\PYG{n}{percent} \PYG{o}{=} \PYG{l+m+mf}{0.05} \PYG{c+c1}{\PYGZsh{} NOTE: for params percent should be \PYGZlt{}= 0.05}
\PYG{n}{sample\PYGZus{}size} \PYG{o}{=} \PYG{l+m+mi}{100}
\PYG{n}{variable} \PYG{o}{=} \PYG{l+s+s1}{\PYGZsq{}}\PYG{l+s+s1}{params}\PYG{l+s+s1}{\PYGZsq{}}    \PYG{c+c1}{\PYGZsh{} influent/params/ic}
\PYG{n}{method} \PYG{o}{=} \PYG{l+s+s1}{\PYGZsq{}}\PYG{l+s+s1}{lhs}\PYG{l+s+s1}{\PYGZsq{}}    \PYG{c+c1}{\PYGZsh{}\PYGZsq{}uniform\PYGZsq{} or \PYGZsq{}lhs\PYGZsq{}}
\end{sphinxVerbatim}
}

{
\sphinxsetup{VerbatimColor={named}{nbsphinx-code-bg}}
\sphinxsetup{VerbatimBorderColor={named}{nbsphinx-code-border}}
\begin{sphinxVerbatim}[commandchars=\\\{\}]
\llap{\color{nbsphinxin}[9]:\,\hspace{\fboxrule}\hspace{\fboxsep}}\PYG{n}{index}\PYG{o}{=}\PYG{n}{adm1fu}\PYG{o}{.}\PYG{n}{create\PYGZus{}a\PYGZus{}sample\PYGZus{}matrix}\PYG{p}{(}\PYG{n}{variable}\PYG{p}{,}\PYG{n}{method}\PYG{p}{,}\PYG{n}{percent}\PYG{p}{,}\PYG{n}{sample\PYGZus{}size}\PYG{p}{)}
\PYG{n+nb}{print} \PYG{p}{(}\PYG{p}{)}
\PYG{n+nb}{print} \PYG{p}{(}\PYG{l+s+s1}{\PYGZsq{}}\PYG{l+s+s1}{Number of elements participated in the sampling:}\PYG{l+s+s1}{\PYGZsq{}}\PYG{p}{,}\PYG{n+nb}{len}\PYG{p}{(}\PYG{n}{index}\PYG{p}{)}\PYG{p}{)}
\end{sphinxVerbatim}
}

{

\kern-\sphinxverbatimsmallskipamount\kern-\baselineskip
\kern+\FrameHeightAdjust\kern-\fboxrule
\vspace{\nbsphinxcodecellspacing}

\sphinxsetup{VerbatimColor={named}{white}}
\sphinxsetup{VerbatimBorderColor={named}{nbsphinx-code-border}}
\begin{sphinxVerbatim}[commandchars=\\\{\}]
Saves a sampling matrix [sample\_size,array\_size] into var\_params.csv
sample\_size,array\_size:  (100, 92)
Each column of the matrix corresponds to a variable perturbed 100 times around its original value
var\_params.csv SAVED!

Number of elements participated in the sampling: 92
\end{sphinxVerbatim}
}

{
\sphinxsetup{VerbatimColor={named}{nbsphinx-code-bg}}
\sphinxsetup{VerbatimBorderColor={named}{nbsphinx-code-border}}
\begin{sphinxVerbatim}[commandchars=\\\{\}]
\llap{\color{nbsphinxin}[10]:\,\hspace{\fboxrule}\hspace{\fboxsep}}\PYG{n}{exe\PYGZus{}time}\PYG{o}{=}\PYG{n}{adm1fu}\PYG{o}{.}\PYG{n}{adm1f\PYGZus{}output\PYGZus{}sampling}\PYG{p}{(}\PYG{n}{ADM1F\PYGZus{}EXE}\PYG{p}{,}\PYG{n}{variable}\PYG{p}{,}\PYG{n}{index}\PYG{p}{)}
\end{sphinxVerbatim}
}

{

\kern-\sphinxverbatimsmallskipamount\kern-\baselineskip
\kern+\FrameHeightAdjust\kern-\fboxrule
\vspace{\nbsphinxcodecellspacing}

\sphinxsetup{VerbatimColor={named}{white}}
\sphinxsetup{VerbatimBorderColor={named}{nbsphinx-code-border}}
\begin{sphinxVerbatim}[commandchars=\\\{\}]
All 100 runs were successfully computed
outputs\_params.csv SAVED!
\end{sphinxVerbatim}
}

{
\sphinxsetup{VerbatimColor={named}{nbsphinx-code-bg}}
\sphinxsetup{VerbatimBorderColor={named}{nbsphinx-code-border}}
\begin{sphinxVerbatim}[commandchars=\\\{\}]
\llap{\color{nbsphinxin}[11]:\,\hspace{\fboxrule}\hspace{\fboxsep}}\PYG{n}{plot\PYGZus{}exec\PYGZus{}time}\PYG{p}{(}\PYG{n}{exe\PYGZus{}time}\PYG{p}{)}
\end{sphinxVerbatim}
}

{

\kern-\sphinxverbatimsmallskipamount\kern-\baselineskip
\kern+\FrameHeightAdjust\kern-\fboxrule
\vspace{\nbsphinxcodecellspacing}

\sphinxsetup{VerbatimColor={named}{white}}
\sphinxsetup{VerbatimBorderColor={named}{nbsphinx-code-border}}
\begin{sphinxVerbatim}[commandchars=\\\{\}]
cumulative time: 20.32 seconds
mean time: 0.2 seconds
min time: 0.0 seconds
max time: 0.34 seconds
\end{sphinxVerbatim}
}

\hrule height -\fboxrule\relax
\vspace{\nbsphinxcodecellspacing}

\makeatletter\setbox\nbsphinxpromptbox\box\voidb@x\makeatother

\begin{nbsphinxfancyoutput}

\noindent\sphinxincludegraphics[width=397\sphinxpxdimen,height=270\sphinxpxdimen]{{jupyter_notebook_execution_time_14_1}.png}

\end{nbsphinxfancyoutput}

{
\sphinxsetup{VerbatimColor={named}{nbsphinx-code-bg}}
\sphinxsetup{VerbatimBorderColor={named}{nbsphinx-code-border}}
\begin{sphinxVerbatim}[commandchars=\\\{\}]
\llap{\color{nbsphinxin}[ ]:\,\hspace{\fboxrule}\hspace{\fboxsep}}
\end{sphinxVerbatim}
}


\section{ADM1F SRT: single tank and two\sphinxhyphen{}phase anaerobic dynamic membrane bioreactor}
\label{\detokenize{jupyter_notebook/ADM1F_SRT_2phase:ADM1F-SRT:-single-tank-and-two-phase-anaerobic-dynamic-membrane-bioreactor}}\label{\detokenize{jupyter_notebook/ADM1F_SRT_2phase::doc}}
\sphinxAtStartPar
This script is used to simulate a single tank suspended anaerobic dynamic membrane digester and a novel two\sphinxhyphen{}phase anaerobic dynamic membrane bioreator with separated SRT and HRT. In the two\sphinxhyphen{}phase reactor the effluent (model output) from the first phase dynamic membrane bioreactor is converted to the influent (model input) for the second\sphinxhyphen{}phase anaerobic dynamic membrane bioreactor as shown in the figure.

\sphinxAtStartPar
\sphinxstylestrong{Note:} Before running the ADM1F SRT simulations see intructions on how to compile \sphinxcode{\sphinxupquote{adm1f\_srt.cxx}} in the User Guide.

\sphinxAtStartPar
\sphinxstylestrong{Mass balance equation with SRT:}

\sphinxAtStartPar
The ADM1 and ADM1F solve the mass balance equation (i.e. mass\_change = mass\_in \textendash{} mass\_out + reaction). ADM1F\_SRT version of the model includes solid retention time (\(t_{res,X}\)) as shown in the equation below.
\begin{equation*}
\begin{split}\frac{d\, S_{\text{liq,i}}}{dt} = \frac{qS_{\text{in,i}}}{V_{\text{liq}}} - \frac{qS_{\text{liq,i}} }{V_{\text{liq}}}+ \sum_{j=1}^{12}\rho_{j} v_{i,j}\end{split}
\end{equation*}\begin{equation*}
\begin{split}\frac{d\, X_{\text{liq,i}}}{dt} = \frac{qX_{\text{in,i}}}{V_{\text{liq}}} - \frac{X_{\text{liq,i}}}{t_{\text{res,X}} + V_{\text{liq}}/q} + \sum_{j=13}^{24}\rho_{j} v_{i,j}\end{split}
\end{equation*}
\sphinxAtStartPar
Authors: Wenjuan Zhang, Elchin Jafarov, Kuang Zhu

{
\sphinxsetup{VerbatimColor={named}{nbsphinx-code-bg}}
\sphinxsetup{VerbatimBorderColor={named}{nbsphinx-code-border}}
\begin{sphinxVerbatim}[commandchars=\\\{\}]
\llap{\color{nbsphinxin}[1]:\,\hspace{\fboxrule}\hspace{\fboxsep}}\PYG{c+c1}{\PYGZsh{} Load packages}
\PYG{k+kn}{import} \PYG{n+nn}{numpy} \PYG{k}{as} \PYG{n+nn}{np}
\PYG{k+kn}{import} \PYG{n+nn}{pandas} \PYG{k}{as} \PYG{n+nn}{pd}
\PYG{k+kn}{import} \PYG{n+nn}{os}
\PYG{k+kn}{import} \PYG{n+nn}{seaborn} \PYG{k}{as} \PYG{n+nn}{sns}
\PYG{k+kn}{import} \PYG{n+nn}{adm1f\PYGZus{}utils} \PYG{k}{as} \PYG{n+nn}{adm1fu}
\PYG{k+kn}{import} \PYG{n+nn}{matplotlib}\PYG{n+nn}{.}\PYG{n+nn}{pyplot} \PYG{k}{as} \PYG{n+nn}{plt}
\PYG{o}{\PYGZpc{}}\PYG{k}{matplotlib} inline
\end{sphinxVerbatim}
}

{
\sphinxsetup{VerbatimColor={named}{nbsphinx-code-bg}}
\sphinxsetup{VerbatimBorderColor={named}{nbsphinx-code-border}}
\begin{sphinxVerbatim}[commandchars=\\\{\}]
\llap{\color{nbsphinxin}[2]:\,\hspace{\fboxrule}\hspace{\fboxsep}}\PYG{c+c1}{\PYGZsh{} navigate to simulations folder}
\PYG{n}{os}\PYG{o}{.}\PYG{n}{chdir}\PYG{p}{(}\PYG{l+s+s1}{\PYGZsq{}}\PYG{l+s+s1}{../simulations}\PYG{l+s+s1}{\PYGZsq{}}\PYG{p}{)}
\end{sphinxVerbatim}
}

{
\sphinxsetup{VerbatimColor={named}{nbsphinx-code-bg}}
\sphinxsetup{VerbatimBorderColor={named}{nbsphinx-code-border}}
\begin{sphinxVerbatim}[commandchars=\\\{\}]
\llap{\color{nbsphinxin}[3]:\,\hspace{\fboxrule}\hspace{\fboxsep}}\PYG{c+c1}{\PYGZsh{} Grab the names and unit of all the outputs}
\PYG{p}{(}\PYG{n}{output\PYGZus{}name}\PYG{p}{,}\PYG{n}{output\PYGZus{}unit}\PYG{p}{)}\PYG{o}{=}\PYG{n}{adm1fu}\PYG{o}{.}\PYG{n}{get\PYGZus{}output\PYGZus{}names}\PYG{p}{(}\PYG{p}{)}
\end{sphinxVerbatim}
}

{
\sphinxsetup{VerbatimColor={named}{nbsphinx-code-bg}}
\sphinxsetup{VerbatimBorderColor={named}{nbsphinx-code-border}}
\begin{sphinxVerbatim}[commandchars=\\\{\}]
\llap{\color{nbsphinxin}[4]:\,\hspace{\fboxrule}\hspace{\fboxsep}}\PYG{c+c1}{\PYGZsh{}check the path to the executable}
\PYG{o}{!}\PYG{n+nb}{echo} \PYG{n+nv}{\PYGZdl{}ADM1F\PYGZus{}EXE}
\end{sphinxVerbatim}
}

{

\kern-\sphinxverbatimsmallskipamount\kern-\baselineskip
\kern+\FrameHeightAdjust\kern-\fboxrule
\vspace{\nbsphinxcodecellspacing}

\sphinxsetup{VerbatimColor={named}{white}}
\sphinxsetup{VerbatimBorderColor={named}{nbsphinx-code-border}}
\begin{sphinxVerbatim}[commandchars=\\\{\}]
/Users/elchin/project/ADM1F\_WM/build/adm1f
\end{sphinxVerbatim}
}


\subsection{Single tank anaerobic dynamic membrane bioreactor (AnDMBR)}
\label{\detokenize{jupyter_notebook/ADM1F_SRT_2phase:Single-tank-anaerobic-dynamic-membrane-bioreactor-(AnDMBR)}}
\sphinxAtStartPar
To simulate the single tank AnDMBR that has separated SRT and HRT, We can call the function\sphinxcode{\sphinxupquote{reactor1}} with different Q (flow\sphinxhyphen{}rate), Vliq (reactor volume), t\_resx (SRT\sphinxhyphen{}HRT) values, the function will return the corresponding output

\sphinxAtStartPar
Usage example: reactor1(Q=100, t\_resx=30, Vliq=300)

\sphinxAtStartPar
\sphinxstylestrong{(Unit)} Q: {[}m3/d{]}, t\_resx: {[}day{]}, Vliq: {[}m3{]}

{
\sphinxsetup{VerbatimColor={named}{nbsphinx-code-bg}}
\sphinxsetup{VerbatimBorderColor={named}{nbsphinx-code-border}}
\begin{sphinxVerbatim}[commandchars=\\\{\}]
\llap{\color{nbsphinxin}[5]:\,\hspace{\fboxrule}\hspace{\fboxsep}}\PYG{c+c1}{\PYGZsh{} testing different SRTs=[0,1,2,...9] on the one\PYGZhy{}phase reactor}
\PYG{n}{resx\PYGZus{}list} \PYG{o}{=} \PYG{p}{[}\PYG{n}{i} \PYG{k}{for} \PYG{n}{i} \PYG{o+ow}{in} \PYG{n+nb}{range}\PYG{p}{(}\PYG{l+m+mi}{10}\PYG{p}{)}\PYG{p}{]}
\PYG{c+c1}{\PYGZsh{} setup the matrix with columns correspoding SRTs and rows to outputs}
\PYG{n}{output1\PYGZus{}resx} \PYG{o}{=} \PYG{n}{np}\PYG{o}{.}\PYG{n}{zeros}\PYG{p}{(}\PYG{p}{(}\PYG{n+nb}{len}\PYG{p}{(}\PYG{n}{resx\PYGZus{}list}\PYG{p}{)}\PYG{p}{,} \PYG{l+m+mi}{67}\PYG{p}{)}\PYG{p}{)}
\PYG{c+c1}{\PYGZsh{} here we utilize back euler solver and adataptive time step}
\PYG{c+c1}{\PYGZsh{} for more command options see \PYGZdq{}User Guide/Running ADM1F/step 5\PYGZdq{}}
\PYG{n}{options}\PYG{o}{=}\PYG{l+s+s1}{\PYGZsq{}}\PYG{l+s+s1}{\PYGZhy{}ts\PYGZus{}type beuler \PYGZhy{}ts\PYGZus{}adapt\PYGZus{}type basic \PYGZhy{}ts\PYGZus{}max\PYGZus{}snes\PYGZus{}failures \PYGZhy{}1 \PYGZhy{}steady}\PYG{l+s+s1}{\PYGZsq{}}

\PYG{k}{for} \PYG{n}{i} \PYG{o+ow}{in} \PYG{n+nb}{range}\PYG{p}{(}\PYG{n+nb}{len}\PYG{p}{(}\PYG{n}{resx\PYGZus{}list}\PYG{p}{)}\PYG{p}{)}\PYG{p}{:}
    \PYG{n}{output1\PYGZus{}resx}\PYG{p}{[}\PYG{n}{i}\PYG{p}{]} \PYG{o}{=} \PYG{n}{adm1fu}\PYG{o}{.}\PYG{n}{reactor1}\PYG{p}{(}\PYG{n}{opt}\PYG{o}{=}\PYG{n}{options}\PYG{p}{,} \PYG{n}{Vliq}\PYG{o}{=}\PYG{l+m+mi}{300}\PYG{p}{,} \PYG{n}{Q}\PYG{o}{=}\PYG{l+m+mi}{600}\PYG{p}{,} \PYG{n}{t\PYGZus{}resx}\PYG{o}{=}\PYG{n}{resx\PYGZus{}list}\PYG{p}{[}\PYG{n}{i}\PYG{p}{]}\PYG{p}{)}

\PYG{n}{np}\PYG{o}{.}\PYG{n}{savetxt}\PYG{p}{(}\PYG{l+s+s1}{\PYGZsq{}}\PYG{l+s+s1}{output\PYGZus{}1phase.csv}\PYG{l+s+s1}{\PYGZsq{}}\PYG{p}{,}\PYG{n}{output1\PYGZus{}resx}\PYG{p}{,}\PYG{n}{delimiter}\PYG{o}{=}\PYG{l+s+s1}{\PYGZsq{}}\PYG{l+s+s1}{,}\PYG{l+s+s1}{\PYGZsq{}}\PYG{p}{,}\PYG{n}{fmt}\PYG{o}{=}\PYG{l+s+s1}{\PYGZsq{}}\PYG{l+s+si}{\PYGZpc{}1.4e}\PYG{l+s+s1}{\PYGZsq{}}\PYG{p}{)}
\end{sphinxVerbatim}
}

{

\kern-\sphinxverbatimsmallskipamount\kern-\baselineskip
\kern+\FrameHeightAdjust\kern-\fboxrule
\vspace{\nbsphinxcodecellspacing}

\sphinxsetup{VerbatimColor={named}{white}}
\sphinxsetup{VerbatimBorderColor={named}{nbsphinx-code-border}}
\begin{sphinxVerbatim}[commandchars=\\\{\}]
Reactor run, phase-one:
\$ADM1F\_EXE -ts\_type beuler -ts\_adapt\_type basic -ts\_max\_snes\_failures -1 -steady -Vliq 300 -t\_resx 0 -influent\_file influent\_cur.dat
indicator-228.out
Reactor run, phase-one:
\$ADM1F\_EXE -ts\_type beuler -ts\_adapt\_type basic -ts\_max\_snes\_failures -1 -steady -Vliq 300 -t\_resx 1 -influent\_file influent\_cur.dat
indicator-307.out
Reactor run, phase-one:
\$ADM1F\_EXE -ts\_type beuler -ts\_adapt\_type basic -ts\_max\_snes\_failures -1 -steady -Vliq 300 -t\_resx 2 -influent\_file influent\_cur.dat
indicator-437.out
Reactor run, phase-one:
\$ADM1F\_EXE -ts\_type beuler -ts\_adapt\_type basic -ts\_max\_snes\_failures -1 -steady -Vliq 300 -t\_resx 3 -influent\_file influent\_cur.dat
indicator-563.out
Reactor run, phase-one:
\$ADM1F\_EXE -ts\_type beuler -ts\_adapt\_type basic -ts\_max\_snes\_failures -1 -steady -Vliq 300 -t\_resx 4 -influent\_file influent\_cur.dat
indicator-521.out
Reactor run, phase-one:
\$ADM1F\_EXE -ts\_type beuler -ts\_adapt\_type basic -ts\_max\_snes\_failures -1 -steady -Vliq 300 -t\_resx 5 -influent\_file influent\_cur.dat
indicator-516.out
Reactor run, phase-one:
\$ADM1F\_EXE -ts\_type beuler -ts\_adapt\_type basic -ts\_max\_snes\_failures -1 -steady -Vliq 300 -t\_resx 6 -influent\_file influent\_cur.dat
indicator-534.out
Reactor run, phase-one:
\$ADM1F\_EXE -ts\_type beuler -ts\_adapt\_type basic -ts\_max\_snes\_failures -1 -steady -Vliq 300 -t\_resx 7 -influent\_file influent\_cur.dat
indicator-542.out
Reactor run, phase-one:
\$ADM1F\_EXE -ts\_type beuler -ts\_adapt\_type basic -ts\_max\_snes\_failures -1 -steady -Vliq 300 -t\_resx 8 -influent\_file influent\_cur.dat
indicator-556.out
Reactor run, phase-one:
\$ADM1F\_EXE -ts\_type beuler -ts\_adapt\_type basic -ts\_max\_snes\_failures -1 -steady -Vliq 300 -t\_resx 9 -influent\_file influent\_cur.dat
indicator-576.out
\end{sphinxVerbatim}
}

{
\sphinxsetup{VerbatimColor={named}{nbsphinx-code-bg}}
\sphinxsetup{VerbatimBorderColor={named}{nbsphinx-code-border}}
\begin{sphinxVerbatim}[commandchars=\\\{\}]
\llap{\color{nbsphinxin}[6]:\,\hspace{\fboxrule}\hspace{\fboxsep}}\PYG{n}{df1\PYGZus{}resx} \PYG{o}{=} \PYG{n}{pd}\PYG{o}{.}\PYG{n}{read\PYGZus{}csv}\PYG{p}{(}\PYG{l+s+s1}{\PYGZsq{}}\PYG{l+s+s1}{output\PYGZus{}1phase.csv}\PYG{l+s+s1}{\PYGZsq{}}\PYG{p}{,} \PYG{n}{sep}\PYG{o}{=}\PYG{l+s+s1}{\PYGZsq{}}\PYG{l+s+s1}{,}\PYG{l+s+s1}{\PYGZsq{}}\PYG{p}{,} \PYG{n}{header}\PYG{o}{=}\PYG{k+kc}{None}\PYG{p}{)}
\PYG{n}{df1\PYGZus{}resx}\PYG{o}{.}\PYG{n}{columns} \PYG{o}{=} \PYG{n}{output\PYGZus{}name}
\PYG{n}{df1\PYGZus{}resx}\PYG{o}{.}\PYG{n}{insert}\PYG{p}{(}\PYG{l+m+mi}{0}\PYG{p}{,}\PYG{l+s+s2}{\PYGZdq{}}\PYG{l+s+s2}{T\PYGZus{}resx}\PYG{l+s+s2}{\PYGZdq{}}\PYG{p}{,}\PYG{n}{resx\PYGZus{}list}\PYG{p}{)}
\end{sphinxVerbatim}
}

\sphinxAtStartPar
\sphinxstylestrong{Relation between t\_resx and output when Vliq=300m3, Q=600m3/d}

{
\sphinxsetup{VerbatimColor={named}{nbsphinx-code-bg}}
\sphinxsetup{VerbatimBorderColor={named}{nbsphinx-code-border}}
\begin{sphinxVerbatim}[commandchars=\\\{\}]
\llap{\color{nbsphinxin}[7]:\,\hspace{\fboxrule}\hspace{\fboxsep}}\PYG{c+c1}{\PYGZsh{} check the results with increasing T\PYGZus{}resx, we should expect decrease in Ssu}
\PYG{n}{df1\PYGZus{}resx}
\end{sphinxVerbatim}
}

{

\kern-\sphinxverbatimsmallskipamount\kern-\baselineskip
\kern+\FrameHeightAdjust\kern-\fboxrule
\vspace{\nbsphinxcodecellspacing}

\sphinxsetup{VerbatimColor={named}{white}}
\sphinxsetup{VerbatimBorderColor={named}{nbsphinx-code-border}}
\begin{sphinxVerbatim}[commandchars=\\\{\}]
\llap{\color{nbsphinxout}[7]:\,\hspace{\fboxrule}\hspace{\fboxsep}}   T\_resx      Ssu       Saa      Sfa       Sva       Sbu      Spro     Sac   \textbackslash{}
0       0  2152.800  324.2700  7382.70  2381.700  3667.300  2542.700  8263.5
1       1   123.060   49.1670  7751.20  2574.400  4202.900  3244.500  9600.7
2       2    63.057   26.6380  7839.00  2610.000  4268.600  3305.000  9742.9
3       3    39.313   17.3450   908.19    41.129    54.118    53.411  3339.7
4       4    30.634   13.5750   457.20    30.592    39.989    35.895  3066.6
5       5    25.261   11.2240   309.54    24.543    31.958    27.254  2856.4
6       6    21.606    9.6181   236.08    20.615    26.775    22.111  2685.8
7       7    18.960    8.4513   192.14    17.859    23.153    18.702  2543.7
8       8    16.955    7.5652   162.87    15.818    20.479    16.276  2423.2
9       9    15.383    6.8691   141.96    14.244    18.422    14.462  2319.3

        Sh2          Sch4   {\ldots}     Alk       NH3      NH4     LCFA   \textbackslash{}
0  25.643000 -8.963700e-82  {\ldots}  44957.0  0.053067  1074.00  7382.70
1  30.352000  3.859800e-12  {\ldots}  59182.0  0.047701  1135.60  7751.20
2  30.772000  1.085400e-09  {\ldots}  61421.0  0.048208  1150.90  7839.00
3   0.000662  2.574900e+02  {\ldots}  16289.0  2.513100   963.03   908.19
4   0.000515  2.653300e+02  {\ldots}  18846.0  3.157300   963.70   457.20
5   0.000424  2.695900e+02  {\ldots}  21319.0  3.693200   966.73   309.54
6   0.000362  2.726300e+02  {\ldots}  23721.0  4.147900   970.17   236.08
7   0.000317  2.750200e+02  {\ldots}  26050.0  4.540600   973.69   192.14
8   0.000284  2.770000e+02  {\ldots}  28311.0  4.885500   977.21   162.87
9   0.000257  2.786900e+02  {\ldots}  30505.0  5.192300   980.68   141.96

    percentch4    energych4   efficiency   VFA/ALK           ACN  sampleT
0 -1.133000e-81       83.808      3.9774  0.351490 -3.203400e-72      0.5
1  1.599000e-12      100.630    -76.4800  0.310840  1.255000e-08      0.5
2  4.422100e-10      102.210   -155.0200  0.304150  2.777100e-07      0.5
3  5.579800e+01      209.370   -218.4400  0.200760  9.097200e+00      0.5
4  5.623900e+01      214.510   -304.2100  0.157850  1.324200e+01      0.5
5  5.648800e+01      217.230   -390.0200  0.129290  1.778600e+01      0.5
6  5.666900e+01      219.140   -475.6200  0.108900  2.279000e+01      0.5
7  5.681200e+01      220.630   -560.8900  0.093693  2.825000e+01      0.5
8  5.692900e+01      221.870   -645.8400  0.081984  3.416600e+01      0.5
9  5.702800e+01      222.920   -730.4900  0.072728  4.053400e+01      0.5

[10 rows x 68 columns]
\end{sphinxVerbatim}
}

{
\sphinxsetup{VerbatimColor={named}{nbsphinx-code-bg}}
\sphinxsetup{VerbatimBorderColor={named}{nbsphinx-code-border}}
\begin{sphinxVerbatim}[commandchars=\\\{\}]
\llap{\color{nbsphinxin}[8]:\,\hspace{\fboxrule}\hspace{\fboxsep}}\PYG{c+c1}{\PYGZsh{} correlation between SRT and output}
\PYG{n}{plt}\PYG{o}{.}\PYG{n}{figure}\PYG{p}{(}\PYG{n}{figsize}\PYG{o}{=}\PYG{p}{(}\PYG{l+m+mi}{24}\PYG{p}{,}\PYG{l+m+mi}{1}\PYG{p}{)}\PYG{p}{)}
\PYG{n}{corr1}\PYG{o}{=}\PYG{n}{df1\PYGZus{}resx}\PYG{o}{.}\PYG{n}{corr}\PYG{p}{(}\PYG{p}{)}
\PYG{n}{sns}\PYG{o}{.}\PYG{n}{heatmap}\PYG{p}{(}\PYG{n}{corr1}\PYG{o}{.}\PYG{n}{iloc}\PYG{p}{[}\PYG{l+m+mi}{0}\PYG{p}{:}\PYG{l+m+mi}{1}\PYG{p}{,}\PYG{o}{\PYGZhy{}}\PYG{l+m+mi}{67}\PYG{p}{:}\PYG{p}{]}\PYG{p}{,} \PYG{n}{xticklabels}\PYG{o}{=}\PYG{n}{df1\PYGZus{}resx}\PYG{o}{.}\PYG{n}{columns}\PYG{p}{[}\PYG{o}{\PYGZhy{}}\PYG{l+m+mi}{67}\PYG{p}{:}\PYG{p}{]}\PYG{p}{,} \PYG{n}{yticklabels}\PYG{o}{=}\PYG{n}{df1\PYGZus{}resx}\PYG{o}{.}\PYG{n}{columns}\PYG{p}{[}\PYG{l+m+mi}{0}\PYG{p}{:}\PYG{l+m+mi}{1}\PYG{p}{]}\PYG{p}{,} \PYG{n}{cmap}\PYG{o}{=}\PYG{n}{sns}\PYG{o}{.}\PYG{n}{diverging\PYGZus{}palette}\PYG{p}{(}\PYG{l+m+mi}{220}\PYG{p}{,} \PYG{l+m+mi}{10}\PYG{p}{,} \PYG{n}{as\PYGZus{}cmap}\PYG{o}{=}\PYG{k+kc}{True}\PYG{p}{)}\PYG{p}{)}
\PYG{n}{plt}\PYG{o}{.}\PYG{n}{title}\PYG{p}{(}\PYG{l+s+s1}{\PYGZsq{}}\PYG{l+s+s1}{Correlation Matrix [t\PYGZus{}resx/results]}\PYG{l+s+s1}{\PYGZsq{}}\PYG{p}{,}\PYG{n}{fontsize}\PYG{o}{=}\PYG{l+m+mi}{20}\PYG{p}{)}\PYG{p}{;}
\end{sphinxVerbatim}
}

\hrule height -\fboxrule\relax
\vspace{\nbsphinxcodecellspacing}

\makeatletter\setbox\nbsphinxpromptbox\box\voidb@x\makeatother

\begin{nbsphinxfancyoutput}

\noindent\sphinxincludegraphics[width=1204\sphinxpxdimen,height=159\sphinxpxdimen]{{jupyter_notebook_ADM1F_SRT_2phase_11_0}.png}

\end{nbsphinxfancyoutput}

\sphinxAtStartPar
\sphinxstylestrong{Configurations}


\begin{savenotes}\sphinxattablestart
\centering
\begin{tabulary}{\linewidth}[t]{|T|T|T|T|}
\hline
\sphinxstyletheadfamily 
\sphinxAtStartPar
Configuration
&\sphinxstyletheadfamily 
\sphinxAtStartPar
Vliq (m\(^3\))
&\sphinxstyletheadfamily 
\sphinxAtStartPar
t\_resx (d)
&\sphinxstyletheadfamily 
\sphinxAtStartPar
Q (m\(^3\)/d)
\\
\hline
\sphinxAtStartPar
Default
&
\sphinxAtStartPar
3400
&
\sphinxAtStartPar
0
&
\sphinxAtStartPar
134
\\
\hline
\sphinxAtStartPar
Phase 1
&
\sphinxAtStartPar
340
&
\sphinxAtStartPar
1.5
&
\sphinxAtStartPar
618
\\
\hline
\sphinxAtStartPar
Phase 2
&
\sphinxAtStartPar
3400
&
\sphinxAtStartPar
700
&
\sphinxAtStartPar
618/—
\\
\hline
\end{tabulary}
\par
\sphinxattableend\end{savenotes}

\sphinxAtStartPar
where t\_resx = SRT \sphinxhyphen{} HRT

{
\sphinxsetup{VerbatimColor={named}{nbsphinx-code-bg}}
\sphinxsetup{VerbatimBorderColor={named}{nbsphinx-code-border}}
\begin{sphinxVerbatim}[commandchars=\\\{\}]
\llap{\color{nbsphinxin}[9]:\,\hspace{\fboxrule}\hspace{\fboxsep}}\PYG{n}{config\PYGZus{}default} \PYG{o}{=} \PYG{p}{\PYGZob{}}\PYG{l+s+s1}{\PYGZsq{}}\PYG{l+s+s1}{Vliq}\PYG{l+s+s1}{\PYGZsq{}}\PYG{p}{:}\PYG{l+m+mi}{3400}\PYG{p}{,} \PYG{l+s+s1}{\PYGZsq{}}\PYG{l+s+s1}{t\PYGZus{}resx}\PYG{l+s+s1}{\PYGZsq{}}\PYG{p}{:}\PYG{l+m+mi}{0}\PYG{p}{,} \PYG{l+s+s1}{\PYGZsq{}}\PYG{l+s+s1}{Q}\PYG{l+s+s1}{\PYGZsq{}}\PYG{p}{:}\PYG{l+m+mi}{134}\PYG{p}{\PYGZcb{}}
\PYG{n}{config1} \PYG{o}{=} \PYG{p}{\PYGZob{}}\PYG{l+s+s1}{\PYGZsq{}}\PYG{l+s+s1}{Vliq}\PYG{l+s+s1}{\PYGZsq{}}\PYG{p}{:}\PYG{l+m+mi}{340}\PYG{p}{,} \PYG{l+s+s1}{\PYGZsq{}}\PYG{l+s+s1}{t\PYGZus{}resx}\PYG{l+s+s1}{\PYGZsq{}}\PYG{p}{:}\PYG{l+m+mf}{1.5}\PYG{p}{,} \PYG{l+s+s1}{\PYGZsq{}}\PYG{l+s+s1}{Q}\PYG{l+s+s1}{\PYGZsq{}}\PYG{p}{:}\PYG{l+m+mi}{618}\PYG{p}{\PYGZcb{}}
\PYG{n}{config2} \PYG{o}{=} \PYG{p}{\PYGZob{}}\PYG{l+s+s1}{\PYGZsq{}}\PYG{l+s+s1}{Vliq}\PYG{l+s+s1}{\PYGZsq{}}\PYG{p}{:}\PYG{l+m+mi}{3400}\PYG{p}{,} \PYG{l+s+s1}{\PYGZsq{}}\PYG{l+s+s1}{t\PYGZus{}resx}\PYG{l+s+s1}{\PYGZsq{}}\PYG{p}{:}\PYG{l+m+mi}{700}\PYG{p}{,} \PYG{l+s+s1}{\PYGZsq{}}\PYG{l+s+s1}{Q}\PYG{l+s+s1}{\PYGZsq{}}\PYG{p}{:}\PYG{l+m+mi}{618}\PYG{p}{\PYGZcb{}}
\end{sphinxVerbatim}
}


\subsubsection{Default Configuration}
\label{\detokenize{jupyter_notebook/ADM1F_SRT_2phase:Default-Configuration}}
{
\sphinxsetup{VerbatimColor={named}{nbsphinx-code-bg}}
\sphinxsetup{VerbatimBorderColor={named}{nbsphinx-code-border}}
\begin{sphinxVerbatim}[commandchars=\\\{\}]
\llap{\color{nbsphinxin}[10]:\,\hspace{\fboxrule}\hspace{\fboxsep}}\PYG{c+c1}{\PYGZsh{} output using default configuration}
\PYG{n}{ls\PYGZus{}default} \PYG{o}{=} \PYG{n}{adm1fu}\PYG{o}{.}\PYG{n}{reactor1}\PYG{p}{(}\PYG{o}{*}\PYG{o}{*}\PYG{n}{config\PYGZus{}default}\PYG{p}{)}\PYG{o}{.}\PYG{n}{tolist}\PYG{p}{(}\PYG{p}{)}
\PYG{n}{df\PYGZus{}default} \PYG{o}{=} \PYG{n}{pd}\PYG{o}{.}\PYG{n}{DataFrame}\PYG{p}{(}\PYG{n}{data} \PYG{o}{=} \PYG{p}{[}\PYG{n}{ls\PYGZus{}default}\PYG{p}{]}\PYG{p}{,}\PYG{n}{columns}\PYG{o}{=}\PYG{n}{output\PYGZus{}name}\PYG{p}{,} \PYG{n}{index}\PYG{o}{=}\PYG{p}{[}\PYG{l+s+s1}{\PYGZsq{}}\PYG{l+s+s1}{Default Configuration}\PYG{l+s+s1}{\PYGZsq{}}\PYG{p}{]}\PYG{p}{)}
\PYG{n}{df\PYGZus{}default}
\end{sphinxVerbatim}
}

{

\kern-\sphinxverbatimsmallskipamount\kern-\baselineskip
\kern+\FrameHeightAdjust\kern-\fboxrule
\vspace{\nbsphinxcodecellspacing}

\sphinxsetup{VerbatimColor={named}{white}}
\sphinxsetup{VerbatimBorderColor={named}{nbsphinx-code-border}}
\begin{sphinxVerbatim}[commandchars=\\\{\}]
Reactor run, phase-one:
\$ADM1F\_EXE  -Vliq 3400 -t\_resx 0 -influent\_file influent\_cur.dat
indicator-034.out
\end{sphinxVerbatim}
}

{

\kern-\sphinxverbatimsmallskipamount\kern-\baselineskip
\kern+\FrameHeightAdjust\kern-\fboxrule
\vspace{\nbsphinxcodecellspacing}

\sphinxsetup{VerbatimColor={named}{white}}
\sphinxsetup{VerbatimBorderColor={named}{nbsphinx-code-border}}
\begin{sphinxVerbatim}[commandchars=\\\{\}]
\llap{\color{nbsphinxout}[10]:\,\hspace{\fboxrule}\hspace{\fboxsep}}                          Ssu      Saa      Sfa      Sva      Sbu     Spro   \textbackslash{}
Default Configuration  7.17674  3.21796  54.9446  6.39778  8.23228  6.08339

                         Sac      Sh2     Sch4      Sic   {\ldots}     Alk   \textbackslash{}
Default Configuration  1963.6  0.00012  48.3303  639.764  {\ldots}  8438.52

                          NH3      NH4     LCFA    percentch4    energych4   \textbackslash{}
Default Configuration  9.32395  1026.69  54.9446        56.905       65.228

                       efficiency   VFA/ALK      ACN  sampleT
Default Configuration     53.3712  0.220453  109.164  25.3731

[1 rows x 67 columns]
\end{sphinxVerbatim}
}


\subsubsection{Configuration 1}
\label{\detokenize{jupyter_notebook/ADM1F_SRT_2phase:Configuration-1}}
{
\sphinxsetup{VerbatimColor={named}{nbsphinx-code-bg}}
\sphinxsetup{VerbatimBorderColor={named}{nbsphinx-code-border}}
\begin{sphinxVerbatim}[commandchars=\\\{\}]
\llap{\color{nbsphinxin}[11]:\,\hspace{\fboxrule}\hspace{\fboxsep}}\PYG{c+c1}{\PYGZsh{} output using configuration 1}
\PYG{n}{ls\PYGZus{}config1} \PYG{o}{=} \PYG{n}{adm1fu}\PYG{o}{.}\PYG{n}{reactor1}\PYG{p}{(}\PYG{o}{*}\PYG{o}{*}\PYG{n}{config1}\PYG{p}{)}\PYG{o}{.}\PYG{n}{tolist}\PYG{p}{(}\PYG{p}{)}
\PYG{n}{df\PYGZus{}config1} \PYG{o}{=} \PYG{n}{pd}\PYG{o}{.}\PYG{n}{DataFrame}\PYG{p}{(}\PYG{n}{data} \PYG{o}{=} \PYG{p}{[}\PYG{n}{ls\PYGZus{}config1}\PYG{p}{]}\PYG{p}{,}\PYG{n}{columns}\PYG{o}{=}\PYG{n}{output\PYGZus{}name}\PYG{p}{,} \PYG{n}{index}\PYG{o}{=}\PYG{p}{[}\PYG{l+s+s1}{\PYGZsq{}}\PYG{l+s+s1}{Phase 1 Configuration}\PYG{l+s+s1}{\PYGZsq{}}\PYG{p}{]}\PYG{p}{)}
\PYG{n}{df\PYGZus{}config1}
\end{sphinxVerbatim}
}

{

\kern-\sphinxverbatimsmallskipamount\kern-\baselineskip
\kern+\FrameHeightAdjust\kern-\fboxrule
\vspace{\nbsphinxcodecellspacing}

\sphinxsetup{VerbatimColor={named}{white}}
\sphinxsetup{VerbatimBorderColor={named}{nbsphinx-code-border}}
\begin{sphinxVerbatim}[commandchars=\\\{\}]
Reactor run, phase-one:
\$ADM1F\_EXE  -Vliq 340 -t\_resx 1.5 -influent\_file influent\_cur.dat
indicator-051.out
\end{sphinxVerbatim}
}

{

\kern-\sphinxverbatimsmallskipamount\kern-\baselineskip
\kern+\FrameHeightAdjust\kern-\fboxrule
\vspace{\nbsphinxcodecellspacing}

\sphinxsetup{VerbatimColor={named}{white}}
\sphinxsetup{VerbatimBorderColor={named}{nbsphinx-code-border}}
\begin{sphinxVerbatim}[commandchars=\\\{\}]
\llap{\color{nbsphinxout}[11]:\,\hspace{\fboxrule}\hspace{\fboxsep}}                          Ssu      Saa      Sfa      Sva      Sbu     Spro   \textbackslash{}
Phase 1 Configuration  80.5616  33.4704  7808.75  2598.14  4246.87  3285.25

                          Sac      Sh2          Sch4      Sic   {\ldots}     Alk   \textbackslash{}
Phase 1 Configuration  9696.05  28.5069  7.737870e-30  147.214  {\ldots}  60316.4

                           NH3     NH4     LCFA    percentch4    energych4   \textbackslash{}
Phase 1 Configuration  0.048011  1145.6  7808.75  3.164870e-30      104.203

                       efficiency   VFA/ALK           ACN   sampleT
Phase 1 Configuration    -104.737  0.308161  1.202840e-17  0.550162

[1 rows x 67 columns]
\end{sphinxVerbatim}
}


\subsubsection{Configuration 2}
\label{\detokenize{jupyter_notebook/ADM1F_SRT_2phase:Configuration-2}}
{
\sphinxsetup{VerbatimColor={named}{nbsphinx-code-bg}}
\sphinxsetup{VerbatimBorderColor={named}{nbsphinx-code-border}}
\begin{sphinxVerbatim}[commandchars=\\\{\}]
\llap{\color{nbsphinxin}[12]:\,\hspace{\fboxrule}\hspace{\fboxsep}}\PYG{c+c1}{\PYGZsh{} output using configuration 2}
\PYG{n}{ls\PYGZus{}config2} \PYG{o}{=} \PYG{n}{adm1fu}\PYG{o}{.}\PYG{n}{reactor1}\PYG{p}{(}\PYG{o}{*}\PYG{o}{*}\PYG{n}{config2}\PYG{p}{)}\PYG{o}{.}\PYG{n}{tolist}\PYG{p}{(}\PYG{p}{)}
\PYG{n}{df\PYGZus{}config2} \PYG{o}{=} \PYG{n}{pd}\PYG{o}{.}\PYG{n}{DataFrame}\PYG{p}{(}\PYG{n}{data} \PYG{o}{=} \PYG{p}{[}\PYG{n}{ls\PYGZus{}config2}\PYG{p}{]}\PYG{p}{,}\PYG{n}{columns}\PYG{o}{=}\PYG{n}{output\PYGZus{}name}\PYG{p}{,} \PYG{n}{index}\PYG{o}{=}\PYG{p}{[}\PYG{l+s+s1}{\PYGZsq{}}\PYG{l+s+s1}{Phase 2 Configuration}\PYG{l+s+s1}{\PYGZsq{}}\PYG{p}{]}\PYG{p}{)}
\PYG{n}{df\PYGZus{}config2}
\end{sphinxVerbatim}
}

{

\kern-\sphinxverbatimsmallskipamount\kern-\baselineskip
\kern+\FrameHeightAdjust\kern-\fboxrule
\vspace{\nbsphinxcodecellspacing}

\sphinxsetup{VerbatimColor={named}{white}}
\sphinxsetup{VerbatimBorderColor={named}{nbsphinx-code-border}}
\begin{sphinxVerbatim}[commandchars=\\\{\}]
Reactor run, phase-one:
\$ADM1F\_EXE  -Vliq 3400 -t\_resx 700 -influent\_file influent\_cur.dat
indicator-025.out
\end{sphinxVerbatim}
}

{

\kern-\sphinxverbatimsmallskipamount\kern-\baselineskip
\kern+\FrameHeightAdjust\kern-\fboxrule
\vspace{\nbsphinxcodecellspacing}

\sphinxsetup{VerbatimColor={named}{white}}
\sphinxsetup{VerbatimBorderColor={named}{nbsphinx-code-border}}
\begin{sphinxVerbatim}[commandchars=\\\{\}]
\llap{\color{nbsphinxout}[12]:\,\hspace{\fboxrule}\hspace{\fboxsep}}                          Ssu      Saa      Sfa      Sva      Sbu     Spro   \textbackslash{}
Phase 2 Configuration  2.70068  1.20723  19.0168  2.34856  3.01652  2.16597

                          Sac       Sh2     Sch4      Sic   {\ldots}     Alk   \textbackslash{}
Phase 2 Configuration  1125.18  0.000045  75.1774  952.376  {\ldots}  21574.7

                          NH3      NH4     LCFA    percentch4    energych4   \textbackslash{}
Phase 2 Configuration  14.7549  1186.64  19.0168        58.194      255.569

                       efficiency   VFA/ALK      ACN  sampleT
Phase 2 Configuration    -899.833  0.049221  3302.96  5.50162

[1 rows x 67 columns]
\end{sphinxVerbatim}
}


\subsection{Two\sphinxhyphen{}Phase anaerobic dynamic membrane bioreactor}
\label{\detokenize{jupyter_notebook/ADM1F_SRT_2phase:Two-Phase-anaerobic-dynamic-membrane-bioreactor}}
\sphinxAtStartPar
To simulate a novel two\sphinxhyphen{}phase AnDMBR, We can call the function \sphinxcode{\sphinxupquote{reactor2}} with different Q1 (phase 1 flow rate), Q2 (phase 2 flow rate, normally equal to Q1), Vliq1 (phase 1 volume), Vliq2 (phase 2 volume), t\_resx1 (SRT\sphinxhyphen{}HRT for phase 1), and t\_resx2 (SRT\sphinxhyphen{}HRT for phase 1) values. In this function, the model output from phase 1 will be extracted and converted to the model input for phase 2. The function will return phase1 output and phase2 output

\sphinxAtStartPar
Usage example: reactor2(Q1=100, t\_resx1=30, t\_resx2=100, Vliq1=300, Vliq2=3000)

{
\sphinxsetup{VerbatimColor={named}{nbsphinx-code-bg}}
\sphinxsetup{VerbatimBorderColor={named}{nbsphinx-code-border}}
\begin{sphinxVerbatim}[commandchars=\\\{\}]
\llap{\color{nbsphinxin}[13]:\,\hspace{\fboxrule}\hspace{\fboxsep}}\PYG{k+kn}{from} \PYG{n+nn}{IPython}\PYG{n+nn}{.}\PYG{n+nn}{display} \PYG{k+kn}{import} \PYG{n}{Image}
\PYG{n+nb}{print} \PYG{p}{(}\PYG{l+s+s1}{\PYGZsq{}}\PYG{l+s+s1}{Schematics of the two\PYGZhy{}phase reactor}\PYG{l+s+s1}{\PYGZsq{}}\PYG{p}{)}
\PYG{n}{Image}\PYG{p}{(}\PYG{n}{filename}\PYG{o}{=}\PYG{l+s+s1}{\PYGZsq{}}\PYG{l+s+s1}{../notebooks/2phase\PYGZus{}reactor.png}\PYG{l+s+s1}{\PYGZsq{}}\PYG{p}{)}
\end{sphinxVerbatim}
}

{

\kern-\sphinxverbatimsmallskipamount\kern-\baselineskip
\kern+\FrameHeightAdjust\kern-\fboxrule
\vspace{\nbsphinxcodecellspacing}

\sphinxsetup{VerbatimColor={named}{white}}
\sphinxsetup{VerbatimBorderColor={named}{nbsphinx-code-border}}
\begin{sphinxVerbatim}[commandchars=\\\{\}]
Schematics of the two-phase reactor
\end{sphinxVerbatim}
}

\hrule height -\fboxrule\relax
\vspace{\nbsphinxcodecellspacing}

\savebox\nbsphinxpromptbox[0pt][r]{\color{nbsphinxout}\Verb|\strut{[13]:}\,|}

\begin{nbsphinxfancyoutput}

\noindent\sphinxincludegraphics[width=1236\sphinxpxdimen,height=634\sphinxpxdimen]{{jupyter_notebook_ADM1F_SRT_2phase_21_1}.png}

\end{nbsphinxfancyoutput}


\subsubsection{Config12}
\label{\detokenize{jupyter_notebook/ADM1F_SRT_2phase:Config12}}
\sphinxAtStartPar
\sphinxstylestrong{Configurations}


\begin{savenotes}\sphinxattablestart
\centering
\begin{tabulary}{\linewidth}[t]{|T|T|T|T|}
\hline
\sphinxstyletheadfamily 
\sphinxAtStartPar
Configuration
&\sphinxstyletheadfamily 
\sphinxAtStartPar
Vliq (m\(^3\))
&\sphinxstyletheadfamily 
\sphinxAtStartPar
t\_resx (d)
&\sphinxstyletheadfamily 
\sphinxAtStartPar
Q (m\(^3\)/d)
\\
\hline
\sphinxAtStartPar
Default
&
\sphinxAtStartPar
3400
&
\sphinxAtStartPar
0
&
\sphinxAtStartPar
134
\\
\hline
\sphinxAtStartPar
Phase 1
&
\sphinxAtStartPar
340
&
\sphinxAtStartPar
1.5
&
\sphinxAtStartPar
618
\\
\hline
\sphinxAtStartPar
Phase 2
&
\sphinxAtStartPar
3400
&
\sphinxAtStartPar
700
&
\sphinxAtStartPar
618/—
\\
\hline
\end{tabulary}
\par
\sphinxattableend\end{savenotes}

\sphinxAtStartPar
where t\_resx = SRT \sphinxhyphen{} HRT

{
\sphinxsetup{VerbatimColor={named}{nbsphinx-code-bg}}
\sphinxsetup{VerbatimBorderColor={named}{nbsphinx-code-border}}
\begin{sphinxVerbatim}[commandchars=\\\{\}]
\llap{\color{nbsphinxin}[14]:\,\hspace{\fboxrule}\hspace{\fboxsep}}\PYG{c+c1}{\PYGZsh{} config12 = \PYGZob{}\PYGZdq{}Vliq1\PYGZdq{}:340, \PYGZdq{}Vliq2\PYGZdq{}:3400, \PYGZdq{}t\PYGZus{}resx1\PYGZdq{}:1.5, \PYGZdq{}t\PYGZus{}resx2\PYGZdq{}:700, \PYGZdq{}Q1\PYGZdq{}:618, \PYGZdq{}Q2\PYGZdq{}:618\PYGZcb{}}
\PYG{n}{config12} \PYG{o}{=} \PYG{p}{\PYGZob{}}\PYG{l+s+s2}{\PYGZdq{}}\PYG{l+s+s2}{Vliq1}\PYG{l+s+s2}{\PYGZdq{}}\PYG{p}{:}\PYG{l+m+mi}{340}\PYG{p}{,} \PYG{l+s+s2}{\PYGZdq{}}\PYG{l+s+s2}{Vliq2}\PYG{l+s+s2}{\PYGZdq{}}\PYG{p}{:}\PYG{l+m+mi}{3400}\PYG{p}{,} \PYG{l+s+s2}{\PYGZdq{}}\PYG{l+s+s2}{t\PYGZus{}resx1}\PYG{l+s+s2}{\PYGZdq{}}\PYG{p}{:}\PYG{l+m+mf}{1.5}\PYG{p}{,} \PYG{l+s+s2}{\PYGZdq{}}\PYG{l+s+s2}{t\PYGZus{}resx2}\PYG{l+s+s2}{\PYGZdq{}}\PYG{p}{:}\PYG{l+m+mi}{700}\PYG{p}{,} \PYG{l+s+s2}{\PYGZdq{}}\PYG{l+s+s2}{Q1}\PYG{l+s+s2}{\PYGZdq{}}\PYG{p}{:}\PYG{l+m+mi}{618}\PYG{p}{\PYGZcb{}}
\end{sphinxVerbatim}
}

{
\sphinxsetup{VerbatimColor={named}{nbsphinx-code-bg}}
\sphinxsetup{VerbatimBorderColor={named}{nbsphinx-code-border}}
\begin{sphinxVerbatim}[commandchars=\\\{\}]
\llap{\color{nbsphinxin}[15]:\,\hspace{\fboxrule}\hspace{\fboxsep}}\PYG{n}{result\PYGZus{}config12} \PYG{o}{=} \PYG{n}{adm1fu}\PYG{o}{.}\PYG{n}{reactor2}\PYG{p}{(}\PYG{o}{*}\PYG{o}{*}\PYG{n}{config12}\PYG{p}{)}
\end{sphinxVerbatim}
}

{

\kern-\sphinxverbatimsmallskipamount\kern-\baselineskip
\kern+\FrameHeightAdjust\kern-\fboxrule
\vspace{\nbsphinxcodecellspacing}

\sphinxsetup{VerbatimColor={named}{white}}
\sphinxsetup{VerbatimBorderColor={named}{nbsphinx-code-border}}
\begin{sphinxVerbatim}[commandchars=\\\{\}]
Reactor run, phase-one:
\$ADM1F\_EXE  -Vliq 340 -t\_resx 1.5 -influent\_file influent\_cur.dat
indicator-051.out
Reactor run, phase-two:
\$ADM1F\_EXE -Vliq 3400 -t\_resx 700 -influent\_file influent\_cur.dat
indicator-024.out
\end{sphinxVerbatim}
}

{
\sphinxsetup{VerbatimColor={named}{nbsphinx-code-bg}}
\sphinxsetup{VerbatimBorderColor={named}{nbsphinx-code-border}}
\begin{sphinxVerbatim}[commandchars=\\\{\}]
\llap{\color{nbsphinxin}[16]:\,\hspace{\fboxrule}\hspace{\fboxsep}}\PYG{c+c1}{\PYGZsh{} output using config12}
\PYG{n}{result\PYGZus{}config12} \PYG{o}{=} \PYG{n}{adm1fu}\PYG{o}{.}\PYG{n}{reactor2}\PYG{p}{(}\PYG{o}{*}\PYG{o}{*}\PYG{n}{config12}\PYG{p}{)}
\PYG{n}{ls1\PYGZus{}config12} \PYG{o}{=} \PYG{n}{result\PYGZus{}config12}\PYG{p}{[}\PYG{l+m+mi}{0}\PYG{p}{]}\PYG{o}{.}\PYG{n}{tolist}\PYG{p}{(}\PYG{p}{)}
\PYG{n}{ls2\PYGZus{}config12} \PYG{o}{=} \PYG{n}{result\PYGZus{}config12}\PYG{p}{[}\PYG{l+m+mi}{1}\PYG{p}{]}\PYG{o}{.}\PYG{n}{tolist}\PYG{p}{(}\PYG{p}{)}

\PYG{n}{df\PYGZus{}config12} \PYG{o}{=} \PYG{n}{pd}\PYG{o}{.}\PYG{n}{DataFrame}\PYG{p}{(}\PYG{n}{data} \PYG{o}{=} \PYG{p}{[}\PYG{n}{ls1\PYGZus{}config12}\PYG{p}{]}\PYG{p}{,}\PYG{n}{columns}\PYG{o}{=}\PYG{n}{output\PYGZus{}name}\PYG{p}{,}\PYG{n}{index}\PYG{o}{=}\PYG{p}{[}\PYG{l+s+s1}{\PYGZsq{}}\PYG{l+s+s1}{Phase1}\PYG{l+s+s1}{\PYGZsq{}}\PYG{p}{]}\PYG{p}{)}
\PYG{n}{df\PYGZus{}config12}\PYG{o}{.}\PYG{n}{loc}\PYG{p}{[}\PYG{l+s+s1}{\PYGZsq{}}\PYG{l+s+s1}{Phase2}\PYG{l+s+s1}{\PYGZsq{}}\PYG{p}{]} \PYG{o}{=} \PYG{n}{ls2\PYGZus{}config12}
\PYG{n}{df\PYGZus{}config12}
\end{sphinxVerbatim}
}

{

\kern-\sphinxverbatimsmallskipamount\kern-\baselineskip
\kern+\FrameHeightAdjust\kern-\fboxrule
\vspace{\nbsphinxcodecellspacing}

\sphinxsetup{VerbatimColor={named}{white}}
\sphinxsetup{VerbatimBorderColor={named}{nbsphinx-code-border}}
\begin{sphinxVerbatim}[commandchars=\\\{\}]
Reactor run, phase-one:
\$ADM1F\_EXE  -Vliq 340 -t\_resx 1.5 -influent\_file influent\_cur.dat
indicator-051.out
Reactor run, phase-two:
\$ADM1F\_EXE -Vliq 3400 -t\_resx 700 -influent\_file influent\_cur.dat
indicator-024.out
\end{sphinxVerbatim}
}

{

\kern-\sphinxverbatimsmallskipamount\kern-\baselineskip
\kern+\FrameHeightAdjust\kern-\fboxrule
\vspace{\nbsphinxcodecellspacing}

\sphinxsetup{VerbatimColor={named}{white}}
\sphinxsetup{VerbatimBorderColor={named}{nbsphinx-code-border}}
\begin{sphinxVerbatim}[commandchars=\\\{\}]
\llap{\color{nbsphinxout}[16]:\,\hspace{\fboxrule}\hspace{\fboxsep}}             Ssu        Saa        Sfa         Sva         Sbu        Spro   \textbackslash{}
Phase1  80.561600  33.470400  7808.7500  2598.14000  4246.87000  3285.25000
Phase2   0.355681   0.132504    19.2757     2.37729     3.05196     2.19399

           Sac        Sh2          Sch4      Sic   {\ldots}     Alk        NH3   \textbackslash{}
Phase1  9696.05  28.506900  7.771710e-32  147.214  {\ldots}  60316.4   0.048011
Phase2  1248.96   0.000046  7.114040e+01  867.356  {\ldots}  17101.8  15.289200

          NH4       LCFA    percentch4    energych4   efficiency   VFA/ALK  \textbackslash{}
Phase1  1145.6  7808.7500  3.068330e-32      104.203    -104.737  0.308161
Phase2  1139.4    19.2757  6.198330e+01      192.277    -590.307  0.068884

                 ACN   sampleT
Phase1  1.216580e-18  0.550162
Phase2 -2.807000e+01  6.241160

[2 rows x 67 columns]
\end{sphinxVerbatim}
}

{
\sphinxsetup{VerbatimColor={named}{nbsphinx-code-bg}}
\sphinxsetup{VerbatimBorderColor={named}{nbsphinx-code-border}}
\begin{sphinxVerbatim}[commandchars=\\\{\}]
\llap{\color{nbsphinxin}[ ]:\,\hspace{\fboxrule}\hspace{\fboxsep}}
\end{sphinxVerbatim}
}


\section{ADM1F\_SRT: Input/output sensitivity}
\label{\detokenize{jupyter_notebook/io_sensitivity_analysis:ADM1F_SRT:-Input/output-sensitivity}}\label{\detokenize{jupyter_notebook/io_sensitivity_analysis::doc}}
\sphinxAtStartPar
Here we explore the relationships between inputs and outputs. In the \sphinxcode{\sphinxupquote{ADM1F: Execution time}} example we showed how to run the models with the perturbed input values from \sphinxcode{\sphinxupquote{influent.dat}} and \sphinxcode{\sphinxupquote{param.dat}} files. Assuming that you run the \sphinxcode{\sphinxupquote{ADM1F: Execution time}} example and produced the \sphinxcode{\sphinxupquote{outputs\_influent.csv}} and \sphinxcode{\sphinxupquote{outputs\_params.csv}} files, we use these outputs here to study the relationship between influents and outputs, and params and outputs. If not just uncomment lines 5 and 18 and
re\sphinxhyphen{}run the simulations.

\sphinxAtStartPar
Authors: Wenjuan Zhang and Elchin Jafarov

{
\sphinxsetup{VerbatimColor={named}{nbsphinx-code-bg}}
\sphinxsetup{VerbatimBorderColor={named}{nbsphinx-code-border}}
\begin{sphinxVerbatim}[commandchars=\\\{\}]
\llap{\color{nbsphinxin}[1]:\,\hspace{\fboxrule}\hspace{\fboxsep}}\PYG{k+kn}{import} \PYG{n+nn}{adm1f\PYGZus{}utils} \PYG{k}{as} \PYG{n+nn}{adm1fu}
\PYG{k+kn}{import} \PYG{n+nn}{os}
\PYG{k+kn}{import} \PYG{n+nn}{pandas} \PYG{k}{as} \PYG{n+nn}{pd}
\PYG{k+kn}{import} \PYG{n+nn}{matplotlib}\PYG{n+nn}{.}\PYG{n+nn}{pyplot} \PYG{k}{as} \PYG{n+nn}{plt}
\PYG{k+kn}{import} \PYG{n+nn}{seaborn} \PYG{k}{as} \PYG{n+nn}{sns}
\PYG{o}{\PYGZpc{}}\PYG{k}{matplotlib} inline
\end{sphinxVerbatim}
}


\subsection{1. Influent/Output sensitivity}
\label{\detokenize{jupyter_notebook/io_sensitivity_analysis:1.-Influent/Output-sensitivity}}
{
\sphinxsetup{VerbatimColor={named}{nbsphinx-code-bg}}
\sphinxsetup{VerbatimBorderColor={named}{nbsphinx-code-border}}
\begin{sphinxVerbatim}[commandchars=\\\{\}]
\llap{\color{nbsphinxin}[2]:\,\hspace{\fboxrule}\hspace{\fboxsep}}\PYG{c+c1}{\PYGZsh{} navigate to simulations folder}
\PYG{n}{os}\PYG{o}{.}\PYG{n}{chdir}\PYG{p}{(}\PYG{l+s+s1}{\PYGZsq{}}\PYG{l+s+s1}{../../simulations}\PYG{l+s+s1}{\PYGZsq{}}\PYG{p}{)}
\end{sphinxVerbatim}
}

{
\sphinxsetup{VerbatimColor={named}{nbsphinx-code-bg}}
\sphinxsetup{VerbatimBorderColor={named}{nbsphinx-code-border}}
\begin{sphinxVerbatim}[commandchars=\\\{\}]
\llap{\color{nbsphinxin}[3]:\,\hspace{\fboxrule}\hspace{\fboxsep}}\PYG{c+c1}{\PYGZsh{}Set the path to the ADM1F executable}
\PYG{n}{ADM1F\PYGZus{}EXE} \PYG{o}{=} \PYG{l+s+s1}{\PYGZsq{}}\PYG{l+s+s1}{/Users/elchin/project/ADM1F\PYGZus{}WM/build/adm1f}\PYG{l+s+s1}{\PYGZsq{}}

\PYG{c+c1}{\PYGZsh{} Set the value of percentage and sample size for lhs}
\PYG{n}{percent} \PYG{o}{=} \PYG{l+m+mf}{0.1} \PYG{c+c1}{\PYGZsh{} NOTE: for params percent should be \PYGZlt{}= 0.05}
\PYG{n}{sample\PYGZus{}size} \PYG{o}{=} \PYG{l+m+mi}{100}
\PYG{n}{variable} \PYG{o}{=} \PYG{l+s+s1}{\PYGZsq{}}\PYG{l+s+s1}{influent}\PYG{l+s+s1}{\PYGZsq{}}    \PYG{c+c1}{\PYGZsh{} influent/params/ic}
\PYG{n}{method} \PYG{o}{=} \PYG{l+s+s1}{\PYGZsq{}}\PYG{l+s+s1}{uniform}\PYG{l+s+s1}{\PYGZsq{}}    \PYG{c+c1}{\PYGZsh{}\PYGZsq{}uniform\PYGZsq{} or \PYGZsq{}lhs\PYGZsq{}}
\end{sphinxVerbatim}
}

{
\sphinxsetup{VerbatimColor={named}{nbsphinx-code-bg}}
\sphinxsetup{VerbatimBorderColor={named}{nbsphinx-code-border}}
\begin{sphinxVerbatim}[commandchars=\\\{\}]
\llap{\color{nbsphinxin}[4]:\,\hspace{\fboxrule}\hspace{\fboxsep}}\PYG{n}{index}\PYG{o}{=}\PYG{n}{adm1fu}\PYG{o}{.}\PYG{n}{create\PYGZus{}a\PYGZus{}sample\PYGZus{}matrix}\PYG{p}{(}\PYG{n}{variable}\PYG{p}{,}\PYG{n}{method}\PYG{p}{,}\PYG{n}{percent}\PYG{p}{,}\PYG{n}{sample\PYGZus{}size}\PYG{p}{)}
\PYG{n+nb}{print} \PYG{p}{(}\PYG{p}{)}
\PYG{n+nb}{print} \PYG{p}{(}\PYG{l+s+s1}{\PYGZsq{}}\PYG{l+s+s1}{Number of elements participated in the sampling:}\PYG{l+s+s1}{\PYGZsq{}}\PYG{p}{,}\PYG{n+nb}{len}\PYG{p}{(}\PYG{n}{index}\PYG{p}{)}\PYG{p}{)}
\end{sphinxVerbatim}
}

{

\kern-\sphinxverbatimsmallskipamount\kern-\baselineskip
\kern+\FrameHeightAdjust\kern-\fboxrule
\vspace{\nbsphinxcodecellspacing}

\sphinxsetup{VerbatimColor={named}{white}}
\sphinxsetup{VerbatimBorderColor={named}{nbsphinx-code-border}}
\begin{sphinxVerbatim}[commandchars=\\\{\}]
Saves a sampling matrix [sample\_size,array\_size] into var\_influent.csv
sample\_size,array\_size:  (100, 11)
Each column of the matrix corresponds to a variable perturbed 100 times around its original value
var\_influent.csv SAVED!

Number of elements participated in the sampling: 11
\end{sphinxVerbatim}
}

{
\sphinxsetup{VerbatimColor={named}{nbsphinx-code-bg}}
\sphinxsetup{VerbatimBorderColor={named}{nbsphinx-code-border}}
\begin{sphinxVerbatim}[commandchars=\\\{\}]
\llap{\color{nbsphinxin}[5]:\,\hspace{\fboxrule}\hspace{\fboxsep}}\PYG{c+c1}{\PYGZsh{}exe\PYGZus{}time=adm1fu.adm1f\PYGZus{}output\PYGZus{}sampling(ADM1F\PYGZus{}EXE,variable,index)}
\end{sphinxVerbatim}
}

{
\sphinxsetup{VerbatimColor={named}{nbsphinx-code-bg}}
\sphinxsetup{VerbatimBorderColor={named}{nbsphinx-code-border}}
\begin{sphinxVerbatim}[commandchars=\\\{\}]
\llap{\color{nbsphinxin}[6]:\,\hspace{\fboxrule}\hspace{\fboxsep}}\PYG{p}{[}\PYG{n}{output\PYGZus{}name}\PYG{p}{,}\PYG{n}{output\PYGZus{}unit}\PYG{p}{]}\PYG{o}{=}\PYG{n}{adm1fu}\PYG{o}{.}\PYG{n}{get\PYGZus{}output\PYGZus{}names}\PYG{p}{(}\PYG{p}{)}
\PYG{n}{alloutputs} \PYG{o}{=} \PYG{n}{pd}\PYG{o}{.}\PYG{n}{read\PYGZus{}csv}\PYG{p}{(}\PYG{l+s+s1}{\PYGZsq{}}\PYG{l+s+s1}{outputs\PYGZus{}influent.csv}\PYG{l+s+s1}{\PYGZsq{}}\PYG{p}{,} \PYG{n}{sep}\PYG{o}{=}\PYG{l+s+s1}{\PYGZsq{}}\PYG{l+s+s1}{,}\PYG{l+s+s1}{\PYGZsq{}}\PYG{p}{,} \PYG{n}{header}\PYG{o}{=}\PYG{k+kc}{None}\PYG{p}{)}
\PYG{n}{alloutputs}\PYG{o}{.}\PYG{n}{columns} \PYG{o}{=} \PYG{n}{output\PYGZus{}name}
\end{sphinxVerbatim}
}

{
\sphinxsetup{VerbatimColor={named}{nbsphinx-code-bg}}
\sphinxsetup{VerbatimBorderColor={named}{nbsphinx-code-border}}
\begin{sphinxVerbatim}[commandchars=\\\{\}]
\llap{\color{nbsphinxin}[7]:\,\hspace{\fboxrule}\hspace{\fboxsep}}\PYG{n}{alloutputs}
\end{sphinxVerbatim}
}

{

\kern-\sphinxverbatimsmallskipamount\kern-\baselineskip
\kern+\FrameHeightAdjust\kern-\fboxrule
\vspace{\nbsphinxcodecellspacing}

\sphinxsetup{VerbatimColor={named}{white}}
\sphinxsetup{VerbatimBorderColor={named}{nbsphinx-code-border}}
\begin{sphinxVerbatim}[commandchars=\\\{\}]
\llap{\color{nbsphinxout}[7]:\,\hspace{\fboxrule}\hspace{\fboxsep}}       Ssu      Saa      Sfa      Sva      Sbu     Spro      Sac       Sh2   \textbackslash{}
0   7.37820  3.30795  56.7230  6.60272  8.46032  6.26942  1908.38  0.000123
1   7.28460  3.26614  55.8953  6.52078  8.34621  6.18289  2033.37  0.000121
2   6.85866  3.07582  52.1661  6.11321  7.84990  5.79147  1926.30  0.000114
3   6.72726  3.01709  51.0289  5.97094  7.70738  5.67153  1931.76  0.000112
4   7.55763  3.38810  58.3212  6.83699  8.63179  6.43593  2242.26  0.000126
..      {\ldots}      {\ldots}      {\ldots}      {\ldots}      {\ldots}      {\ldots}      {\ldots}       {\ldots}
95  7.65859  3.43319  59.2237  6.78604  8.84032  6.52986  1911.45  0.000128
96  7.40606  3.32040  56.9708  6.57190  8.52763  6.29525  2049.40  0.000123
97  7.06582  3.16840  53.9710  6.30301  8.09601  5.98132  1778.88  0.000118
98  6.85456  3.07399  52.1312  6.12749  7.83407  5.78776  2146.27  0.000114
99  7.44704  3.33870  57.3346  6.67541  8.53678  6.33321  1983.08  0.000124

      Sch4      Sic   {\ldots}     Alk       NH3       NH4     LCFA    percentch4   \textbackslash{}
0   48.7774  615.799  {\ldots}  8148.98   8.64671   987.952  56.7230       57.2720
1   48.4068  645.849  {\ldots}  8602.48   9.61694  1048.750  55.8953       56.8728
2   47.7048  651.060  {\ldots}  8442.25   9.57599  1031.740  52.1661       56.7510
3   47.6768  660.852  {\ldots}  8558.31   9.84058  1044.000  51.0289       56.6698
4   49.1049  657.957  {\ldots}  9061.95  10.49270  1109.360  58.3212       57.2318
..      {\ldots}      {\ldots}  {\ldots}      {\ldots}       {\ldots}       {\ldots}      {\ldots}           {\ldots}
95  48.9244  616.258  {\ldots}  8197.36   8.37148   985.507  59.2237       56.5753
96  48.6287  649.584  {\ldots}  8699.15   9.58421  1054.650  56.9708       56.5357
97  48.2397  605.950  {\ldots}  7819.47   8.22490   949.405  53.9710       57.3804
98  48.1629  685.484  {\ldots}  9144.20  11.07250  1119.840  52.1312       56.8556
99  48.5798  627.253  {\ldots}  8349.28   9.04090  1016.750  57.3346       57.0513

     energych4   efficiency   VFA/ALK       ACN  sampleT
0       66.2091     52.7615  0.222004  116.5620  24.3594
1       65.9860     53.8907  0.223891   92.1431  24.8179
2       61.6157     52.8110  0.216106  125.4410  27.1630
3       61.9098     53.9655  0.213730  116.0810  27.9786
4       68.8331     51.4620  0.234238   77.3666  23.5162
..          {\ldots}         {\ldots}       {\ldots}       {\ldots}      {\ldots}
95      71.6467     52.7621  0.221140  125.5700  23.0752
96      69.7230     54.2485  0.223168  102.1390  24.2222
97      61.6582     52.9718  0.215719  129.6520  25.9740
98      64.3136     53.8121  0.222069   88.1808  27.1834
99      66.1387     53.8114  0.225090  103.0960  24.0295

[100 rows x 67 columns]
\end{sphinxVerbatim}
}

{
\sphinxsetup{VerbatimColor={named}{nbsphinx-code-bg}}
\sphinxsetup{VerbatimBorderColor={named}{nbsphinx-code-border}}
\begin{sphinxVerbatim}[commandchars=\\\{\}]
\llap{\color{nbsphinxin}[8]:\,\hspace{\fboxrule}\hspace{\fboxsep}}\PYG{p}{[}\PYG{n}{influent\PYGZus{}name}\PYG{p}{,}\PYG{n}{influent\PYGZus{}index}\PYG{p}{]}\PYG{o}{=}\PYG{n}{adm1fu}\PYG{o}{.}\PYG{n}{get\PYGZus{}influent\PYGZus{}names}\PYG{p}{(}\PYG{p}{)}
\end{sphinxVerbatim}
}

{
\sphinxsetup{VerbatimColor={named}{nbsphinx-code-bg}}
\sphinxsetup{VerbatimBorderColor={named}{nbsphinx-code-border}}
\begin{sphinxVerbatim}[commandchars=\\\{\}]
\llap{\color{nbsphinxin}[9]:\,\hspace{\fboxrule}\hspace{\fboxsep}}\PYG{c+c1}{\PYGZsh{} since we did not use all the columns in the influent.dat (see create\PYGZus{}a\PYGZus{}sample\PYGZus{}matrix)}
\PYG{c+c1}{\PYGZsh{} we use index to select used headers for used values}
\PYG{n}{header}\PYG{o}{=}\PYG{p}{[}\PYG{p}{]}
\PYG{k}{for} \PYG{n}{i} \PYG{o+ow}{in} \PYG{n}{index}\PYG{p}{:}
    \PYG{n}{header}\PYG{o}{.}\PYG{n}{append}\PYG{p}{(}\PYG{n}{influent\PYGZus{}name}\PYG{p}{[}\PYG{n}{i}\PYG{p}{]}\PYG{p}{)}

\PYG{n}{influent\PYGZus{}inputs} \PYG{o}{=} \PYG{n}{pd}\PYG{o}{.}\PYG{n}{read\PYGZus{}csv}\PYG{p}{(}\PYG{l+s+s1}{\PYGZsq{}}\PYG{l+s+s1}{var\PYGZus{}influent.csv}\PYG{l+s+s1}{\PYGZsq{}}\PYG{p}{,} \PYG{n}{sep}\PYG{o}{=}\PYG{l+s+s1}{\PYGZsq{}}\PYG{l+s+s1}{,}\PYG{l+s+s1}{\PYGZsq{}}\PYG{p}{,} \PYG{n}{header}\PYG{o}{=}\PYG{k+kc}{None}\PYG{p}{)}
\PYG{n}{influent\PYGZus{}inputs}\PYG{o}{.}\PYG{n}{columns} \PYG{o}{=} \PYG{n}{header}
\PYG{n}{influent\PYGZus{}inputs}\PYG{o}{.}\PYG{n}{head}\PYG{p}{(}\PYG{p}{)}
\end{sphinxVerbatim}
}

{

\kern-\sphinxverbatimsmallskipamount\kern-\baselineskip
\kern+\FrameHeightAdjust\kern-\fboxrule
\vspace{\nbsphinxcodecellspacing}

\sphinxsetup{VerbatimColor={named}{white}}
\sphinxsetup{VerbatimBorderColor={named}{nbsphinx-code-border}}
\begin{sphinxVerbatim}[commandchars=\\\{\}]
\llap{\color{nbsphinxout}[9]:\,\hspace{\fboxrule}\hspace{\fboxsep}}   S\_su\_in  S\_aa\_in  S\_fa\_in  S\_ac\_in  S\_IN\_in  X\_ch\_biom\_in  X\_pr\_biom\_in  \textbackslash{}
0  2.41679  4.54049  3.26551  1.06715  0.00794       8.21103       7.65897
1  2.71197  4.44197  2.94117  0.97991  0.00798       8.47247       8.44312
2  2.37594  4.05372  3.09329  1.10619  0.00801       8.84280       8.55680
3  2.70155  4.55292  3.31319  1.00561  0.00784       9.14260       8.30096
4  2.35939  4.30043  3.00319  1.05070  0.00860       8.26193       9.19056

   X\_li\_biom\_in    X\_I\_in          Q      Temp
0       5.45471  18.04704  139.57635  31.64409
1       5.01331  16.95071  136.99766  32.47646
2       4.62146  18.06979  125.17005  31.95536
3       4.69829  17.67225  121.52161  37.86524
4       5.36216  19.24420  144.58137  35.68530
\end{sphinxVerbatim}
}

{
\sphinxsetup{VerbatimColor={named}{nbsphinx-code-bg}}
\sphinxsetup{VerbatimBorderColor={named}{nbsphinx-code-border}}
\begin{sphinxVerbatim}[commandchars=\\\{\}]
\llap{\color{nbsphinxin}[10]:\,\hspace{\fboxrule}\hspace{\fboxsep}}\PYG{c+c1}{\PYGZsh{} merge influent and output datasets}
\PYG{n}{inout}\PYG{o}{=}\PYG{n}{pd}\PYG{o}{.}\PYG{n}{concat}\PYG{p}{(}\PYG{p}{[}\PYG{n}{influent\PYGZus{}inputs}\PYG{p}{,}\PYG{n}{alloutputs}\PYG{p}{]}\PYG{p}{,} \PYG{n}{axis}\PYG{o}{=}\PYG{l+m+mi}{1}\PYG{p}{)}
\PYG{n}{inout}\PYG{o}{.}\PYG{n}{head}\PYG{p}{(}\PYG{p}{)}
\end{sphinxVerbatim}
}

{

\kern-\sphinxverbatimsmallskipamount\kern-\baselineskip
\kern+\FrameHeightAdjust\kern-\fboxrule
\vspace{\nbsphinxcodecellspacing}

\sphinxsetup{VerbatimColor={named}{white}}
\sphinxsetup{VerbatimBorderColor={named}{nbsphinx-code-border}}
\begin{sphinxVerbatim}[commandchars=\\\{\}]
\llap{\color{nbsphinxout}[10]:\,\hspace{\fboxrule}\hspace{\fboxsep}}   S\_su\_in  S\_aa\_in  S\_fa\_in  S\_ac\_in  S\_IN\_in  X\_ch\_biom\_in  X\_pr\_biom\_in  \textbackslash{}
0  2.41679  4.54049  3.26551  1.06715  0.00794       8.21103       7.65897
1  2.71197  4.44197  2.94117  0.97991  0.00798       8.47247       8.44312
2  2.37594  4.05372  3.09329  1.10619  0.00801       8.84280       8.55680
3  2.70155  4.55292  3.31319  1.00561  0.00784       9.14260       8.30096
4  2.35939  4.30043  3.00319  1.05070  0.00860       8.26193       9.19056

   X\_li\_biom\_in    X\_I\_in          Q  {\ldots}     Alk       NH3       NH4   \textbackslash{}
0       5.45471  18.04704  139.57635  {\ldots}  8148.98   8.64671   987.952
1       5.01331  16.95071  136.99766  {\ldots}  8602.48   9.61694  1048.750
2       4.62146  18.06979  125.17005  {\ldots}  8442.25   9.57599  1031.740
3       4.69829  17.67225  121.52161  {\ldots}  8558.31   9.84058  1044.000
4       5.36216  19.24420  144.58137  {\ldots}  9061.95  10.49270  1109.360

     LCFA    percentch4    energych4   efficiency   VFA/ALK       ACN  sampleT
0  56.7230       57.2720      66.2091     52.7615  0.222004  116.5620  24.3594
1  55.8953       56.8728      65.9860     53.8907  0.223891   92.1431  24.8179
2  52.1661       56.7510      61.6157     52.8110  0.216106  125.4410  27.1630
3  51.0289       56.6698      61.9098     53.9655  0.213730  116.0810  27.9786
4  58.3212       57.2318      68.8331     51.4620  0.234238   77.3666  23.5162

[5 rows x 78 columns]
\end{sphinxVerbatim}
}

\sphinxAtStartPar
The correlation heat map matrix below shows that four influents have the highest impact on the results: \sphinxcode{\sphinxupquote{X\_ch\_biom\_in}}, \sphinxcode{\sphinxupquote{X\_pr\_biom\_in}}, \sphinxcode{\sphinxupquote{X\_li\_biom\_in}}, and \sphinxcode{\sphinxupquote{Q}}.

{
\sphinxsetup{VerbatimColor={named}{nbsphinx-code-bg}}
\sphinxsetup{VerbatimBorderColor={named}{nbsphinx-code-border}}
\begin{sphinxVerbatim}[commandchars=\\\{\}]
\llap{\color{nbsphinxin}[11]:\,\hspace{\fboxrule}\hspace{\fboxsep}}\PYG{n}{corr}\PYG{o}{=}\PYG{n}{inout}\PYG{o}{.}\PYG{n}{corr}\PYG{p}{(}\PYG{p}{)}
\PYG{n}{plt}\PYG{o}{.}\PYG{n}{figure}\PYG{p}{(}\PYG{n}{figsize}\PYG{o}{=}\PYG{p}{(}\PYG{l+m+mi}{21}\PYG{p}{,}\PYG{l+m+mi}{5}\PYG{p}{)}\PYG{p}{)}
\PYG{n}{sns}\PYG{o}{.}\PYG{n}{heatmap}\PYG{p}{(}\PYG{n}{corr}\PYG{o}{.}\PYG{n}{iloc}\PYG{p}{[}\PYG{l+m+mi}{0}\PYG{p}{:}\PYG{l+m+mi}{11}\PYG{p}{,}\PYG{l+m+mi}{11}\PYG{p}{:}\PYG{o}{\PYGZhy{}}\PYG{l+m+mi}{1}\PYG{p}{]}\PYG{p}{,} \PYG{n}{cmap}\PYG{o}{=}\PYG{n}{sns}\PYG{o}{.}\PYG{n}{diverging\PYGZus{}palette}\PYG{p}{(}\PYG{l+m+mi}{220}\PYG{p}{,} \PYG{l+m+mi}{10}\PYG{p}{,} \PYG{n}{as\PYGZus{}cmap}\PYG{o}{=}\PYG{k+kc}{True}\PYG{p}{)}\PYG{p}{)}
\PYG{n}{plt}\PYG{o}{.}\PYG{n}{title}\PYG{p}{(}\PYG{l+s+s1}{\PYGZsq{}}\PYG{l+s+s1}{Correlation Matrix [influent/results]}\PYG{l+s+s1}{\PYGZsq{}}\PYG{p}{,}\PYG{n}{fontsize}\PYG{o}{=}\PYG{l+m+mi}{16}\PYG{p}{)}\PYG{p}{;}
\PYG{n}{plt}\PYG{o}{.}\PYG{n}{ylabel}\PYG{p}{(}\PYG{l+s+s1}{\PYGZsq{}}\PYG{l+s+s1}{Influent}\PYG{l+s+s1}{\PYGZsq{}}\PYG{p}{,}\PYG{n}{fontsize}\PYG{o}{=}\PYG{l+m+mi}{16}\PYG{p}{)}
\PYG{n}{plt}\PYG{o}{.}\PYG{n}{xlabel}\PYG{p}{(}\PYG{l+s+s1}{\PYGZsq{}}\PYG{l+s+s1}{Outputs}\PYG{l+s+s1}{\PYGZsq{}}\PYG{p}{,}\PYG{n}{fontsize}\PYG{o}{=}\PYG{l+m+mi}{16}\PYG{p}{)}
\end{sphinxVerbatim}
}

{

\kern-\sphinxverbatimsmallskipamount\kern-\baselineskip
\kern+\FrameHeightAdjust\kern-\fboxrule
\vspace{\nbsphinxcodecellspacing}

\sphinxsetup{VerbatimColor={named}{white}}
\sphinxsetup{VerbatimBorderColor={named}{nbsphinx-code-border}}
\begin{sphinxVerbatim}[commandchars=\\\{\}]
\llap{\color{nbsphinxout}[11]:\,\hspace{\fboxrule}\hspace{\fboxsep}}Text(0.5, 23.09375, 'Outputs')
\end{sphinxVerbatim}
}

\hrule height -\fboxrule\relax
\vspace{\nbsphinxcodecellspacing}

\makeatletter\setbox\nbsphinxpromptbox\box\voidb@x\makeatother

\begin{nbsphinxfancyoutput}

\noindent\sphinxincludegraphics[width=1155\sphinxpxdimen,height=392\sphinxpxdimen]{{jupyter_notebook_io_sensitivity_analysis_13_1}.png}

\end{nbsphinxfancyoutput}

\sphinxAtStartPar
Let’s merge \sphinxcode{\sphinxupquote{ph}} values from the results with the influents and explore the correlations.

{
\sphinxsetup{VerbatimColor={named}{nbsphinx-code-bg}}
\sphinxsetup{VerbatimBorderColor={named}{nbsphinx-code-border}}
\begin{sphinxVerbatim}[commandchars=\\\{\}]
\llap{\color{nbsphinxin}[12]:\,\hspace{\fboxrule}\hspace{\fboxsep}}\PYG{n}{influent\PYGZus{}ph}\PYG{o}{=}\PYG{n}{pd}\PYG{o}{.}\PYG{n}{concat}\PYG{p}{(}\PYG{p}{[}\PYG{n}{influent\PYGZus{}inputs}\PYG{p}{,}\PYG{n}{alloutputs}\PYG{p}{[}\PYG{l+s+s1}{\PYGZsq{}}\PYG{l+s+s1}{ pH }\PYG{l+s+s1}{\PYGZsq{}}\PYG{p}{]}\PYG{p}{]}\PYG{p}{,} \PYG{n}{axis}\PYG{o}{=}\PYG{l+m+mi}{1}\PYG{p}{)}
\end{sphinxVerbatim}
}

{
\sphinxsetup{VerbatimColor={named}{nbsphinx-code-bg}}
\sphinxsetup{VerbatimBorderColor={named}{nbsphinx-code-border}}
\begin{sphinxVerbatim}[commandchars=\\\{\}]
\llap{\color{nbsphinxin}[13]:\,\hspace{\fboxrule}\hspace{\fboxsep}}\PYG{n}{plt}\PYG{o}{.}\PYG{n}{figure}\PYG{p}{(}\PYG{n}{figsize}\PYG{o}{=}\PYG{p}{(}\PYG{l+m+mi}{10}\PYG{p}{,}\PYG{l+m+mi}{4}\PYG{p}{)}\PYG{p}{)}
\PYG{n}{influent\PYGZus{}ph}\PYG{o}{.}\PYG{n}{corr}\PYG{p}{(}\PYG{p}{)}\PYG{o}{.}\PYG{n}{iloc}\PYG{p}{[}\PYG{o}{\PYGZhy{}}\PYG{l+m+mi}{1}\PYG{p}{]}\PYG{o}{.}\PYG{n}{plot}\PYG{p}{(}\PYG{n}{linewidth}\PYG{o}{=}\PYG{l+m+mi}{2}\PYG{p}{)}
\PYG{n}{plt}\PYG{o}{.}\PYG{n}{title}\PYG{p}{(}\PYG{l+s+s1}{\PYGZsq{}}\PYG{l+s+s1}{Correlation plot (ph)}\PYG{l+s+s1}{\PYGZsq{}}\PYG{p}{)}
\PYG{n}{plt}\PYG{o}{.}\PYG{n}{xlabel}\PYG{p}{(}\PYG{l+s+s1}{\PYGZsq{}}\PYG{l+s+s1}{Influents}\PYG{l+s+s1}{\PYGZsq{}}\PYG{p}{)}
\PYG{n}{plt}\PYG{o}{.}\PYG{n}{ylabel}\PYG{p}{(}\PYG{l+s+s1}{\PYGZsq{}}\PYG{l+s+s1}{Correlation}\PYG{l+s+s1}{\PYGZsq{}}\PYG{p}{)}
\PYG{n}{plt}\PYG{o}{.}\PYG{n}{ylim}\PYG{p}{(}\PYG{p}{[}\PYG{o}{\PYGZhy{}}\PYG{l+m+mi}{1}\PYG{p}{,}\PYG{l+m+mi}{1}\PYG{p}{]}\PYG{p}{)}
\end{sphinxVerbatim}
}

{

\kern-\sphinxverbatimsmallskipamount\kern-\baselineskip
\kern+\FrameHeightAdjust\kern-\fboxrule
\vspace{\nbsphinxcodecellspacing}

\sphinxsetup{VerbatimColor={named}{white}}
\sphinxsetup{VerbatimBorderColor={named}{nbsphinx-code-border}}
\begin{sphinxVerbatim}[commandchars=\\\{\}]
\llap{\color{nbsphinxout}[13]:\,\hspace{\fboxrule}\hspace{\fboxsep}}(-1.0, 1.0)
\end{sphinxVerbatim}
}

\hrule height -\fboxrule\relax
\vspace{\nbsphinxcodecellspacing}

\makeatletter\setbox\nbsphinxpromptbox\box\voidb@x\makeatother

\begin{nbsphinxfancyoutput}

\noindent\sphinxincludegraphics[width=624\sphinxpxdimen,height=279\sphinxpxdimen]{{jupyter_notebook_io_sensitivity_analysis_16_1}.png}

\end{nbsphinxfancyoutput}

{
\sphinxsetup{VerbatimColor={named}{nbsphinx-code-bg}}
\sphinxsetup{VerbatimBorderColor={named}{nbsphinx-code-border}}
\begin{sphinxVerbatim}[commandchars=\\\{\}]
\llap{\color{nbsphinxin}[14]:\,\hspace{\fboxrule}\hspace{\fboxsep}}\PYG{n}{plt}\PYG{o}{.}\PYG{n}{scatter}\PYG{p}{(} \PYG{n}{influent\PYGZus{}ph}\PYG{p}{[}\PYG{l+s+s1}{\PYGZsq{}}\PYG{l+s+s1}{X\PYGZus{}pr\PYGZus{}biom\PYGZus{}in}\PYG{l+s+s1}{\PYGZsq{}}\PYG{p}{]}\PYG{p}{,}\PYG{n}{influent\PYGZus{}ph}\PYG{p}{[}\PYG{l+s+s1}{\PYGZsq{}}\PYG{l+s+s1}{ pH }\PYG{l+s+s1}{\PYGZsq{}}\PYG{p}{]}\PYG{p}{)}
\PYG{n}{plt}\PYG{o}{.}\PYG{n}{xlabel}\PYG{p}{(}\PYG{l+s+s1}{\PYGZsq{}}\PYG{l+s+s1}{X\PYGZus{}pr\PYGZus{}biom\PYGZus{}in}\PYG{l+s+s1}{\PYGZsq{}}\PYG{p}{,}\PYG{n}{fontsize}\PYG{o}{=}\PYG{l+m+mi}{18}\PYG{p}{)}
\PYG{n}{plt}\PYG{o}{.}\PYG{n}{ylabel}\PYG{p}{(}\PYG{l+s+s1}{\PYGZsq{}}\PYG{l+s+s1}{pH}\PYG{l+s+s1}{\PYGZsq{}}\PYG{p}{,}\PYG{n}{fontsize}\PYG{o}{=}\PYG{l+m+mi}{18}\PYG{p}{)}\PYG{p}{;}
\end{sphinxVerbatim}
}

\hrule height -\fboxrule\relax
\vspace{\nbsphinxcodecellspacing}

\makeatletter\setbox\nbsphinxpromptbox\box\voidb@x\makeatother

\begin{nbsphinxfancyoutput}

\noindent\sphinxincludegraphics[width=402\sphinxpxdimen,height=271\sphinxpxdimen]{{jupyter_notebook_io_sensitivity_analysis_17_0}.png}

\end{nbsphinxfancyoutput}

{
\sphinxsetup{VerbatimColor={named}{nbsphinx-code-bg}}
\sphinxsetup{VerbatimBorderColor={named}{nbsphinx-code-border}}
\begin{sphinxVerbatim}[commandchars=\\\{\}]
\llap{\color{nbsphinxin}[15]:\,\hspace{\fboxrule}\hspace{\fboxsep}}\PYG{k+kn}{from} \PYG{n+nn}{pandas}\PYG{n+nn}{.}\PYG{n+nn}{plotting} \PYG{k+kn}{import} \PYG{n}{scatter\PYGZus{}matrix}
\PYG{n}{scatter\PYGZus{}matrix}\PYG{p}{(}\PYG{n}{influent\PYGZus{}ph}\PYG{p}{,} \PYG{n}{alpha}\PYG{o}{=}\PYG{l+m+mf}{0.6}\PYG{p}{,}\PYG{n}{figsize}\PYG{o}{=}\PYG{p}{(}\PYG{l+m+mi}{10}\PYG{p}{,}\PYG{l+m+mi}{10}\PYG{p}{)}\PYG{p}{)}\PYG{p}{;}
\end{sphinxVerbatim}
}

\hrule height -\fboxrule\relax
\vspace{\nbsphinxcodecellspacing}

\makeatletter\setbox\nbsphinxpromptbox\box\voidb@x\makeatother

\begin{nbsphinxfancyoutput}

\noindent\sphinxincludegraphics[width=616\sphinxpxdimen,height=602\sphinxpxdimen]{{jupyter_notebook_io_sensitivity_analysis_18_0}.png}

\end{nbsphinxfancyoutput}


\subsection{2. Params/Output sensitivity}
\label{\detokenize{jupyter_notebook/io_sensitivity_analysis:2.-Params/Output-sensitivity}}
{
\sphinxsetup{VerbatimColor={named}{nbsphinx-code-bg}}
\sphinxsetup{VerbatimBorderColor={named}{nbsphinx-code-border}}
\begin{sphinxVerbatim}[commandchars=\\\{\}]
\llap{\color{nbsphinxin}[16]:\,\hspace{\fboxrule}\hspace{\fboxsep}}\PYG{c+c1}{\PYGZsh{} Set the value of percentage and sample size for lhs}
\PYG{n}{percent} \PYG{o}{=} \PYG{l+m+mf}{0.1} \PYG{c+c1}{\PYGZsh{} NOTE: for params percent should be \PYGZlt{}= 0.05}
\PYG{n}{sample\PYGZus{}size} \PYG{o}{=} \PYG{l+m+mi}{100}
\PYG{n}{variable} \PYG{o}{=} \PYG{l+s+s1}{\PYGZsq{}}\PYG{l+s+s1}{params}\PYG{l+s+s1}{\PYGZsq{}}    \PYG{c+c1}{\PYGZsh{} influent/params/ic}
\PYG{n}{method} \PYG{o}{=} \PYG{l+s+s1}{\PYGZsq{}}\PYG{l+s+s1}{uniform}\PYG{l+s+s1}{\PYGZsq{}}    \PYG{c+c1}{\PYGZsh{}\PYGZsq{}uniform\PYGZsq{} or \PYGZsq{}lhs\PYGZsq{}}
\end{sphinxVerbatim}
}

{
\sphinxsetup{VerbatimColor={named}{nbsphinx-code-bg}}
\sphinxsetup{VerbatimBorderColor={named}{nbsphinx-code-border}}
\begin{sphinxVerbatim}[commandchars=\\\{\}]
\llap{\color{nbsphinxin}[17]:\,\hspace{\fboxrule}\hspace{\fboxsep}}\PYG{n}{index}\PYG{o}{=}\PYG{n}{adm1fu}\PYG{o}{.}\PYG{n}{create\PYGZus{}a\PYGZus{}sample\PYGZus{}matrix}\PYG{p}{(}\PYG{n}{variable}\PYG{p}{,}\PYG{n}{method}\PYG{p}{,}\PYG{n}{percent}\PYG{p}{,}\PYG{n}{sample\PYGZus{}size}\PYG{p}{)}
\PYG{n+nb}{print} \PYG{p}{(}\PYG{p}{)}
\PYG{n+nb}{print} \PYG{p}{(}\PYG{l+s+s1}{\PYGZsq{}}\PYG{l+s+s1}{Number of elements participated in the sampling:}\PYG{l+s+s1}{\PYGZsq{}}\PYG{p}{,}\PYG{n+nb}{len}\PYG{p}{(}\PYG{n}{index}\PYG{p}{)}\PYG{p}{)}
\end{sphinxVerbatim}
}

{

\kern-\sphinxverbatimsmallskipamount\kern-\baselineskip
\kern+\FrameHeightAdjust\kern-\fboxrule
\vspace{\nbsphinxcodecellspacing}

\sphinxsetup{VerbatimColor={named}{white}}
\sphinxsetup{VerbatimBorderColor={named}{nbsphinx-code-border}}
\begin{sphinxVerbatim}[commandchars=\\\{\}]
Saves a sampling matrix [sample\_size,array\_size] into var\_params.csv
sample\_size,array\_size:  (100, 92)
Each column of the matrix corresponds to a variable perturbed 100 times around its original value
var\_params.csv SAVED!

Number of elements participated in the sampling: 92
\end{sphinxVerbatim}
}

{
\sphinxsetup{VerbatimColor={named}{nbsphinx-code-bg}}
\sphinxsetup{VerbatimBorderColor={named}{nbsphinx-code-border}}
\begin{sphinxVerbatim}[commandchars=\\\{\}]
\llap{\color{nbsphinxin}[18]:\,\hspace{\fboxrule}\hspace{\fboxsep}}\PYG{c+c1}{\PYGZsh{}exe\PYGZus{}time=adm1fu.adm1f\PYGZus{}output\PYGZus{}sampling(ADM1F\PYGZus{}EXE,variable,index)}
\end{sphinxVerbatim}
}

{
\sphinxsetup{VerbatimColor={named}{nbsphinx-code-bg}}
\sphinxsetup{VerbatimBorderColor={named}{nbsphinx-code-border}}
\begin{sphinxVerbatim}[commandchars=\\\{\}]
\llap{\color{nbsphinxin}[19]:\,\hspace{\fboxrule}\hspace{\fboxsep}}\PYG{p}{[}\PYG{n}{output\PYGZus{}name}\PYG{p}{,}\PYG{n}{output\PYGZus{}unit}\PYG{p}{]}\PYG{o}{=}\PYG{n}{adm1fu}\PYG{o}{.}\PYG{n}{get\PYGZus{}output\PYGZus{}names}\PYG{p}{(}\PYG{p}{)}
\PYG{n}{alloutputs} \PYG{o}{=} \PYG{n}{pd}\PYG{o}{.}\PYG{n}{read\PYGZus{}csv}\PYG{p}{(}\PYG{l+s+s1}{\PYGZsq{}}\PYG{l+s+s1}{outputs\PYGZus{}params.csv}\PYG{l+s+s1}{\PYGZsq{}}\PYG{p}{,} \PYG{n}{sep}\PYG{o}{=}\PYG{l+s+s1}{\PYGZsq{}}\PYG{l+s+s1}{,}\PYG{l+s+s1}{\PYGZsq{}}\PYG{p}{,} \PYG{n}{header}\PYG{o}{=}\PYG{k+kc}{None}\PYG{p}{)}
\PYG{n}{alloutputs}\PYG{o}{.}\PYG{n}{columns} \PYG{o}{=} \PYG{n}{output\PYGZus{}name}
\end{sphinxVerbatim}
}

{
\sphinxsetup{VerbatimColor={named}{nbsphinx-code-bg}}
\sphinxsetup{VerbatimBorderColor={named}{nbsphinx-code-border}}
\begin{sphinxVerbatim}[commandchars=\\\{\}]
\llap{\color{nbsphinxin}[20]:\,\hspace{\fboxrule}\hspace{\fboxsep}}\PYG{p}{[}\PYG{n}{param\PYGZus{}name}\PYG{p}{,}\PYG{n}{param\PYGZus{}index}\PYG{p}{]}\PYG{o}{=}\PYG{n}{adm1fu}\PYG{o}{.}\PYG{n}{get\PYGZus{}param\PYGZus{}names}\PYG{p}{(}\PYG{p}{)}
\end{sphinxVerbatim}
}

{
\sphinxsetup{VerbatimColor={named}{nbsphinx-code-bg}}
\sphinxsetup{VerbatimBorderColor={named}{nbsphinx-code-border}}
\begin{sphinxVerbatim}[commandchars=\\\{\}]
\llap{\color{nbsphinxin}[21]:\,\hspace{\fboxrule}\hspace{\fboxsep}}\PYG{c+c1}{\PYGZsh{} since we did not use all the columns in the influent.dat (see create\PYGZus{}a\PYGZus{}sample\PYGZus{}matrix)}
\PYG{c+c1}{\PYGZsh{} we use index to select used headers for used values}
\PYG{n}{header}\PYG{o}{=}\PYG{p}{[}\PYG{p}{]}
\PYG{k}{for} \PYG{n}{i} \PYG{o+ow}{in} \PYG{n}{index}\PYG{p}{:}
    \PYG{n}{header}\PYG{o}{.}\PYG{n}{append}\PYG{p}{(}\PYG{n}{param\PYGZus{}name}\PYG{p}{[}\PYG{n}{i}\PYG{p}{]}\PYG{p}{)}

\PYG{n}{param\PYGZus{}inputs} \PYG{o}{=} \PYG{n}{pd}\PYG{o}{.}\PYG{n}{read\PYGZus{}csv}\PYG{p}{(}\PYG{l+s+s1}{\PYGZsq{}}\PYG{l+s+s1}{var\PYGZus{}params.csv}\PYG{l+s+s1}{\PYGZsq{}}\PYG{p}{,} \PYG{n}{sep}\PYG{o}{=}\PYG{l+s+s1}{\PYGZsq{}}\PYG{l+s+s1}{,}\PYG{l+s+s1}{\PYGZsq{}}\PYG{p}{,} \PYG{n}{header}\PYG{o}{=}\PYG{k+kc}{None}\PYG{p}{)}
\PYG{n}{param\PYGZus{}inputs}\PYG{o}{.}\PYG{n}{columns} \PYG{o}{=} \PYG{n}{header}
\PYG{n}{param\PYGZus{}inputs}\PYG{o}{.}\PYG{n}{head}\PYG{p}{(}\PYG{p}{)}
\end{sphinxVerbatim}
}

{

\kern-\sphinxverbatimsmallskipamount\kern-\baselineskip
\kern+\FrameHeightAdjust\kern-\fboxrule
\vspace{\nbsphinxcodecellspacing}

\sphinxsetup{VerbatimColor={named}{white}}
\sphinxsetup{VerbatimBorderColor={named}{nbsphinx-code-border}}
\begin{sphinxVerbatim}[commandchars=\\\{\}]
\llap{\color{nbsphinxout}[21]:\,\hspace{\fboxrule}\hspace{\fboxsep}}   f\_sI\_xc  f\_xI\_xc  f\_ch\_xc  f\_pr\_xc  f\_li\_xc     N\_xc      N\_I     N\_aa  \textbackslash{}
0  0.09116  0.27346  0.21535  0.21711  0.27475  0.00251  0.00420  0.00736
1  0.09692  0.25605  0.18183  0.21486  0.27367  0.00293  0.00450  0.00648
2  0.09653  0.25089  0.18351  0.19403  0.22666  0.00246  0.00420  0.00649
3  0.09891  0.25074  0.19437  0.20372  0.23318  0.00263  0.00468  0.00666
4  0.10742  0.26412  0.20268  0.20954  0.26893  0.00263  0.00414  0.00723

      C\_xc     C\_sI  {\ldots}      k\_A\_Bpro       k\_A\_Bac      k\_A\_Bco2  \textbackslash{}
0  0.02895  0.02792  {\ldots}  1.026524e+10  1.086406e+10  9.205019e+09
1  0.02930  0.02715  {\ldots}  9.961014e+09  9.209860e+09  9.484090e+09
2  0.02824  0.03114  {\ldots}  1.043165e+10  1.081807e+10  9.359366e+09
3  0.02873  0.02895  {\ldots}  9.431350e+09  1.031777e+10  9.787729e+09
4  0.02958  0.03157  {\ldots}  1.090281e+10  1.020322e+10  1.063838e+10

        k\_A\_BIN        kLa  K\_H\_h2o\_base  K\_H\_co2\_base  K\_H\_ch4\_base  \textbackslash{}
0  1.087446e+10  207.51543       0.02859       0.03361       0.00146
1  1.097333e+10  185.69982       0.03129       0.03583       0.00146
2  9.475087e+09  218.85580       0.02930       0.03748       0.00140
3  1.030247e+10  184.26372       0.03229       0.03850       0.00127
4  1.076841e+10  189.12319       0.02950       0.03578       0.00138

   K\_H\_h2\_base          k\_P
0      0.00071  50821.70460
1      0.00079  45097.70847
2      0.00074  53707.49901
3      0.00086  49069.07961
4      0.00083  54000.23123

[5 rows x 92 columns]
\end{sphinxVerbatim}
}

{
\sphinxsetup{VerbatimColor={named}{nbsphinx-code-bg}}
\sphinxsetup{VerbatimBorderColor={named}{nbsphinx-code-border}}
\begin{sphinxVerbatim}[commandchars=\\\{\}]
\llap{\color{nbsphinxin}[22]:\,\hspace{\fboxrule}\hspace{\fboxsep}}\PYG{c+c1}{\PYGZsh{} merge influent and output datasets}
\PYG{n}{inout}\PYG{o}{=}\PYG{n}{pd}\PYG{o}{.}\PYG{n}{concat}\PYG{p}{(}\PYG{p}{[}\PYG{n}{param\PYGZus{}inputs}\PYG{p}{,}\PYG{n}{alloutputs}\PYG{p}{]}\PYG{p}{,} \PYG{n}{axis}\PYG{o}{=}\PYG{l+m+mi}{1}\PYG{p}{)}
\PYG{n}{inout}\PYG{o}{.}\PYG{n}{head}\PYG{p}{(}\PYG{p}{)}
\end{sphinxVerbatim}
}

{

\kern-\sphinxverbatimsmallskipamount\kern-\baselineskip
\kern+\FrameHeightAdjust\kern-\fboxrule
\vspace{\nbsphinxcodecellspacing}

\sphinxsetup{VerbatimColor={named}{white}}
\sphinxsetup{VerbatimBorderColor={named}{nbsphinx-code-border}}
\begin{sphinxVerbatim}[commandchars=\\\{\}]
\llap{\color{nbsphinxout}[22]:\,\hspace{\fboxrule}\hspace{\fboxsep}}   f\_sI\_xc  f\_xI\_xc  f\_ch\_xc  f\_pr\_xc  f\_li\_xc     N\_xc      N\_I     N\_aa  \textbackslash{}
0  0.09116  0.27346  0.21535  0.21711  0.27475  0.00251  0.00420  0.00736
1  0.09692  0.25605  0.18183  0.21486  0.27367  0.00293  0.00450  0.00648
2  0.09653  0.25089  0.18351  0.19403  0.22666  0.00246  0.00420  0.00649
3  0.09891  0.25074  0.19437  0.20372  0.23318  0.00263  0.00468  0.00666
4  0.10742  0.26412  0.20268  0.20954  0.26893  0.00263  0.00414  0.00723

      C\_xc     C\_sI  {\ldots}     Alk       NH3       NH4     LCFA    percentch4   \textbackslash{}
0  0.02895  0.02792  {\ldots}  8909.63   9.53919  1060.990  61.5561       56.0945
1  0.02930  0.02715  {\ldots}  7707.69  10.88420   927.238  56.1139       62.1148
2  0.02824  0.03114  {\ldots}  7873.10   8.67396   927.390  68.2248       58.8253
3  0.02873  0.02895  {\ldots}  7961.45  10.83540   953.650  64.3854       57.4965
4  0.02958  0.03157  {\ldots}  8755.37  10.43910  1081.950  67.7978       56.9557

    energych4   efficiency   VFA/ALK        ACN  sampleT
0      65.7436     52.8869  0.222452    87.1484  25.3731
1      62.3272     55.0303  0.147138   654.5290  25.3731
2      64.9817     54.1026  0.191081   196.3670  25.3731
3      66.4942     55.2003  0.118815 -1663.6400  25.3731
4      64.9350     53.1639  0.222974    94.6821  25.3731

[5 rows x 159 columns]
\end{sphinxVerbatim}
}

{
\sphinxsetup{VerbatimColor={named}{nbsphinx-code-bg}}
\sphinxsetup{VerbatimBorderColor={named}{nbsphinx-code-border}}
\begin{sphinxVerbatim}[commandchars=\\\{\}]
\llap{\color{nbsphinxin}[23]:\,\hspace{\fboxrule}\hspace{\fboxsep}}\PYG{n}{corr}\PYG{o}{=}\PYG{n}{inout}\PYG{o}{.}\PYG{n}{corr}\PYG{p}{(}\PYG{p}{)}
\PYG{n}{plt}\PYG{o}{.}\PYG{n}{figure}\PYG{p}{(}\PYG{n}{figsize}\PYG{o}{=}\PYG{p}{(}\PYG{l+m+mi}{25}\PYG{p}{,}\PYG{l+m+mi}{15}\PYG{p}{)}\PYG{p}{)}
\PYG{n}{sns}\PYG{o}{.}\PYG{n}{heatmap}\PYG{p}{(}\PYG{n}{corr}\PYG{o}{.}\PYG{n}{iloc}\PYG{p}{[}\PYG{l+m+mi}{0}\PYG{p}{:}\PYG{l+m+mi}{92}\PYG{p}{,}\PYG{l+m+mi}{92}\PYG{p}{:}\PYG{o}{\PYGZhy{}}\PYG{l+m+mi}{1}\PYG{p}{]}\PYG{p}{,} \PYG{n}{cmap}\PYG{o}{=}\PYG{n}{sns}\PYG{o}{.}\PYG{n}{diverging\PYGZus{}palette}\PYG{p}{(}\PYG{l+m+mi}{220}\PYG{p}{,} \PYG{l+m+mi}{10}\PYG{p}{,} \PYG{n}{as\PYGZus{}cmap}\PYG{o}{=}\PYG{k+kc}{True}\PYG{p}{)}\PYG{p}{)}
\PYG{n}{plt}\PYG{o}{.}\PYG{n}{title}\PYG{p}{(}\PYG{l+s+s1}{\PYGZsq{}}\PYG{l+s+s1}{Correlation Matrix [params/results]}\PYG{l+s+s1}{\PYGZsq{}}\PYG{p}{,}\PYG{n}{fontsize}\PYG{o}{=}\PYG{l+m+mi}{16}\PYG{p}{)}\PYG{p}{;}
\PYG{n}{plt}\PYG{o}{.}\PYG{n}{ylabel}\PYG{p}{(}\PYG{l+s+s1}{\PYGZsq{}}\PYG{l+s+s1}{Parameters}\PYG{l+s+s1}{\PYGZsq{}}\PYG{p}{,}\PYG{n}{fontsize}\PYG{o}{=}\PYG{l+m+mi}{16}\PYG{p}{)}
\PYG{n}{plt}\PYG{o}{.}\PYG{n}{xlabel}\PYG{p}{(}\PYG{l+s+s1}{\PYGZsq{}}\PYG{l+s+s1}{Outputs}\PYG{l+s+s1}{\PYGZsq{}}\PYG{p}{,}\PYG{n}{fontsize}\PYG{o}{=}\PYG{l+m+mi}{16}\PYG{p}{)}
\end{sphinxVerbatim}
}

{

\kern-\sphinxverbatimsmallskipamount\kern-\baselineskip
\kern+\FrameHeightAdjust\kern-\fboxrule
\vspace{\nbsphinxcodecellspacing}

\sphinxsetup{VerbatimColor={named}{white}}
\sphinxsetup{VerbatimBorderColor={named}{nbsphinx-code-border}}
\begin{sphinxVerbatim}[commandchars=\\\{\}]
\llap{\color{nbsphinxout}[23]:\,\hspace{\fboxrule}\hspace{\fboxsep}}Text(0.5, 113.09375, 'Outputs')
\end{sphinxVerbatim}
}

\hrule height -\fboxrule\relax
\vspace{\nbsphinxcodecellspacing}

\makeatletter\setbox\nbsphinxpromptbox\box\voidb@x\makeatother

\begin{nbsphinxfancyoutput}

\noindent\sphinxincludegraphics[width=1379\sphinxpxdimen,height=936\sphinxpxdimen]{{jupyter_notebook_io_sensitivity_analysis_27_1}.png}

\end{nbsphinxfancyoutput}

{
\sphinxsetup{VerbatimColor={named}{nbsphinx-code-bg}}
\sphinxsetup{VerbatimBorderColor={named}{nbsphinx-code-border}}
\begin{sphinxVerbatim}[commandchars=\\\{\}]
\llap{\color{nbsphinxin}[ ]:\,\hspace{\fboxrule}\hspace{\fboxsep}}
\end{sphinxVerbatim}
}


\section{ADM1F SRT: Synthetic Model\sphinxhyphen{}Data Calibration}
\label{\detokenize{jupyter_notebook/parameter-calibration:ADM1F-SRT:-Synthetic-Model-Data-Calibration}}\label{\detokenize{jupyter_notebook/parameter-calibration::doc}}
\sphinxAtStartPar
Authors: Wenjuan Zhang and Elchin Jafarov

\sphinxAtStartPar
Before starting model clibration some additional softwares needs to be installed
\begin{enumerate}
\sphinxsetlistlabels{\arabic}{enumi}{enumii}{}{.}%
\item {} 
\sphinxAtStartPar
Install Julia

\item {} 
\sphinxAtStartPar
Install Mads

\end{enumerate}
\begin{itemize}
\item {} 
\sphinxAtStartPar
“import Pkg; Pkg.add(”Mads“)” \sphinxhyphen{}\textgreater{} this command should be enough

\item {} 
\sphinxAtStartPar
Mads github page is \sphinxurl{https://github.com/madsjulia/Mads.jl}

\item {} 
\sphinxAtStartPar
Mads documentation documentation page is \sphinxurl{http://madsjulia.github.io/Mads.jl/}

\end{itemize}
\begin{enumerate}
\sphinxsetlistlabels{\arabic}{enumi}{enumii}{}{.}%
\setcounter{enumi}{2}
\item {} 
\sphinxAtStartPar
To use jupyter notebook to display Julia, add JuliaHub, check this page \sphinxurl{https://juliahub.com/ui/Packages/IJulia/nfu7T/1.23.1}

\end{enumerate}


\subsection{1. Sensitivity of the model output to parameter changes}
\label{\detokenize{jupyter_notebook/parameter-calibration:1.-Sensitivity-of-the-model-output-to-parameter-changes}}
\sphinxAtStartPar
Before starting model clibration we need to make sure that results are senstive to the parameter changes.

{
\sphinxsetup{VerbatimColor={named}{nbsphinx-code-bg}}
\sphinxsetup{VerbatimBorderColor={named}{nbsphinx-code-border}}
\begin{sphinxVerbatim}[commandchars=\\\{\}]
\llap{\color{nbsphinxin}[2]:\,\hspace{\fboxrule}\hspace{\fboxsep}}\PYG{k}{import} \PYG{n}{Mads}
\PYG{k}{import} \PYG{n}{DelimitedFiles}
\PYG{k}{import} \PYG{n}{OrderedCollections}
\end{sphinxVerbatim}
}

{
\sphinxsetup{VerbatimColor={named}{nbsphinx-code-bg}}
\sphinxsetup{VerbatimBorderColor={named}{nbsphinx-code-border}}
\begin{sphinxVerbatim}[commandchars=\\\{\}]
\llap{\color{nbsphinxin}[3]:\,\hspace{\fboxrule}\hspace{\fboxsep}}\PYG{c}{\PYGZsh{} check if you are in the right folder}
\PYG{n}{pwd}\PYG{p}{(}\PYG{p}{)}
\end{sphinxVerbatim}
}

{

\kern-\sphinxverbatimsmallskipamount\kern-\baselineskip
\kern+\FrameHeightAdjust\kern-\fboxrule
\vspace{\nbsphinxcodecellspacing}

\sphinxsetup{VerbatimColor={named}{white}}
\sphinxsetup{VerbatimBorderColor={named}{nbsphinx-code-border}}
\begin{sphinxVerbatim}[commandchars=\\\{\}]
\llap{\color{nbsphinxout}[3]:\,\hspace{\fboxrule}\hspace{\fboxsep}}"/Users/elchin/project/ADM1F\_WM/calibration"
\end{sphinxVerbatim}
}

{
\sphinxsetup{VerbatimColor={named}{nbsphinx-code-bg}}
\sphinxsetup{VerbatimBorderColor={named}{nbsphinx-code-border}}
\begin{sphinxVerbatim}[commandchars=\\\{\}]
\llap{\color{nbsphinxin}[4]:\,\hspace{\fboxrule}\hspace{\fboxsep}}\PYG{c}{\PYGZsh{} navigate to the calibration folder}
\PYG{n}{cd}\PYG{p}{(}\PYG{l+s}{\PYGZdq{}}\PYG{l+s}{/}\PYG{l+s}{U}\PYG{l+s}{s}\PYG{l+s}{e}\PYG{l+s}{r}\PYG{l+s}{s}\PYG{l+s}{/}\PYG{l+s}{e}\PYG{l+s}{l}\PYG{l+s}{c}\PYG{l+s}{h}\PYG{l+s}{i}\PYG{l+s}{n}\PYG{l+s}{/}\PYG{l+s}{p}\PYG{l+s}{r}\PYG{l+s}{o}\PYG{l+s}{j}\PYG{l+s}{e}\PYG{l+s}{c}\PYG{l+s}{t}\PYG{l+s}{/}\PYG{l+s}{A}\PYG{l+s}{D}\PYG{l+s}{M}\PYG{l+s}{1}\PYG{l+s}{F}\PYG{l+s}{\PYGZus{}}\PYG{l+s}{W}\PYG{l+s}{M}\PYG{l+s}{/}\PYG{l+s}{c}\PYG{l+s}{a}\PYG{l+s}{l}\PYG{l+s}{i}\PYG{l+s}{b}\PYG{l+s}{r}\PYG{l+s}{a}\PYG{l+s}{t}\PYG{l+s}{i}\PYG{l+s}{o}\PYG{l+s}{n}\PYG{l+s}{\PYGZdq{}}\PYG{p}{)}
\end{sphinxVerbatim}
}

{
\sphinxsetup{VerbatimColor={named}{nbsphinx-code-bg}}
\sphinxsetup{VerbatimBorderColor={named}{nbsphinx-code-border}}
\begin{sphinxVerbatim}[commandchars=\\\{\}]
\llap{\color{nbsphinxin}[5]:\,\hspace{\fboxrule}\hspace{\fboxsep}}\PYG{c}{\PYGZsh{} passing \PYGZdq{}tim.mads\PYGZdq{} to the Mads. \PYGZdq{}tim.mads\PYGZdq{} includes the \PYGZdq{}tim.jl\PYGZdq{} that executes the}
\PYG{c}{\PYGZsh{} ADM1F model and saves seven outputs of interest.}
\PYG{n}{filename} \PYG{o}{=} \PYG{l+s}{\PYGZdq{}}\PYG{l+s}{t}\PYG{l+s}{i}\PYG{l+s}{m}\PYG{l+s}{.}\PYG{l+s}{m}\PYG{l+s}{a}\PYG{l+s}{d}\PYG{l+s}{s}\PYG{l+s}{\PYGZdq{}}
\PYG{n}{md} \PYG{o}{=} \PYG{n}{Mads}\PYG{o}{.}\PYG{n}{loadmadsfile}\PYG{p}{(}\PYG{n}{filename}\PYG{p}{)}
\end{sphinxVerbatim}
}

{

\kern-\sphinxverbatimsmallskipamount\kern-\baselineskip
\kern+\FrameHeightAdjust\kern-\fboxrule
\vspace{\nbsphinxcodecellspacing}

\sphinxsetup{VerbatimColor={named}{white}}
\sphinxsetup{VerbatimBorderColor={named}{nbsphinx-code-border}}
\begin{sphinxVerbatim}[commandchars=\\\{\}]
\llap{\color{nbsphinxout}[5]:\,\hspace{\fboxrule}\hspace{\fboxsep}}Dict\{String,Any\} with 4 entries:
  "Parameters"    => OrderedCollections.OrderedDict\{String,OrderedCollections.O…
  "Observations"  => OrderedCollections.OrderedDict\{String,OrderedCollections.O…
  "Julia command" => "./tim.jl"
  "Filename"      => "tim.mads"
\end{sphinxVerbatim}
}

{
\sphinxsetup{VerbatimColor={named}{nbsphinx-code-bg}}
\sphinxsetup{VerbatimBorderColor={named}{nbsphinx-code-border}}
\begin{sphinxVerbatim}[commandchars=\\\{\}]
\llap{\color{nbsphinxin}[6]:\,\hspace{\fboxrule}\hspace{\fboxsep}}\PYG{c}{\PYGZsh{} run the model with the default pamaters from \PYGZdq{}tim.mads\PYGZdq{}}
\PYG{c}{\PYGZsh{} we denote the synthetic truth/observations}
\PYG{n}{output\PYGZus{}truth}\PYG{o}{=}\PYG{n}{Mads}\PYG{o}{.}\PYG{n}{forward}\PYG{p}{(}\PYG{n}{md}\PYG{p}{)}
\end{sphinxVerbatim}
}

{

\kern-\sphinxverbatimsmallskipamount\kern-\baselineskip
\kern+\FrameHeightAdjust\kern-\fboxrule
\vspace{\nbsphinxcodecellspacing}

\sphinxsetup{VerbatimColor={named}{white}}
\sphinxsetup{VerbatimBorderColor={named}{nbsphinx-code-border}}
\begin{sphinxVerbatim}[commandchars=\\\{\}]
\llap{\color{nbsphinxout}[6]:\,\hspace{\fboxrule}\hspace{\fboxsep}}OrderedCollections.OrderedDict\{Any,Float64\} with 7 entries:
  "o2" => 145.362
  "o3" => 48.578
  "o1" => 363.333
  "o4" => 35.1163
  "o5" => 11.6628
  "o6" => 6.51573
  "o7" => 3.04116e-312
\end{sphinxVerbatim}
}

{
\sphinxsetup{VerbatimColor={named}{nbsphinx-code-bg}}
\sphinxsetup{VerbatimBorderColor={named}{nbsphinx-code-border}}
\begin{sphinxVerbatim}[commandchars=\\\{\}]
\llap{\color{nbsphinxin}[7]:\,\hspace{\fboxrule}\hspace{\fboxsep}}\PYG{c}{\PYGZsh{} the variables in the dictinary that have \PYGZdq{}type\PYGZdq{} equal to \PYGZdq{}opt\PYGZdq{} will participate in the}
\PYG{c}{\PYGZsh{} model\PYGZhy{}data calibration}
\PYG{n}{md}\PYG{p}{[}\PYG{l+s}{\PYGZdq{}}\PYG{l+s}{P}\PYG{l+s}{a}\PYG{l+s}{r}\PYG{l+s}{a}\PYG{l+s}{m}\PYG{l+s}{e}\PYG{l+s}{t}\PYG{l+s}{e}\PYG{l+s}{r}\PYG{l+s}{s}\PYG{l+s}{\PYGZdq{}}\PYG{p}{]}\PYG{p}{[}\PYG{l+s}{\PYGZdq{}}\PYG{l+s}{p}\PYG{l+s}{4}\PYG{l+s}{8}\PYG{l+s}{\PYGZdq{}}\PYG{p}{]}\PYG{c}{\PYGZsh{}[\PYGZdq{}init\PYGZdq{}]}
\end{sphinxVerbatim}
}

{

\kern-\sphinxverbatimsmallskipamount\kern-\baselineskip
\kern+\FrameHeightAdjust\kern-\fboxrule
\vspace{\nbsphinxcodecellspacing}

\sphinxsetup{VerbatimColor={named}{white}}
\sphinxsetup{VerbatimBorderColor={named}{nbsphinx-code-border}}
\begin{sphinxVerbatim}[commandchars=\\\{\}]
\llap{\color{nbsphinxout}[7]:\,\hspace{\fboxrule}\hspace{\fboxsep}}OrderedCollections.OrderedDict\{String,Any\} with 4 entries:
  "init" => 0.5
  "max"  => 0.6
  "min"  => 0.4
  "type" => "opt"
\end{sphinxVerbatim}
}

{
\sphinxsetup{VerbatimColor={named}{nbsphinx-code-bg}}
\sphinxsetup{VerbatimBorderColor={named}{nbsphinx-code-border}}
\begin{sphinxVerbatim}[commandchars=\\\{\}]
\llap{\color{nbsphinxin}[8]:\,\hspace{\fboxrule}\hspace{\fboxsep}}\PYG{n}{Mads}\PYG{o}{.}\PYG{n}{showobservations}\PYG{p}{(}\PYG{n}{md}\PYG{p}{)}
\end{sphinxVerbatim}
}

{

\kern-\sphinxverbatimsmallskipamount\kern-\baselineskip
\kern+\FrameHeightAdjust\kern-\fboxrule
\vspace{\nbsphinxcodecellspacing}

\sphinxsetup{VerbatimColor={named}{white}}
\sphinxsetup{VerbatimBorderColor={named}{nbsphinx-code-border}}
\begin{sphinxVerbatim}[commandchars=\\\{\}]
o1         target =          43.524 weight =               1
o2         target =         19.1588 weight =               1
o3         target =         6.23771 weight =               1
o4         target =         0.00779 weight =               1
o5         target =               0 weight =               1
o6         target =         6.87726 weight =               1
o7         target =         0.63904 weight =               1
Number of observations is 7
\end{sphinxVerbatim}
}

{
\sphinxsetup{VerbatimColor={named}{nbsphinx-code-bg}}
\sphinxsetup{VerbatimBorderColor={named}{nbsphinx-code-border}}
\begin{sphinxVerbatim}[commandchars=\\\{\}]
\llap{\color{nbsphinxin}[10]:\,\hspace{\fboxrule}\hspace{\fboxsep}}\PYG{c}{\PYGZsh{} set observations equal to model runs with md[\PYGZdq{}Parameters\PYGZdq{}][\PYGZdq{}p48\PYGZdq{}][init]=0.5}
\PYG{n}{md}\PYG{p}{[}\PYG{l+s}{\PYGZdq{}}\PYG{l+s}{O}\PYG{l+s}{b}\PYG{l+s}{s}\PYG{l+s}{e}\PYG{l+s}{r}\PYG{l+s}{v}\PYG{l+s}{a}\PYG{l+s}{t}\PYG{l+s}{i}\PYG{l+s}{o}\PYG{l+s}{n}\PYG{l+s}{s}\PYG{l+s}{\PYGZdq{}}\PYG{p}{]}\PYG{p}{[}\PYG{l+s}{\PYGZdq{}}\PYG{l+s}{o}\PYG{l+s}{1}\PYG{l+s}{\PYGZdq{}}\PYG{p}{]}\PYG{p}{[}\PYG{l+s}{\PYGZdq{}}\PYG{l+s}{t}\PYG{l+s}{a}\PYG{l+s}{r}\PYG{l+s}{g}\PYG{l+s}{e}\PYG{l+s}{t}\PYG{l+s}{\PYGZdq{}}\PYG{p}{]}\PYG{o}{=}\PYG{n}{output\PYGZus{}truth}\PYG{p}{[}\PYG{l+s}{\PYGZdq{}}\PYG{l+s}{o}\PYG{l+s}{1}\PYG{l+s}{\PYGZdq{}}\PYG{p}{]}
\PYG{n}{md}\PYG{p}{[}\PYG{l+s}{\PYGZdq{}}\PYG{l+s}{O}\PYG{l+s}{b}\PYG{l+s}{s}\PYG{l+s}{e}\PYG{l+s}{r}\PYG{l+s}{v}\PYG{l+s}{a}\PYG{l+s}{t}\PYG{l+s}{i}\PYG{l+s}{o}\PYG{l+s}{n}\PYG{l+s}{s}\PYG{l+s}{\PYGZdq{}}\PYG{p}{]}\PYG{p}{[}\PYG{l+s}{\PYGZdq{}}\PYG{l+s}{o}\PYG{l+s}{2}\PYG{l+s}{\PYGZdq{}}\PYG{p}{]}\PYG{p}{[}\PYG{l+s}{\PYGZdq{}}\PYG{l+s}{t}\PYG{l+s}{a}\PYG{l+s}{r}\PYG{l+s}{g}\PYG{l+s}{e}\PYG{l+s}{t}\PYG{l+s}{\PYGZdq{}}\PYG{p}{]}\PYG{o}{=}\PYG{n}{output\PYGZus{}truth}\PYG{p}{[}\PYG{l+s}{\PYGZdq{}}\PYG{l+s}{o}\PYG{l+s}{2}\PYG{l+s}{\PYGZdq{}}\PYG{p}{]}
\PYG{n}{md}\PYG{p}{[}\PYG{l+s}{\PYGZdq{}}\PYG{l+s}{O}\PYG{l+s}{b}\PYG{l+s}{s}\PYG{l+s}{e}\PYG{l+s}{r}\PYG{l+s}{v}\PYG{l+s}{a}\PYG{l+s}{t}\PYG{l+s}{i}\PYG{l+s}{o}\PYG{l+s}{n}\PYG{l+s}{s}\PYG{l+s}{\PYGZdq{}}\PYG{p}{]}\PYG{p}{[}\PYG{l+s}{\PYGZdq{}}\PYG{l+s}{o}\PYG{l+s}{3}\PYG{l+s}{\PYGZdq{}}\PYG{p}{]}\PYG{p}{[}\PYG{l+s}{\PYGZdq{}}\PYG{l+s}{t}\PYG{l+s}{a}\PYG{l+s}{r}\PYG{l+s}{g}\PYG{l+s}{e}\PYG{l+s}{t}\PYG{l+s}{\PYGZdq{}}\PYG{p}{]}\PYG{o}{=}\PYG{n}{output\PYGZus{}truth}\PYG{p}{[}\PYG{l+s}{\PYGZdq{}}\PYG{l+s}{o}\PYG{l+s}{3}\PYG{l+s}{\PYGZdq{}}\PYG{p}{]}
\PYG{n}{md}\PYG{p}{[}\PYG{l+s}{\PYGZdq{}}\PYG{l+s}{O}\PYG{l+s}{b}\PYG{l+s}{s}\PYG{l+s}{e}\PYG{l+s}{r}\PYG{l+s}{v}\PYG{l+s}{a}\PYG{l+s}{t}\PYG{l+s}{i}\PYG{l+s}{o}\PYG{l+s}{n}\PYG{l+s}{s}\PYG{l+s}{\PYGZdq{}}\PYG{p}{]}\PYG{p}{[}\PYG{l+s}{\PYGZdq{}}\PYG{l+s}{o}\PYG{l+s}{4}\PYG{l+s}{\PYGZdq{}}\PYG{p}{]}\PYG{p}{[}\PYG{l+s}{\PYGZdq{}}\PYG{l+s}{t}\PYG{l+s}{a}\PYG{l+s}{r}\PYG{l+s}{g}\PYG{l+s}{e}\PYG{l+s}{t}\PYG{l+s}{\PYGZdq{}}\PYG{p}{]}\PYG{o}{=}\PYG{n}{output\PYGZus{}truth}\PYG{p}{[}\PYG{l+s}{\PYGZdq{}}\PYG{l+s}{o}\PYG{l+s}{4}\PYG{l+s}{\PYGZdq{}}\PYG{p}{]}
\PYG{n}{md}\PYG{p}{[}\PYG{l+s}{\PYGZdq{}}\PYG{l+s}{O}\PYG{l+s}{b}\PYG{l+s}{s}\PYG{l+s}{e}\PYG{l+s}{r}\PYG{l+s}{v}\PYG{l+s}{a}\PYG{l+s}{t}\PYG{l+s}{i}\PYG{l+s}{o}\PYG{l+s}{n}\PYG{l+s}{s}\PYG{l+s}{\PYGZdq{}}\PYG{p}{]}\PYG{p}{[}\PYG{l+s}{\PYGZdq{}}\PYG{l+s}{o}\PYG{l+s}{5}\PYG{l+s}{\PYGZdq{}}\PYG{p}{]}\PYG{p}{[}\PYG{l+s}{\PYGZdq{}}\PYG{l+s}{t}\PYG{l+s}{a}\PYG{l+s}{r}\PYG{l+s}{g}\PYG{l+s}{e}\PYG{l+s}{t}\PYG{l+s}{\PYGZdq{}}\PYG{p}{]}\PYG{o}{=}\PYG{n}{output\PYGZus{}truth}\PYG{p}{[}\PYG{l+s}{\PYGZdq{}}\PYG{l+s}{o}\PYG{l+s}{5}\PYG{l+s}{\PYGZdq{}}\PYG{p}{]}
\PYG{n}{md}\PYG{p}{[}\PYG{l+s}{\PYGZdq{}}\PYG{l+s}{O}\PYG{l+s}{b}\PYG{l+s}{s}\PYG{l+s}{e}\PYG{l+s}{r}\PYG{l+s}{v}\PYG{l+s}{a}\PYG{l+s}{t}\PYG{l+s}{i}\PYG{l+s}{o}\PYG{l+s}{n}\PYG{l+s}{s}\PYG{l+s}{\PYGZdq{}}\PYG{p}{]}\PYG{p}{[}\PYG{l+s}{\PYGZdq{}}\PYG{l+s}{o}\PYG{l+s}{6}\PYG{l+s}{\PYGZdq{}}\PYG{p}{]}\PYG{p}{[}\PYG{l+s}{\PYGZdq{}}\PYG{l+s}{t}\PYG{l+s}{a}\PYG{l+s}{r}\PYG{l+s}{g}\PYG{l+s}{e}\PYG{l+s}{t}\PYG{l+s}{\PYGZdq{}}\PYG{p}{]}\PYG{o}{=}\PYG{n}{output\PYGZus{}truth}\PYG{p}{[}\PYG{l+s}{\PYGZdq{}}\PYG{l+s}{o}\PYG{l+s}{6}\PYG{l+s}{\PYGZdq{}}\PYG{p}{]}
\PYG{n}{md}\PYG{p}{[}\PYG{l+s}{\PYGZdq{}}\PYG{l+s}{O}\PYG{l+s}{b}\PYG{l+s}{s}\PYG{l+s}{e}\PYG{l+s}{r}\PYG{l+s}{v}\PYG{l+s}{a}\PYG{l+s}{t}\PYG{l+s}{i}\PYG{l+s}{o}\PYG{l+s}{n}\PYG{l+s}{s}\PYG{l+s}{\PYGZdq{}}\PYG{p}{]}\PYG{p}{[}\PYG{l+s}{\PYGZdq{}}\PYG{l+s}{o}\PYG{l+s}{7}\PYG{l+s}{\PYGZdq{}}\PYG{p}{]}\PYG{p}{[}\PYG{l+s}{\PYGZdq{}}\PYG{l+s}{t}\PYG{l+s}{a}\PYG{l+s}{r}\PYG{l+s}{g}\PYG{l+s}{e}\PYG{l+s}{t}\PYG{l+s}{\PYGZdq{}}\PYG{p}{]}\PYG{o}{=}\PYG{n}{output\PYGZus{}truth}\PYG{p}{[}\PYG{l+s}{\PYGZdq{}}\PYG{l+s}{o}\PYG{l+s}{7}\PYG{l+s}{\PYGZdq{}}\PYG{p}{]}
\end{sphinxVerbatim}
}

{

\kern-\sphinxverbatimsmallskipamount\kern-\baselineskip
\kern+\FrameHeightAdjust\kern-\fboxrule
\vspace{\nbsphinxcodecellspacing}

\sphinxsetup{VerbatimColor={named}{white}}
\sphinxsetup{VerbatimBorderColor={named}{nbsphinx-code-border}}
\begin{sphinxVerbatim}[commandchars=\\\{\}]
\llap{\color{nbsphinxout}[10]:\,\hspace{\fboxrule}\hspace{\fboxsep}}3.04116075262e-312
\end{sphinxVerbatim}
}

{
\sphinxsetup{VerbatimColor={named}{nbsphinx-code-bg}}
\sphinxsetup{VerbatimBorderColor={named}{nbsphinx-code-border}}
\begin{sphinxVerbatim}[commandchars=\\\{\}]
\llap{\color{nbsphinxin}[11]:\,\hspace{\fboxrule}\hspace{\fboxsep}}\PYG{c}{\PYGZsh{} new synthetic observations}
\PYG{n}{Mads}\PYG{o}{.}\PYG{n}{showobservations}\PYG{p}{(}\PYG{n}{md}\PYG{p}{)}
\end{sphinxVerbatim}
}

{

\kern-\sphinxverbatimsmallskipamount\kern-\baselineskip
\kern+\FrameHeightAdjust\kern-\fboxrule
\vspace{\nbsphinxcodecellspacing}

\sphinxsetup{VerbatimColor={named}{white}}
\sphinxsetup{VerbatimBorderColor={named}{nbsphinx-code-border}}
\begin{sphinxVerbatim}[commandchars=\\\{\}]
o1         target =         363.333 weight =               1
o2         target =         145.362 weight =               1
o3         target =          48.578 weight =               1
o4         target =         35.1163 weight =               1
o5         target =         11.6628 weight =               1
o6         target =         6.51573 weight =               1
o7         target =    3.04116e-312 weight =               1
Number of observations is 7
\end{sphinxVerbatim}
}

{
\sphinxsetup{VerbatimColor={named}{nbsphinx-code-bg}}
\sphinxsetup{VerbatimBorderColor={named}{nbsphinx-code-border}}
\begin{sphinxVerbatim}[commandchars=\\\{\}]
\llap{\color{nbsphinxin}[27]:\,\hspace{\fboxrule}\hspace{\fboxsep}}\PYG{c}{\PYGZsh{}find out how many parameters are opt\PYGZhy{}in}
\PYG{k}{for} \PYG{n}{i} \PYG{k+kp}{in} \PYG{l+m+mi}{1}\PYG{o}{:}\PYG{l+m+mi}{100}
    \PYG{n}{s1}\PYG{o}{=}\PYG{n}{string}\PYG{o}{.}\PYG{p}{(}\PYG{n}{i}\PYG{p}{)}
    \PYG{n}{s2}\PYG{o}{=}\PYG{l+s}{\PYGZdq{}}\PYG{l+s}{p}\PYG{l+s}{\PYGZdq{}}
    \PYG{k}{if} \PYG{n}{md}\PYG{p}{[}\PYG{l+s}{\PYGZdq{}}\PYG{l+s}{P}\PYG{l+s}{a}\PYG{l+s}{r}\PYG{l+s}{a}\PYG{l+s}{m}\PYG{l+s}{e}\PYG{l+s}{t}\PYG{l+s}{e}\PYG{l+s}{r}\PYG{l+s}{s}\PYG{l+s}{\PYGZdq{}}\PYG{p}{]}\PYG{p}{[}\PYG{n}{s2}\PYG{o}{*}\PYG{n}{s1}\PYG{p}{]}\PYG{p}{[}\PYG{l+s}{\PYGZdq{}}\PYG{l+s}{t}\PYG{l+s}{y}\PYG{l+s}{p}\PYG{l+s}{e}\PYG{l+s}{\PYGZdq{}}\PYG{p}{]}\PYG{o}{==}\PYG{l+s}{\PYGZdq{}}\PYG{l+s}{o}\PYG{l+s}{p}\PYG{l+s}{t}\PYG{l+s}{\PYGZdq{}}
        \PYG{n}{println}\PYG{p}{(}\PYG{n}{i}\PYG{p}{)}
        \PYG{n}{println}\PYG{p}{(}\PYG{n}{md}\PYG{p}{[}\PYG{l+s}{\PYGZdq{}}\PYG{l+s}{P}\PYG{l+s}{a}\PYG{l+s}{r}\PYG{l+s}{a}\PYG{l+s}{m}\PYG{l+s}{e}\PYG{l+s}{t}\PYG{l+s}{e}\PYG{l+s}{r}\PYG{l+s}{s}\PYG{l+s}{\PYGZdq{}}\PYG{p}{]}\PYG{p}{[}\PYG{n}{s2}\PYG{o}{*}\PYG{n}{s1}\PYG{p}{]}\PYG{p}{)}
    \PYG{k}{end}
\PYG{k}{end}
\end{sphinxVerbatim}
}

{

\kern-\sphinxverbatimsmallskipamount\kern-\baselineskip
\kern+\FrameHeightAdjust\kern-\fboxrule
\vspace{\nbsphinxcodecellspacing}

\sphinxsetup{VerbatimColor={named}{white}}
\sphinxsetup{VerbatimBorderColor={named}{nbsphinx-code-border}}
\begin{sphinxVerbatim}[commandchars=\\\{\}]
48
OrderedCollections.OrderedDict\{String,Any\}("init" => 0.41,"max" => 0.6,"min" => 0.4,"type" => "opt")
54
OrderedCollections.OrderedDict\{String,Any\}("init" => 0.031,"max" => 0.045599999999999995,"min" => 0.0304,"type" => "opt")
60
OrderedCollections.OrderedDict\{String,Any\}("init" => 0.013,"max" => 0.018,"min" => 0.012,"type" => "opt")
\end{sphinxVerbatim}
}

\sphinxAtStartPar
Two paramaters are opt\sphinxhyphen{}in from the tim.mads. This file can be changed manually or here by chaning the initial value (\sphinxcode{\sphinxupquote{init}}) to a slightly different one.

{
\sphinxsetup{VerbatimColor={named}{nbsphinx-code-bg}}
\sphinxsetup{VerbatimBorderColor={named}{nbsphinx-code-border}}
\begin{sphinxVerbatim}[commandchars=\\\{\}]
\llap{\color{nbsphinxin}[20]:\,\hspace{\fboxrule}\hspace{\fboxsep}}\PYG{c}{\PYGZsh{}let\PYGZsq{}s opt\PYGZhy{}in a new parameter}
\PYG{n}{md}\PYG{p}{[}\PYG{l+s}{\PYGZdq{}}\PYG{l+s}{P}\PYG{l+s}{a}\PYG{l+s}{r}\PYG{l+s}{a}\PYG{l+s}{m}\PYG{l+s}{e}\PYG{l+s}{t}\PYG{l+s}{e}\PYG{l+s}{r}\PYG{l+s}{s}\PYG{l+s}{\PYGZdq{}}\PYG{p}{]}\PYG{p}{[}\PYG{l+s}{\PYGZdq{}}\PYG{l+s}{p}\PYG{l+s}{5}\PYG{l+s}{4}\PYG{l+s}{\PYGZdq{}}\PYG{p}{]}\PYG{p}{[}\PYG{l+s}{\PYGZdq{}}\PYG{l+s}{t}\PYG{l+s}{y}\PYG{l+s}{p}\PYG{l+s}{e}\PYG{l+s}{\PYGZdq{}}\PYG{p}{]}\PYG{o}{=}\PYG{l+s}{\PYGZdq{}}\PYG{l+s}{o}\PYG{l+s}{p}\PYG{l+s}{t}\PYG{l+s}{\PYGZdq{}} \PYG{c}{\PYGZsh{}K\PYGZus{}S\PYGZus{}fa}
\PYG{n}{md}\PYG{p}{[}\PYG{l+s}{\PYGZdq{}}\PYG{l+s}{P}\PYG{l+s}{a}\PYG{l+s}{r}\PYG{l+s}{a}\PYG{l+s}{m}\PYG{l+s}{e}\PYG{l+s}{t}\PYG{l+s}{e}\PYG{l+s}{r}\PYG{l+s}{s}\PYG{l+s}{\PYGZdq{}}\PYG{p}{]}\PYG{p}{[}\PYG{l+s}{\PYGZdq{}}\PYG{l+s}{p}\PYG{l+s}{5}\PYG{l+s}{4}\PYG{l+s}{\PYGZdq{}}\PYG{p}{]}
\end{sphinxVerbatim}
}

{

\kern-\sphinxverbatimsmallskipamount\kern-\baselineskip
\kern+\FrameHeightAdjust\kern-\fboxrule
\vspace{\nbsphinxcodecellspacing}

\sphinxsetup{VerbatimColor={named}{white}}
\sphinxsetup{VerbatimBorderColor={named}{nbsphinx-code-border}}
\begin{sphinxVerbatim}[commandchars=\\\{\}]
\llap{\color{nbsphinxout}[20]:\,\hspace{\fboxrule}\hspace{\fboxsep}}OrderedCollections.OrderedDict\{String,Any\} with 4 entries:
  "init" => 0.032
  "max"  => 0.0456
  "min"  => 0.0304
  "type" => "opt"
\end{sphinxVerbatim}
}

{
\sphinxsetup{VerbatimColor={named}{nbsphinx-code-bg}}
\sphinxsetup{VerbatimBorderColor={named}{nbsphinx-code-border}}
\begin{sphinxVerbatim}[commandchars=\\\{\}]
\llap{\color{nbsphinxin}[21]:\,\hspace{\fboxrule}\hspace{\fboxsep}}\PYG{n}{println}\PYG{p}{(}\PYG{n}{md}\PYG{p}{[}\PYG{l+s}{\PYGZdq{}}\PYG{l+s}{P}\PYG{l+s}{a}\PYG{l+s}{r}\PYG{l+s}{a}\PYG{l+s}{m}\PYG{l+s}{e}\PYG{l+s}{t}\PYG{l+s}{e}\PYG{l+s}{r}\PYG{l+s}{s}\PYG{l+s}{\PYGZdq{}}\PYG{p}{]}\PYG{p}{[}\PYG{l+s}{\PYGZdq{}}\PYG{l+s}{p}\PYG{l+s}{4}\PYG{l+s}{8}\PYG{l+s}{\PYGZdq{}}\PYG{p}{]}\PYG{p}{)}
\PYG{n}{println}\PYG{p}{(}\PYG{n}{md}\PYG{p}{[}\PYG{l+s}{\PYGZdq{}}\PYG{l+s}{P}\PYG{l+s}{a}\PYG{l+s}{r}\PYG{l+s}{a}\PYG{l+s}{m}\PYG{l+s}{e}\PYG{l+s}{t}\PYG{l+s}{e}\PYG{l+s}{r}\PYG{l+s}{s}\PYG{l+s}{\PYGZdq{}}\PYG{p}{]}\PYG{p}{[}\PYG{l+s}{\PYGZdq{}}\PYG{l+s}{p}\PYG{l+s}{6}\PYG{l+s}{0}\PYG{l+s}{\PYGZdq{}}\PYG{p}{]}\PYG{p}{)}
\PYG{n}{println}\PYG{p}{(}\PYG{n}{md}\PYG{p}{[}\PYG{l+s}{\PYGZdq{}}\PYG{l+s}{P}\PYG{l+s}{a}\PYG{l+s}{r}\PYG{l+s}{a}\PYG{l+s}{m}\PYG{l+s}{e}\PYG{l+s}{t}\PYG{l+s}{e}\PYG{l+s}{r}\PYG{l+s}{s}\PYG{l+s}{\PYGZdq{}}\PYG{p}{]}\PYG{p}{[}\PYG{l+s}{\PYGZdq{}}\PYG{l+s}{p}\PYG{l+s}{5}\PYG{l+s}{4}\PYG{l+s}{\PYGZdq{}}\PYG{p}{]}\PYG{p}{)}
\end{sphinxVerbatim}
}

{

\kern-\sphinxverbatimsmallskipamount\kern-\baselineskip
\kern+\FrameHeightAdjust\kern-\fboxrule
\vspace{\nbsphinxcodecellspacing}

\sphinxsetup{VerbatimColor={named}{white}}
\sphinxsetup{VerbatimBorderColor={named}{nbsphinx-code-border}}
\begin{sphinxVerbatim}[commandchars=\\\{\}]
OrderedCollections.OrderedDict\{String,Any\}("init" => 0.55,"max" => 0.6,"min" => 0.4,"type" => "opt")
OrderedCollections.OrderedDict\{String,Any\}("init" => 0.017,"max" => 0.018,"min" => 0.012,"type" => "opt")
OrderedCollections.OrderedDict\{String,Any\}("init" => 0.032,"max" => 0.045599999999999995,"min" => 0.0304,"type" => "opt")
\end{sphinxVerbatim}
}

{
\sphinxsetup{VerbatimColor={named}{nbsphinx-code-bg}}
\sphinxsetup{VerbatimBorderColor={named}{nbsphinx-code-border}}
\begin{sphinxVerbatim}[commandchars=\\\{\}]
\llap{\color{nbsphinxin}[31]:\,\hspace{\fboxrule}\hspace{\fboxsep}}\PYG{c}{\PYGZsh{}change to new initial values}
\PYG{n}{md}\PYG{p}{[}\PYG{l+s}{\PYGZdq{}}\PYG{l+s}{P}\PYG{l+s}{a}\PYG{l+s}{r}\PYG{l+s}{a}\PYG{l+s}{m}\PYG{l+s}{e}\PYG{l+s}{t}\PYG{l+s}{e}\PYG{l+s}{r}\PYG{l+s}{s}\PYG{l+s}{\PYGZdq{}}\PYG{p}{]}\PYG{p}{[}\PYG{l+s}{\PYGZdq{}}\PYG{l+s}{p}\PYG{l+s}{4}\PYG{l+s}{8}\PYG{l+s}{\PYGZdq{}}\PYG{p}{]}\PYG{p}{[}\PYG{l+s}{\PYGZdq{}}\PYG{l+s}{i}\PYG{l+s}{n}\PYG{l+s}{i}\PYG{l+s}{t}\PYG{l+s}{\PYGZdq{}}\PYG{p}{]}\PYG{o}{=}\PYG{l+m+mf}{0.58}
\PYG{n}{md}\PYG{p}{[}\PYG{l+s}{\PYGZdq{}}\PYG{l+s}{P}\PYG{l+s}{a}\PYG{l+s}{r}\PYG{l+s}{a}\PYG{l+s}{m}\PYG{l+s}{e}\PYG{l+s}{t}\PYG{l+s}{e}\PYG{l+s}{r}\PYG{l+s}{s}\PYG{l+s}{\PYGZdq{}}\PYG{p}{]}\PYG{p}{[}\PYG{l+s}{\PYGZdq{}}\PYG{l+s}{p}\PYG{l+s}{6}\PYG{l+s}{0}\PYG{l+s}{\PYGZdq{}}\PYG{p}{]}\PYG{p}{[}\PYG{l+s}{\PYGZdq{}}\PYG{l+s}{i}\PYG{l+s}{n}\PYG{l+s}{i}\PYG{l+s}{t}\PYG{l+s}{\PYGZdq{}}\PYG{p}{]}\PYG{o}{=}\PYG{l+m+mf}{0.0165} \PYG{c}{\PYGZsh{}K\PYGZus{}S\PYGZus{}pro}
\PYG{n}{md}\PYG{p}{[}\PYG{l+s}{\PYGZdq{}}\PYG{l+s}{P}\PYG{l+s}{a}\PYG{l+s}{r}\PYG{l+s}{a}\PYG{l+s}{m}\PYG{l+s}{e}\PYG{l+s}{t}\PYG{l+s}{e}\PYG{l+s}{r}\PYG{l+s}{s}\PYG{l+s}{\PYGZdq{}}\PYG{p}{]}\PYG{p}{[}\PYG{l+s}{\PYGZdq{}}\PYG{l+s}{p}\PYG{l+s}{5}\PYG{l+s}{4}\PYG{l+s}{\PYGZdq{}}\PYG{p}{]}\PYG{p}{[}\PYG{l+s}{\PYGZdq{}}\PYG{l+s}{i}\PYG{l+s}{n}\PYG{l+s}{i}\PYG{l+s}{t}\PYG{l+s}{\PYGZdq{}}\PYG{p}{]}\PYG{o}{=}\PYG{l+m+mf}{0.045}
\PYG{n}{println}\PYG{p}{(}\PYG{n}{md}\PYG{p}{[}\PYG{l+s}{\PYGZdq{}}\PYG{l+s}{P}\PYG{l+s}{a}\PYG{l+s}{r}\PYG{l+s}{a}\PYG{l+s}{m}\PYG{l+s}{e}\PYG{l+s}{t}\PYG{l+s}{e}\PYG{l+s}{r}\PYG{l+s}{s}\PYG{l+s}{\PYGZdq{}}\PYG{p}{]}\PYG{p}{[}\PYG{l+s}{\PYGZdq{}}\PYG{l+s}{p}\PYG{l+s}{4}\PYG{l+s}{8}\PYG{l+s}{\PYGZdq{}}\PYG{p}{]}\PYG{p}{)}
\PYG{n}{println}\PYG{p}{(}\PYG{n}{md}\PYG{p}{[}\PYG{l+s}{\PYGZdq{}}\PYG{l+s}{P}\PYG{l+s}{a}\PYG{l+s}{r}\PYG{l+s}{a}\PYG{l+s}{m}\PYG{l+s}{e}\PYG{l+s}{t}\PYG{l+s}{e}\PYG{l+s}{r}\PYG{l+s}{s}\PYG{l+s}{\PYGZdq{}}\PYG{p}{]}\PYG{p}{[}\PYG{l+s}{\PYGZdq{}}\PYG{l+s}{p}\PYG{l+s}{6}\PYG{l+s}{0}\PYG{l+s}{\PYGZdq{}}\PYG{p}{]}\PYG{p}{)}
\PYG{n}{println}\PYG{p}{(}\PYG{n}{md}\PYG{p}{[}\PYG{l+s}{\PYGZdq{}}\PYG{l+s}{P}\PYG{l+s}{a}\PYG{l+s}{r}\PYG{l+s}{a}\PYG{l+s}{m}\PYG{l+s}{e}\PYG{l+s}{t}\PYG{l+s}{e}\PYG{l+s}{r}\PYG{l+s}{s}\PYG{l+s}{\PYGZdq{}}\PYG{p}{]}\PYG{p}{[}\PYG{l+s}{\PYGZdq{}}\PYG{l+s}{p}\PYG{l+s}{5}\PYG{l+s}{4}\PYG{l+s}{\PYGZdq{}}\PYG{p}{]}\PYG{p}{)}
\end{sphinxVerbatim}
}

{

\kern-\sphinxverbatimsmallskipamount\kern-\baselineskip
\kern+\FrameHeightAdjust\kern-\fboxrule
\vspace{\nbsphinxcodecellspacing}

\sphinxsetup{VerbatimColor={named}{white}}
\sphinxsetup{VerbatimBorderColor={named}{nbsphinx-code-border}}
\begin{sphinxVerbatim}[commandchars=\\\{\}]
OrderedCollections.OrderedDict\{String,Any\}("init" => 0.58,"max" => 0.6,"min" => 0.4,"type" => "opt")
OrderedCollections.OrderedDict\{String,Any\}("init" => 0.0165,"max" => 0.018,"min" => 0.012,"type" => "opt")
OrderedCollections.OrderedDict\{String,Any\}("init" => 0.045,"max" => 0.045599999999999995,"min" => 0.0304,"type" => "opt")
\end{sphinxVerbatim}
}

{
\sphinxsetup{VerbatimColor={named}{nbsphinx-code-bg}}
\sphinxsetup{VerbatimBorderColor={named}{nbsphinx-code-border}}
\begin{sphinxVerbatim}[commandchars=\\\{\}]
\llap{\color{nbsphinxin}[9]:\,\hspace{\fboxrule}\hspace{\fboxsep}}\PYG{c}{\PYGZsh{}we can also change the boundary of the allowed interval}
\PYG{c}{\PYGZsh{}md[\PYGZdq{}Parameters\PYGZdq{}][\PYGZdq{}p48\PYGZdq{}][\PYGZdq{}init\PYGZdq{}]=0.59}
\PYG{c}{\PYGZsh{}md[\PYGZdq{}Parameters\PYGZdq{}][\PYGZdq{}p48\PYGZdq{}][\PYGZdq{}max\PYGZdq{}]=0.7}
\PYG{c}{\PYGZsh{}md[\PYGZdq{}Parameters\PYGZdq{}][\PYGZdq{}p48\PYGZdq{}][\PYGZdq{}min\PYGZdq{}]=0.35}
\end{sphinxVerbatim}
}

{

\kern-\sphinxverbatimsmallskipamount\kern-\baselineskip
\kern+\FrameHeightAdjust\kern-\fboxrule
\vspace{\nbsphinxcodecellspacing}

\sphinxsetup{VerbatimColor={named}{white}}
\sphinxsetup{VerbatimBorderColor={named}{nbsphinx-code-border}}
\begin{sphinxVerbatim}[commandchars=\\\{\}]
\llap{\color{nbsphinxout}[9]:\,\hspace{\fboxrule}\hspace{\fboxsep}}0.35
\end{sphinxVerbatim}
}

{
\sphinxsetup{VerbatimColor={named}{nbsphinx-code-bg}}
\sphinxsetup{VerbatimBorderColor={named}{nbsphinx-code-border}}
\begin{sphinxVerbatim}[commandchars=\\\{\}]
\llap{\color{nbsphinxin}[32]:\,\hspace{\fboxrule}\hspace{\fboxsep}}\PYG{c}{\PYGZsh{}rerun the model and compare the results with the previous forward run}
\PYG{n}{output1}\PYG{o}{=}\PYG{n}{Mads}\PYG{o}{.}\PYG{n}{forward}\PYG{p}{(}\PYG{n}{md}\PYG{p}{)}
\end{sphinxVerbatim}
}

{

\kern-\sphinxverbatimsmallskipamount\kern-\baselineskip
\kern+\FrameHeightAdjust\kern-\fboxrule
\vspace{\nbsphinxcodecellspacing}

\sphinxsetup{VerbatimColor={named}{white}}
\sphinxsetup{VerbatimBorderColor={named}{nbsphinx-code-border}}
\begin{sphinxVerbatim}[commandchars=\\\{\}]
\llap{\color{nbsphinxout}[32]:\,\hspace{\fboxrule}\hspace{\fboxsep}}OrderedCollections.OrderedDict\{Any,Float64\} with 7 entries:
  "o2" => 145.06
  "o3" => 48.3842
  "o1" => 363.488
  "o4" => 35.0174
  "o5" => 11.6574
  "o6" => 6.51656
  "o7" => -1.14602e-312
\end{sphinxVerbatim}
}

\sphinxAtStartPar
“o2” =\textgreater{} 145.362 “o3” =\textgreater{} 48.578 “o1” =\textgreater{} 363.333 “o4” =\textgreater{} 35.1163 “o5” =\textgreater{} 11.6628 “o6” =\textgreater{} 6.51573 “o7” =\textgreater{} 3.04116e\sphinxhyphen{}312


\subsection{2. Model Calibration}
\label{\detokenize{jupyter_notebook/parameter-calibration:2.-Model-Calibration}}
\sphinxAtStartPar
Once the sensitivity of the model outputs to changes in model parameters are established we can start the calibration. The elaborated sensitivy analysis is required, for an in\sphinxhyphen{}depth model calibration.

{
\sphinxsetup{VerbatimColor={named}{nbsphinx-code-bg}}
\sphinxsetup{VerbatimBorderColor={named}{nbsphinx-code-border}}
\begin{sphinxVerbatim}[commandchars=\\\{\}]
\llap{\color{nbsphinxin}[34]:\,\hspace{\fboxrule}\hspace{\fboxsep}}\PYG{n}{p}\PYG{p}{,} \PYG{n}{r} \PYG{o}{=} \PYG{n}{Mads}\PYG{o}{.}\PYG{n}{calibraterandom}\PYG{p}{(}\PYG{n}{md}\PYG{p}{)}
\end{sphinxVerbatim}
}

{

\kern-\sphinxverbatimsmallskipamount\kern-\baselineskip
\kern+\FrameHeightAdjust\kern-\fboxrule
\vspace{\nbsphinxcodecellspacing}

\sphinxsetup{VerbatimColor={named}{white}}
\sphinxsetup{VerbatimBorderColor={named}{nbsphinx-code-border}}
\begin{sphinxVerbatim}[commandchars=\\\{\}]
\llap{\color{nbsphinxout}[34]:\,\hspace{\fboxrule}\hspace{\fboxsep}}(OrderedCollections.OrderedDict("p1" => 0.09060657861205099,"p2" => 0.2964692632944874,"p3" => 0.18538822521329718,"p4" => 0.21021796952211014,"p5" => 0.20993757892890663,"p6" => 0.0025810430070856297,"p7" => 0.004153144361053321,"p8" => 0.006812701647314224,"p9" => 0.029624259152454625,"p10" => 0.028993549342676592…), OptimBase.MultivariateOptimizationResults\{LsqFit.LevenbergMarquardt,Float64,1\}(LsqFit.LevenbergMarquardt(), [-0.4889181593485484, 1.1927438925052574, -0.3739490066652115, 0.2583123288342893, -0.9293788161320726, -0.8970848356758778, -0.6686810831789546, -0.5646712759835035, 0.3221739693428631, -0.16853854003002233  …  -0.515933804055272, -0.9755102180590636, -1.1189154560307426, 1.0593944723620046, 0.1384639308865269, -0.5026261086108612, -0.012128162836436718, 0.28788576425004025, -0.0029143278533535384, -1.1138366569543616], [-0.4889181593485484, 1.1927438925052574, -0.3739490066652115, 0.2583123288342893, -0.9293788161320726, -0.8970848356758778, -0.6686810831789546, -0.5646712759835035, 0.3221739693428631, -0.16853854003002233  …  -0.515933804055272, -0.9755102180590636, -1.1189154560307426, 1.0593944723620046, 0.1384639308865269, -0.5026261086108612, -0.012128162836436718, 0.28788576425004025, -0.0029143278533535384, -1.1138366569543616], 150.99483823310115, 10, true, false, 0.0001, 0.0, false, 0.001, 0.0, false, 1.0e-6, 0.0, false, Iter     Function value   Gradient norm
------   --------------   --------------
, 1101, 10, 0))
\end{sphinxVerbatim}
}

{
\sphinxsetup{VerbatimColor={named}{nbsphinx-code-bg}}
\sphinxsetup{VerbatimBorderColor={named}{nbsphinx-code-border}}
\begin{sphinxVerbatim}[commandchars=\\\{\}]
\llap{\color{nbsphinxin}[35]:\,\hspace{\fboxrule}\hspace{\fboxsep}}\PYG{n}{Mads}\PYG{o}{.}\PYG{n}{showobservations}\PYG{p}{(}\PYG{n}{md}\PYG{p}{)}
\end{sphinxVerbatim}
}

{

\kern-\sphinxverbatimsmallskipamount\kern-\baselineskip
\kern+\FrameHeightAdjust\kern-\fboxrule
\vspace{\nbsphinxcodecellspacing}

\sphinxsetup{VerbatimColor={named}{white}}
\sphinxsetup{VerbatimBorderColor={named}{nbsphinx-code-border}}
\begin{sphinxVerbatim}[commandchars=\\\{\}]
o1         target =         363.333 weight =               1
o2         target =         145.362 weight =               1
o3         target =          48.578 weight =               1
o4         target =         35.1163 weight =               1
o5         target =         11.6628 weight =               1
o6         target =         6.51573 weight =               1
o7         target =    3.04116e-312 weight =               1
Number of observations is 7
\end{sphinxVerbatim}
}

{
\sphinxsetup{VerbatimColor={named}{nbsphinx-code-bg}}
\sphinxsetup{VerbatimBorderColor={named}{nbsphinx-code-border}}
\begin{sphinxVerbatim}[commandchars=\\\{\}]
\llap{\color{nbsphinxin}[52]:\,\hspace{\fboxrule}\hspace{\fboxsep}}\PYG{n}{println}\PYG{p}{(}\PYG{l+s}{\PYGZdq{}}\PYG{l+s}{i}\PYG{l+s}{n}\PYG{l+s}{i}\PYG{l+s}{t}\PYG{l+s}{i}\PYG{l+s}{a}\PYG{l+s}{l}\PYG{l+s}{ }\PYG{l+s}{g}\PYG{l+s}{u}\PYG{l+s}{e}\PYG{l+s}{s}\PYG{l+s}{s}\PYG{l+s}{:}\PYG{l+s}{\PYGZdq{}}\PYG{p}{)}
\PYG{n}{println}\PYG{p}{(}\PYG{l+s}{\PYGZdq{}}\PYG{l+s}{p}\PYG{l+s}{4}\PYG{l+s}{8}\PYG{l+s}{=}\PYG{l+s}{\PYGZdq{}}\PYG{p}{,}\PYG{n}{md}\PYG{p}{[}\PYG{l+s}{\PYGZdq{}}\PYG{l+s}{P}\PYG{l+s}{a}\PYG{l+s}{r}\PYG{l+s}{a}\PYG{l+s}{m}\PYG{l+s}{e}\PYG{l+s}{t}\PYG{l+s}{e}\PYG{l+s}{r}\PYG{l+s}{s}\PYG{l+s}{\PYGZdq{}}\PYG{p}{]}\PYG{p}{[}\PYG{l+s}{\PYGZdq{}}\PYG{l+s}{p}\PYG{l+s}{4}\PYG{l+s}{8}\PYG{l+s}{\PYGZdq{}}\PYG{p}{]}\PYG{p}{[}\PYG{l+s}{\PYGZdq{}}\PYG{l+s}{i}\PYG{l+s}{n}\PYG{l+s}{i}\PYG{l+s}{t}\PYG{l+s}{\PYGZdq{}}\PYG{p}{]}\PYG{p}{)}
\PYG{n}{println}\PYG{p}{(}\PYG{l+s}{\PYGZdq{}}\PYG{l+s}{p}\PYG{l+s}{6}\PYG{l+s}{0}\PYG{l+s}{=}\PYG{l+s}{\PYGZdq{}}\PYG{p}{,}\PYG{n}{md}\PYG{p}{[}\PYG{l+s}{\PYGZdq{}}\PYG{l+s}{P}\PYG{l+s}{a}\PYG{l+s}{r}\PYG{l+s}{a}\PYG{l+s}{m}\PYG{l+s}{e}\PYG{l+s}{t}\PYG{l+s}{e}\PYG{l+s}{r}\PYG{l+s}{s}\PYG{l+s}{\PYGZdq{}}\PYG{p}{]}\PYG{p}{[}\PYG{l+s}{\PYGZdq{}}\PYG{l+s}{p}\PYG{l+s}{6}\PYG{l+s}{0}\PYG{l+s}{\PYGZdq{}}\PYG{p}{]}\PYG{p}{[}\PYG{l+s}{\PYGZdq{}}\PYG{l+s}{i}\PYG{l+s}{n}\PYG{l+s}{i}\PYG{l+s}{t}\PYG{l+s}{\PYGZdq{}}\PYG{p}{]}\PYG{p}{)}
\PYG{n}{println}\PYG{p}{(}\PYG{l+s}{\PYGZdq{}}\PYG{l+s}{p}\PYG{l+s}{5}\PYG{l+s}{4}\PYG{l+s}{=}\PYG{l+s}{\PYGZdq{}}\PYG{p}{,}\PYG{n}{md}\PYG{p}{[}\PYG{l+s}{\PYGZdq{}}\PYG{l+s}{P}\PYG{l+s}{a}\PYG{l+s}{r}\PYG{l+s}{a}\PYG{l+s}{m}\PYG{l+s}{e}\PYG{l+s}{t}\PYG{l+s}{e}\PYG{l+s}{r}\PYG{l+s}{s}\PYG{l+s}{\PYGZdq{}}\PYG{p}{]}\PYG{p}{[}\PYG{l+s}{\PYGZdq{}}\PYG{l+s}{p}\PYG{l+s}{5}\PYG{l+s}{4}\PYG{l+s}{\PYGZdq{}}\PYG{p}{]}\PYG{p}{[}\PYG{l+s}{\PYGZdq{}}\PYG{l+s}{i}\PYG{l+s}{n}\PYG{l+s}{i}\PYG{l+s}{t}\PYG{l+s}{\PYGZdq{}}\PYG{p}{]}\PYG{p}{)}
\PYG{n}{println}\PYG{p}{(}\PYG{l+s}{\PYGZdq{}}\PYG{l+s}{ }\PYG{l+s}{\PYGZdq{}}\PYG{p}{)}
\PYG{n}{println}\PYG{p}{(}\PYG{l+s}{\PYGZdq{}}\PYG{l+s}{c}\PYG{l+s}{a}\PYG{l+s}{l}\PYG{l+s}{i}\PYG{l+s}{b}\PYG{l+s}{r}\PYG{l+s}{a}\PYG{l+s}{t}\PYG{l+s}{e}\PYG{l+s}{d}\PYG{l+s}{ }\PYG{l+s}{p}\PYG{l+s}{a}\PYG{l+s}{r}\PYG{l+s}{a}\PYG{l+s}{m}\PYG{l+s}{e}\PYG{l+s}{t}\PYG{l+s}{e}\PYG{l+s}{r}\PYG{l+s}{s}\PYG{l+s}{:}\PYG{l+s}{\PYGZdq{}}\PYG{p}{)}
\PYG{n}{println}\PYG{p}{(}\PYG{l+s}{\PYGZdq{}}\PYG{l+s}{p}\PYG{l+s}{4}\PYG{l+s}{8}\PYG{l+s}{=}\PYG{l+s}{\PYGZdq{}}\PYG{p}{,}\PYG{n}{round}\PYG{p}{(}\PYG{n}{p}\PYG{p}{[}\PYG{l+s}{\PYGZdq{}}\PYG{l+s}{p}\PYG{l+s}{4}\PYG{l+s}{8}\PYG{l+s}{\PYGZdq{}}\PYG{p}{]}\PYG{p}{,}\PYG{n}{digits}\PYG{o}{=}\PYG{l+m+mi}{3}\PYG{p}{)}\PYG{p}{)}
\PYG{n}{println}\PYG{p}{(}\PYG{l+s}{\PYGZdq{}}\PYG{l+s}{p}\PYG{l+s}{6}\PYG{l+s}{0}\PYG{l+s}{=}\PYG{l+s}{\PYGZdq{}}\PYG{p}{,}\PYG{n}{round}\PYG{p}{(}\PYG{n}{p}\PYG{p}{[}\PYG{l+s}{\PYGZdq{}}\PYG{l+s}{p}\PYG{l+s}{6}\PYG{l+s}{0}\PYG{l+s}{\PYGZdq{}}\PYG{p}{]}\PYG{p}{,}\PYG{n}{digits}\PYG{o}{=}\PYG{l+m+mi}{3}\PYG{p}{)}\PYG{p}{)}
\PYG{n}{println}\PYG{p}{(}\PYG{l+s}{\PYGZdq{}}\PYG{l+s}{p}\PYG{l+s}{5}\PYG{l+s}{4}\PYG{l+s}{=}\PYG{l+s}{\PYGZdq{}}\PYG{p}{,}\PYG{n}{round}\PYG{p}{(}\PYG{n}{p}\PYG{p}{[}\PYG{l+s}{\PYGZdq{}}\PYG{l+s}{p}\PYG{l+s}{5}\PYG{l+s}{4}\PYG{l+s}{\PYGZdq{}}\PYG{p}{]}\PYG{p}{,}\PYG{n}{digits}\PYG{o}{=}\PYG{l+m+mi}{3}\PYG{p}{)}\PYG{p}{)}
\PYG{n}{println}\PYG{p}{(}\PYG{l+s}{\PYGZdq{}}\PYG{l+s}{ }\PYG{l+s}{\PYGZdq{}}\PYG{p}{)}
\PYG{n}{println}\PYG{p}{(}\PYG{l+s}{\PYGZdq{}}\PYG{l+s}{t}\PYG{l+s}{r}\PYG{l+s}{u}\PYG{l+s}{e}\PYG{l+s}{ }\PYG{l+s}{p}\PYG{l+s}{a}\PYG{l+s}{r}\PYG{l+s}{a}\PYG{l+s}{m}\PYG{l+s}{e}\PYG{l+s}{t}\PYG{l+s}{e}\PYG{l+s}{r}\PYG{l+s}{s}\PYG{l+s}{:}\PYG{l+s}{\PYGZdq{}}\PYG{p}{)}
\PYG{n}{println}\PYG{p}{(}\PYG{l+s}{\PYGZdq{}}\PYG{l+s}{p}\PYG{l+s}{4}\PYG{l+s}{8}\PYG{l+s}{=}\PYG{l+s}{\PYGZdq{}}\PYG{p}{,}\PYG{l+m+mf}{0.41}\PYG{p}{)}
\PYG{n}{println}\PYG{p}{(}\PYG{l+s}{\PYGZdq{}}\PYG{l+s}{p}\PYG{l+s}{6}\PYG{l+s}{0}\PYG{l+s}{=}\PYG{l+s}{\PYGZdq{}}\PYG{p}{,}\PYG{l+m+mf}{0.013}\PYG{p}{)}
\PYG{n}{println}\PYG{p}{(}\PYG{l+s}{\PYGZdq{}}\PYG{l+s}{p}\PYG{l+s}{5}\PYG{l+s}{4}\PYG{l+s}{=}\PYG{l+s}{\PYGZdq{}}\PYG{p}{,}\PYG{l+m+mf}{0.031}\PYG{p}{)}
\end{sphinxVerbatim}
}

{

\kern-\sphinxverbatimsmallskipamount\kern-\baselineskip
\kern+\FrameHeightAdjust\kern-\fboxrule
\vspace{\nbsphinxcodecellspacing}

\sphinxsetup{VerbatimColor={named}{white}}
\sphinxsetup{VerbatimBorderColor={named}{nbsphinx-code-border}}
\begin{sphinxVerbatim}[commandchars=\\\{\}]
initial guess:
p48=0.58
p60=0.0165
p54=0.045

calibrated parameters:
p48=0.486
p60=0.013
p54=0.039

true parameters:
p48=0.41
p60=0.013
p54=0.031
\end{sphinxVerbatim}
}

{
\sphinxsetup{VerbatimColor={named}{nbsphinx-code-bg}}
\sphinxsetup{VerbatimBorderColor={named}{nbsphinx-code-border}}
\begin{sphinxVerbatim}[commandchars=\\\{\}]
\llap{\color{nbsphinxin}[51]:\,\hspace{\fboxrule}\hspace{\fboxsep}}\PYG{c}{\PYGZsh{}Only plot the outputs which participate in this match}

\PYG{n}{est} \PYG{o}{=} \PYG{n}{Mads}\PYG{o}{.}\PYG{n}{forward}\PYG{p}{(}\PYG{n}{md}\PYG{p}{,} \PYG{n}{p}\PYG{p}{)}\PYG{p}{;}
\PYG{n}{obs} \PYG{o}{=} \PYG{n}{Mads}\PYG{o}{.}\PYG{n}{getobstarget}\PYG{p}{(}\PYG{n}{md}\PYG{p}{)}\PYG{p}{;}

\PYG{n}{idx\PYGZus{}out} \PYG{o}{=} \PYG{n}{collect}\PYG{p}{(}\PYG{l+m+mi}{1}\PYG{o}{:}\PYG{l+m+mi}{1}\PYG{o}{:}\PYG{l+m+mi}{7}\PYG{p}{)}
\PYG{n}{key\PYGZus{}out} \PYG{o}{=} \PYG{p}{[}\PYG{n}{string}\PYG{p}{(}\PYG{l+s+sc}{\PYGZsq{}o\PYGZsq{}}\PYG{p}{,}\PYG{n}{i}\PYG{p}{)} \PYG{k}{for} \PYG{n}{i} \PYG{k+kp}{in} \PYG{n}{idx\PYGZus{}out}\PYG{p}{]}

\PYG{n}{est\PYGZus{}filt} \PYG{o}{=} \PYG{p}{[}\PYG{n}{est}\PYG{p}{[}\PYG{n}{i}\PYG{p}{]} \PYG{k}{for} \PYG{n}{i} \PYG{k+kp}{in} \PYG{n}{key\PYGZus{}out}\PYG{p}{]}\PYG{p}{;}
\PYG{n}{obs\PYGZus{}filt} \PYG{o}{=} \PYG{p}{[}\PYG{n}{obs}\PYG{p}{[}\PYG{n}{i}\PYG{p}{]} \PYG{k}{for} \PYG{n}{i} \PYG{k+kp}{in} \PYG{n}{idx\PYGZus{}out}\PYG{p}{]}\PYG{p}{;}
\PYG{n}{Mads}\PYG{o}{.}\PYG{n}{plotseries}\PYG{p}{(}\PYG{p}{[}\PYG{n}{obs\PYGZus{}filt} \PYG{n}{est\PYGZus{}filt}\PYG{p}{]}\PYG{p}{;} \PYG{n}{names}\PYG{o}{=}\PYG{p}{[}\PYG{l+s}{\PYGZdq{}}\PYG{l+s}{T}\PYG{l+s}{r}\PYG{l+s}{u}\PYG{l+s}{t}\PYG{l+s}{h}\PYG{l+s}{\PYGZdq{}}\PYG{p}{,} \PYG{l+s}{\PYGZdq{}}\PYG{l+s}{E}\PYG{l+s}{s}\PYG{l+s}{t}\PYG{l+s}{.}\PYG{l+s}{\PYGZdq{}}\PYG{p}{]}\PYG{p}{)}
\end{sphinxVerbatim}
}

\hrule height -\fboxrule\relax
\vspace{\nbsphinxcodecellspacing}

\makeatletter\setbox\nbsphinxpromptbox\box\voidb@x\makeatother

\begin{nbsphinxfancyoutput}

\noindent\sphinxincludegraphics{{jupyter_notebook_parameter-calibration_24_0}.svg}

\end{nbsphinxfancyoutput}

{

\kern-\sphinxverbatimsmallskipamount\kern-\baselineskip
\kern+\FrameHeightAdjust\kern-\fboxrule
\vspace{\nbsphinxcodecellspacing}

\sphinxsetup{VerbatimColor={named}{white}}
\sphinxsetup{VerbatimBorderColor={named}{nbsphinx-code-border}}
\begin{sphinxVerbatim}[commandchars=\\\{\}]

\end{sphinxVerbatim}
}

{
\sphinxsetup{VerbatimColor={named}{nbsphinx-code-bg}}
\sphinxsetup{VerbatimBorderColor={named}{nbsphinx-code-border}}
\begin{sphinxVerbatim}[commandchars=\\\{\}]
\llap{\color{nbsphinxin}[ ]:\,\hspace{\fboxrule}\hspace{\fboxsep}}
\end{sphinxVerbatim}
}


\section{PH Control}
\label{\detokenize{phcontrol:ph-control}}\label{\detokenize{phcontrol::doc}}
\sphinxAtStartPar
The two\sphinxhyphen{}phase AnDMBR simulates a rumen environment to enhance the rates of hydrolysis and acidogenesis. A large amount of volatile fatty acids that decreases PH is produced, which could lead to digester failure. In real\sphinxhyphen{}life conditions, sodium hydroxide is added into the first\sphinxhyphen{}phase reactor. This maintains a PH around 6.3 and ensures optimal microbial activities, which is similar to the rumen reactor of a cow. Similarly, the PH in the second\sphinxhyphen{}phase methane\sphinxhyphen{}producing AnDMBR is maintained at 7.2 to provide optimal reactor functionality. In the AnDMBR model, we retain the appropriate levels of PH within each reactor phase by adding corresponding cation mass, which is estimated via the data consistent inversion method %
\begin{footnote}[1]\sphinxAtStartFootnote
\sphinxhref{https://epubs.siam.org/doi/abs/10.1137/16M1087229?mobileUi=0}{T. Butler and J. Jakeman and T. Wildey, Combining Push\sphinxhyphen{}Forward Measures and Bayes’ Rule to Construct Consistent Solutions to Stochastic Inverse Problems, SIAM Journal on Scientific Computing, 40, A984\sphinxhyphen{}A1011 (2018).}
%
\end{footnote}. The cation amount can then be used to calculate the amount of sodium hydroxide in real implementation. The method and more in\sphinxhyphen{}depth mathematical description of the \sphinxtitleref{ph\sphinxhyphen{}control} methods can be found in Dr. Zhang’s Ph.D. thesis.

\sphinxAtStartPar
\sphinxstylestrong{Data Consistent Inversion Method}

\sphinxAtStartPar
Data consistent inversion method is used here to determine the cation needed to adjust the PH level in the reactor. It is first introduced for uncertainty quantification in inverse problems in \sphinxfootnotemark[1] where the ideas and derivations behind it were discussed in great detail. Since the goal here is to solve the inverse problem where target PH, \(\mu^*\), can be achieved, we can apply this method to our model. Our goal here is to control the PH in the reactor within an acceptable range. We denote the target PH value by \(\mu^*\), a small deviation is allowed around \(\mu^*\) to meet real\sphinxhyphen{}life conditions.

\sphinxAtStartPar
Gaussian (or normal) distribution denoted by \(N(\mu, \sigma^2)\) has a property that
there is a probability of 99.73\% that its observation denoted by \(X\) will lie within three standard deviations (\(3\sigma\)) of the mean (\(\mu\)). This is the so\sphinxhyphen{}called three\sphinxhyphen{}sigma rule of thumb, to write it in mathematical notation cite\{3sigma\},
\begin{equation*}
\begin{split}P(\mu-3\sigma \leq X \leq \mu+3\sigma) \approx 99.73\%.\end{split}
\end{equation*}
\sphinxAtStartPar
We would like to use this special property of Gaussian distribution to achieve a high chance of predicted PH being in the three standard deviations of \(\mu^*\). Therefore we assume that our predicted PH value follows a Gaussian distribution %
\begin{footnote}[2]\sphinxAtStartFootnote
We can also assume different distributions for PH.
%
\end{footnote}, \(N(\mu^*, \sigma^2)\). We first propose an initial probability  density function (pdf) of cationfootnote\{A uniform distribution \(U(0,0.2)\) is assumed in our model.\}, then the data consistent inversion method is applied to update the pdf  of cation such that the pdf of \(N(\mu^*, \sigma^2)\) is returned when the updated pdf is propagated through the model.

\sphinxAtStartPar
The following figure describes the relation between cation density, model and PH density.

\noindent\sphinxincludegraphics[width=1050\sphinxpxdimen,height=375\sphinxpxdimen]{{pdftonormal}.png}

\sphinxAtStartPar
The methodology of data consistent inversion %
\begin{footnote}[3]\sphinxAtStartFootnote
Assumptions are not discussed here for the ease of explanation.
%
\end{footnote} guarantees that when a sample of the updated pdf of cation is put in the model input, there is a probability of 99.73\% that the predicted PH will be within three standard deviations around \(\mu^*\) (grey area). It is straightforward to get the conclusion  that the smaller \(\sigma\) we choose in \(N(\mu^*, \sigma^2)\), the more accurate the PH will be.


\subsection{PH Control for One\sphinxhyphen{}Phase and Two\sphinxhyphen{}Phase Reactor}
\label{\detokenize{ph12:ph-control-for-one-phase-and-two-phase-reactor}}\label{\detokenize{ph12::doc}}
\sphinxAtStartPar
Phase 1 of the digester represents the condition where membrane blocks the solid materials that are part of the effluent. Here we setup the corresponding SRT to define the time required for the solids to dilute in phase 1. As mentioned above, during this phase VFAs are produced, which reduces the ph of the digester.

\sphinxAtStartPar
To prevent the ph from further decreasing, we developed the ph\sphinxhyphen{}control algorithm that allows us to adjust the ph at the beginning of each phase.
Implemented in this study ph\sphinxhyphen{}control method (see \sphinxcode{\sphinxupquote{ph\_control}} function in  \sphinxcode{\sphinxupquote{adm1f\_utils.py}}) is designed to implement data consistent inversion method in PH control (described in the section below). The \sphinxcode{\sphinxupquote{ph}} method can be used to predict the PH value. The \sphinxcode{\sphinxupquote{ph\_control}} method is used to return the cation concentration to be changed in the input in order to reach the target PH value (\(\mu^*\)). \(\sigma\) is used to denote the standard deviation we allow the PH value to vary around \(\mu^*\).

\sphinxAtStartPar
The following figure describes the basic strategy used for PH control

\noindent\sphinxincludegraphics[width=559\sphinxpxdimen,height=416\sphinxpxdimen]{{phcontrol}.png}

\sphinxAtStartPar
The idea can also be explained in the following flow chart

\noindent\sphinxincludegraphics[width=1502\sphinxpxdimen,height=1125\sphinxpxdimen]{{flow1}.png}

\sphinxAtStartPar
Similar idea can then be applied to the two\sphinxhyphen{}phase reactor. We can configure the two\sphinxhyphen{}phase reactor where volume, flow rate, t\_resx (t\_resx = SRT\sphinxhyphen{}HRT) of each phase can be set manually to simulate the real reactor. The implementations in both phase 1 and 2 are similar to the above implementation in the one\sphinxhyphen{}phase reactor, the only difference is that the output of phase 1 is now the input of phase 2.

\sphinxAtStartPar
In the {\hyperref[\detokenize{jupyter_notebook/ph-control::doc}]{\sphinxcrossref{\DUrole{doc}{Ipython notebook}}}}, \sphinxcode{\sphinxupquote{reactor2\_cat(target\_1, target\_2, **kwargs)}} function is used to calculate the corresponding cation for each phase in order to control the PH level for each phase. Note that the target PH for phase 1 and 2 (target\_1, target\_2) can be set different when using this function under the corresponding configurations.


\subsection{ADM1F\_SRT: Ph control method}
\label{\detokenize{jupyter_notebook/ph-control:ADM1F_SRT:-Ph-control-method}}\label{\detokenize{jupyter_notebook/ph-control::doc}}
\sphinxAtStartPar
The \sphinxcode{\sphinxupquote{ph control}} method was developed by Wenjuan Zhang and uses Data Consistent Inversion Method.

\sphinxAtStartPar
Authors: Wenjuan Zhang and Elchin Jafarov

{
\sphinxsetup{VerbatimColor={named}{nbsphinx-code-bg}}
\sphinxsetup{VerbatimBorderColor={named}{nbsphinx-code-border}}
\begin{sphinxVerbatim}[commandchars=\\\{\}]
\llap{\color{nbsphinxin}[1]:\,\hspace{\fboxrule}\hspace{\fboxsep}}\PYG{k+kn}{import} \PYG{n+nn}{os}
\PYG{k+kn}{import} \PYG{n+nn}{adm1f\PYGZus{}utils} \PYG{k}{as} \PYG{n+nn}{adm1fu}
\PYG{k+kn}{import} \PYG{n+nn}{numpy} \PYG{k}{as} \PYG{n+nn}{np}
\PYG{k+kn}{from} \PYG{n+nn}{scipy}\PYG{n+nn}{.}\PYG{n+nn}{stats} \PYG{k+kn}{import} \PYG{n}{norm}
\PYG{k+kn}{import} \PYG{n+nn}{matplotlib}\PYG{n+nn}{.}\PYG{n+nn}{pyplot} \PYG{k}{as} \PYG{n+nn}{plt}
\PYG{o}{\PYGZpc{}}\PYG{k}{matplotlib} inline
\end{sphinxVerbatim}
}


\subsubsection{1. Relation between cation and PH}
\label{\detokenize{jupyter_notebook/ph-control:1.-Relation-between-cation-and-PH}}
\sphinxAtStartPar
Here we explore the cation/ph relationships using diffrerent configurations. Note, based in the results will be different based on the ADM1F code version (i.e. original or SRT).

{
\sphinxsetup{VerbatimColor={named}{nbsphinx-code-bg}}
\sphinxsetup{VerbatimBorderColor={named}{nbsphinx-code-border}}
\begin{sphinxVerbatim}[commandchars=\\\{\}]
\llap{\color{nbsphinxin}[2]:\,\hspace{\fboxrule}\hspace{\fboxsep}}\PYG{c+c1}{\PYGZsh{} navigate to simulations folder}
\PYG{n}{os}\PYG{o}{.}\PYG{n}{chdir}\PYG{p}{(}\PYG{l+s+s1}{\PYGZsq{}}\PYG{l+s+s1}{../simulations}\PYG{l+s+s1}{\PYGZsq{}}\PYG{p}{)}
\end{sphinxVerbatim}
}

{
\sphinxsetup{VerbatimColor={named}{nbsphinx-code-bg}}
\sphinxsetup{VerbatimBorderColor={named}{nbsphinx-code-border}}
\begin{sphinxVerbatim}[commandchars=\\\{\}]
\llap{\color{nbsphinxin}[3]:\,\hspace{\fboxrule}\hspace{\fboxsep}}\PYG{c+c1}{\PYGZsh{} Configuration of the one\PYGZhy{}phase reactor}
\PYG{n}{config\PYGZus{}default} \PYG{o}{=} \PYG{p}{\PYGZob{}}\PYG{l+s+s1}{\PYGZsq{}}\PYG{l+s+s1}{Vliq}\PYG{l+s+s1}{\PYGZsq{}}\PYG{p}{:}\PYG{l+m+mi}{3400}\PYG{p}{,} \PYG{l+s+s1}{\PYGZsq{}}\PYG{l+s+s1}{t\PYGZus{}resx}\PYG{l+s+s1}{\PYGZsq{}}\PYG{p}{:}\PYG{l+m+mi}{0}\PYG{p}{,} \PYG{l+s+s1}{\PYGZsq{}}\PYG{l+s+s1}{Q}\PYG{l+s+s1}{\PYGZsq{}}\PYG{p}{:}\PYG{l+m+mi}{134}\PYG{p}{\PYGZcb{}}
\PYG{n}{config1} \PYG{o}{=} \PYG{p}{\PYGZob{}}\PYG{l+s+s1}{\PYGZsq{}}\PYG{l+s+s1}{Vliq}\PYG{l+s+s1}{\PYGZsq{}}\PYG{p}{:}\PYG{l+m+mi}{340}\PYG{p}{,} \PYG{l+s+s1}{\PYGZsq{}}\PYG{l+s+s1}{t\PYGZus{}resx}\PYG{l+s+s1}{\PYGZsq{}}\PYG{p}{:}\PYG{l+m+mf}{1.5}\PYG{p}{,} \PYG{l+s+s1}{\PYGZsq{}}\PYG{l+s+s1}{Q}\PYG{l+s+s1}{\PYGZsq{}}\PYG{p}{:}\PYG{l+m+mi}{618}\PYG{p}{\PYGZcb{}}
\PYG{n}{config2} \PYG{o}{=} \PYG{p}{\PYGZob{}}\PYG{l+s+s1}{\PYGZsq{}}\PYG{l+s+s1}{Vliq}\PYG{l+s+s1}{\PYGZsq{}}\PYG{p}{:}\PYG{l+m+mi}{3400}\PYG{p}{,} \PYG{l+s+s1}{\PYGZsq{}}\PYG{l+s+s1}{t\PYGZus{}resx}\PYG{l+s+s1}{\PYGZsq{}}\PYG{p}{:}\PYG{l+m+mi}{700}\PYG{p}{,} \PYG{l+s+s1}{\PYGZsq{}}\PYG{l+s+s1}{Q}\PYG{l+s+s1}{\PYGZsq{}}\PYG{p}{:}\PYG{l+m+mi}{618}\PYG{p}{\PYGZcb{}}
\end{sphinxVerbatim}
}

\sphinxAtStartPar
\sphinxstylestrong{Configurations}


\begin{savenotes}\sphinxattablestart
\centering
\begin{tabulary}{\linewidth}[t]{|T|T|T|T|}
\hline
\sphinxstyletheadfamily 
\sphinxAtStartPar
Configuration
&\sphinxstyletheadfamily 
\sphinxAtStartPar
Vliq (m\(^3\))
&\sphinxstyletheadfamily 
\sphinxAtStartPar
t\_resx (d)
&\sphinxstyletheadfamily 
\sphinxAtStartPar
Q (m\(^3\)/d)
\\
\hline
\sphinxAtStartPar
Default
&
\sphinxAtStartPar
3400
&
\sphinxAtStartPar
0
&
\sphinxAtStartPar
134
\\
\hline
\sphinxAtStartPar
Phase 1
&
\sphinxAtStartPar
340
&
\sphinxAtStartPar
1.5
&
\sphinxAtStartPar
618
\\
\hline
\sphinxAtStartPar
Phase 2
&
\sphinxAtStartPar
3400
&
\sphinxAtStartPar
700
&
\sphinxAtStartPar
618/—
\\
\hline
\end{tabulary}
\par
\sphinxattableend\end{savenotes}

\sphinxAtStartPar
where t\_resx = SRT \sphinxhyphen{} HRT

{
\sphinxsetup{VerbatimColor={named}{nbsphinx-code-bg}}
\sphinxsetup{VerbatimBorderColor={named}{nbsphinx-code-border}}
\begin{sphinxVerbatim}[commandchars=\\\{\}]
\llap{\color{nbsphinxin}[4]:\,\hspace{\fboxrule}\hspace{\fboxsep}}\PYG{c+c1}{\PYGZsh{} check if file exsits read from file}
\PYG{c+c1}{\PYGZsh{} otherwise run the simulations with different cations `cat\PYGZus{}test`}
\PYG{n}{cat\PYGZus{}test} \PYG{o}{=} \PYG{p}{[}\PYG{n}{i}\PYG{o}{*}\PYG{l+m+mf}{0.001} \PYG{k}{for} \PYG{n}{i} \PYG{o+ow}{in} \PYG{n+nb}{range}\PYG{p}{(}\PYG{l+m+mi}{200}\PYG{p}{)}\PYG{p}{]}

\PYG{n}{filename}\PYG{o}{=}\PYG{l+s+s1}{\PYGZsq{}}\PYG{l+s+s1}{data/no\PYGZhy{}configuration.dat}\PYG{l+s+s1}{\PYGZsq{}}
\PYG{k}{if} \PYG{n}{adm1fu}\PYG{o}{.}\PYG{n}{check\PYGZus{}filename}\PYG{p}{(}\PYG{n}{filename}\PYG{p}{)}\PYG{p}{:}
    \PYG{n}{ph\PYGZus{}test} \PYG{o}{=} \PYG{n}{np}\PYG{o}{.}\PYG{n}{loadtxt}\PYG{p}{(}\PYG{n}{filename}\PYG{p}{)}
\PYG{k}{else}\PYG{p}{:}
    \PYG{n}{ph\PYGZus{}test} \PYG{o}{=} \PYG{p}{[}\PYG{n}{adm1fu}\PYG{o}{.}\PYG{n}{ph}\PYG{p}{(}\PYG{n}{i}\PYG{p}{,}\PYG{n}{verbose}\PYG{o}{=}\PYG{l+s+s1}{\PYGZsq{}}\PYG{l+s+s1}{off}\PYG{l+s+s1}{\PYGZsq{}}\PYG{p}{,}\PYG{o}{*}\PYG{o}{*}\PYG{n}{config\PYGZus{}default}\PYG{p}{)}\PYG{p}{[}\PYG{l+m+mi}{0}\PYG{p}{]} \PYG{k}{for} \PYG{n}{i} \PYG{o+ow}{in} \PYG{n}{cat\PYGZus{}test}\PYG{p}{]}
    \PYG{n}{np}\PYG{o}{.}\PYG{n}{savetxt}\PYG{p}{(}\PYG{n}{filename}\PYG{p}{,} \PYG{n}{ph\PYGZus{}test}\PYG{p}{,} \PYG{n}{fmt}\PYG{o}{=}\PYG{l+s+s1}{\PYGZsq{}}\PYG{l+s+si}{\PYGZpc{}5.6f}\PYG{l+s+s1}{\PYGZsq{}}\PYG{p}{)}
\end{sphinxVerbatim}
}

\sphinxAtStartPar
\sphinxstylestrong{Relation b/t cation and Ph under Default config}

{
\sphinxsetup{VerbatimColor={named}{nbsphinx-code-bg}}
\sphinxsetup{VerbatimBorderColor={named}{nbsphinx-code-border}}
\begin{sphinxVerbatim}[commandchars=\\\{\}]
\llap{\color{nbsphinxin}[5]:\,\hspace{\fboxrule}\hspace{\fboxsep}}\PYG{c+c1}{\PYGZsh{} Relation b/t cation and Ph under Default config}
\PYG{n}{plt}\PYG{o}{.}\PYG{n}{figure}\PYG{p}{(}\PYG{n}{figsize}\PYG{o}{=}\PYG{p}{(}\PYG{l+m+mi}{12}\PYG{p}{,}\PYG{l+m+mi}{5}\PYG{p}{)}\PYG{p}{)}
\PYG{n}{plt}\PYG{o}{.}\PYG{n}{scatter}\PYG{p}{(}\PYG{n}{cat\PYGZus{}test}\PYG{p}{,} \PYG{n}{ph\PYGZus{}test}\PYG{p}{)}
\PYG{n}{plt}\PYG{o}{.}\PYG{n}{ylabel}\PYG{p}{(}\PYG{l+s+s1}{\PYGZsq{}}\PYG{l+s+s1}{PH}\PYG{l+s+s1}{\PYGZsq{}}\PYG{p}{,}\PYG{n}{fontsize}\PYG{o}{=}\PYG{l+m+mi}{15}\PYG{p}{)}
\PYG{n}{plt}\PYG{o}{.}\PYG{n}{xticks}\PYG{p}{(}\PYG{n}{fontsize}\PYG{o}{=}\PYG{l+m+mi}{15}\PYG{p}{)}
\PYG{n}{plt}\PYG{o}{.}\PYG{n}{yticks}\PYG{p}{(}\PYG{n}{fontsize}\PYG{o}{=}\PYG{l+m+mi}{15}\PYG{p}{)}
\PYG{n}{plt}\PYG{o}{.}\PYG{n}{xlabel}\PYG{p}{(}\PYG{l+s+s1}{\PYGZsq{}}\PYG{l+s+s1}{Cation (kmole/m3)}\PYG{l+s+s1}{\PYGZsq{}}\PYG{p}{,}\PYG{n}{fontsize}\PYG{o}{=}\PYG{l+m+mi}{15}\PYG{p}{)}
\PYG{n}{plt}\PYG{o}{.}\PYG{n}{title}\PYG{p}{(}\PYG{l+s+s2}{\PYGZdq{}}\PYG{l+s+s2}{Default configuration}\PYG{l+s+s2}{\PYGZdq{}}\PYG{p}{,}\PYG{n}{fontsize}\PYG{o}{=}\PYG{l+m+mi}{20}\PYG{p}{)}\PYG{p}{;}
\end{sphinxVerbatim}
}

\hrule height -\fboxrule\relax
\vspace{\nbsphinxcodecellspacing}

\makeatletter\setbox\nbsphinxpromptbox\box\voidb@x\makeatother

\begin{nbsphinxfancyoutput}

\noindent\sphinxincludegraphics[width=733\sphinxpxdimen,height=348\sphinxpxdimen]{{jupyter_notebook_ph-control_8_0}.png}

\end{nbsphinxfancyoutput}

\sphinxAtStartPar
\sphinxstylestrong{Relation b/t cation and Ph under Phase 1 config}

{
\sphinxsetup{VerbatimColor={named}{nbsphinx-code-bg}}
\sphinxsetup{VerbatimBorderColor={named}{nbsphinx-code-border}}
\begin{sphinxVerbatim}[commandchars=\\\{\}]
\llap{\color{nbsphinxin}[6]:\,\hspace{\fboxrule}\hspace{\fboxsep}}\PYG{n}{filename}\PYG{o}{=}\PYG{l+s+s1}{\PYGZsq{}}\PYG{l+s+s1}{data/configuration1.dat}\PYG{l+s+s1}{\PYGZsq{}}
\PYG{k}{if} \PYG{n}{adm1fu}\PYG{o}{.}\PYG{n}{check\PYGZus{}filename}\PYG{p}{(}\PYG{n}{filename}\PYG{p}{)}\PYG{p}{:}
    \PYG{n}{ph\PYGZus{}test\PYGZus{}config1} \PYG{o}{=} \PYG{n}{np}\PYG{o}{.}\PYG{n}{loadtxt}\PYG{p}{(}\PYG{n}{filename}\PYG{p}{)}
\PYG{k}{else}\PYG{p}{:}
    \PYG{n}{ph\PYGZus{}test\PYGZus{}config1} \PYG{o}{=} \PYG{p}{[}\PYG{n}{adm1fu}\PYG{o}{.}\PYG{n}{ph}\PYG{p}{(}\PYG{n}{i}\PYG{p}{,} \PYG{n}{verbose}\PYG{o}{=}\PYG{l+s+s1}{\PYGZsq{}}\PYG{l+s+s1}{off}\PYG{l+s+s1}{\PYGZsq{}}\PYG{p}{,} \PYG{o}{*}\PYG{o}{*}\PYG{n}{config1}\PYG{p}{)}\PYG{p}{[}\PYG{l+m+mi}{0}\PYG{p}{]} \PYG{k}{for} \PYG{n}{i} \PYG{o+ow}{in} \PYG{n}{cat\PYGZus{}test}\PYG{p}{]}
    \PYG{n}{np}\PYG{o}{.}\PYG{n}{savetxt}\PYG{p}{(}\PYG{n}{filename}\PYG{p}{,} \PYG{n}{ph\PYGZus{}test\PYGZus{}config1}\PYG{p}{,} \PYG{n}{fmt}\PYG{o}{=}\PYG{l+s+s1}{\PYGZsq{}}\PYG{l+s+si}{\PYGZpc{}5.6f}\PYG{l+s+s1}{\PYGZsq{}}\PYG{p}{)}
\end{sphinxVerbatim}
}

{
\sphinxsetup{VerbatimColor={named}{nbsphinx-code-bg}}
\sphinxsetup{VerbatimBorderColor={named}{nbsphinx-code-border}}
\begin{sphinxVerbatim}[commandchars=\\\{\}]
\llap{\color{nbsphinxin}[7]:\,\hspace{\fboxrule}\hspace{\fboxsep}}\PYG{n}{plt}\PYG{o}{.}\PYG{n}{figure}\PYG{p}{(}\PYG{n}{figsize}\PYG{o}{=}\PYG{p}{(}\PYG{l+m+mi}{12}\PYG{p}{,}\PYG{l+m+mi}{5}\PYG{p}{)}\PYG{p}{)}
\PYG{n}{plt}\PYG{o}{.}\PYG{n}{scatter}\PYG{p}{(}\PYG{n}{cat\PYGZus{}test}\PYG{p}{,} \PYG{n}{ph\PYGZus{}test\PYGZus{}config1}\PYG{p}{)}
\PYG{n}{plt}\PYG{o}{.}\PYG{n}{ylabel}\PYG{p}{(}\PYG{l+s+s1}{\PYGZsq{}}\PYG{l+s+s1}{PH}\PYG{l+s+s1}{\PYGZsq{}}\PYG{p}{,}\PYG{n}{fontsize}\PYG{o}{=}\PYG{l+m+mi}{15}\PYG{p}{)}
\PYG{n}{plt}\PYG{o}{.}\PYG{n}{xticks}\PYG{p}{(}\PYG{n}{fontsize}\PYG{o}{=}\PYG{l+m+mi}{15}\PYG{p}{)}
\PYG{n}{plt}\PYG{o}{.}\PYG{n}{yticks}\PYG{p}{(}\PYG{n}{fontsize}\PYG{o}{=}\PYG{l+m+mi}{15}\PYG{p}{)}
\PYG{n}{plt}\PYG{o}{.}\PYG{n}{xlabel}\PYG{p}{(}\PYG{l+s+s1}{\PYGZsq{}}\PYG{l+s+s1}{Cation (kmole/m3)}\PYG{l+s+s1}{\PYGZsq{}}\PYG{p}{,}\PYG{n}{fontsize}\PYG{o}{=}\PYG{l+m+mi}{15}\PYG{p}{)}
\PYG{n}{plt}\PYG{o}{.}\PYG{n}{title}\PYG{p}{(}\PYG{l+s+s2}{\PYGZdq{}}\PYG{l+s+s2}{Phase 1 Configuration with only cation being changed}\PYG{l+s+s2}{\PYGZdq{}}\PYG{p}{,}\PYG{n}{fontsize}\PYG{o}{=}\PYG{l+m+mi}{20}\PYG{p}{)}\PYG{p}{;}
\end{sphinxVerbatim}
}

\hrule height -\fboxrule\relax
\vspace{\nbsphinxcodecellspacing}

\makeatletter\setbox\nbsphinxpromptbox\box\voidb@x\makeatother

\begin{nbsphinxfancyoutput}

\noindent\sphinxincludegraphics[width=733\sphinxpxdimen,height=348\sphinxpxdimen]{{jupyter_notebook_ph-control_11_0}.png}

\end{nbsphinxfancyoutput}

\sphinxAtStartPar
\sphinxstylestrong{Relation b/t cation and Ph under Phase 2 config}

{
\sphinxsetup{VerbatimColor={named}{nbsphinx-code-bg}}
\sphinxsetup{VerbatimBorderColor={named}{nbsphinx-code-border}}
\begin{sphinxVerbatim}[commandchars=\\\{\}]
\llap{\color{nbsphinxin}[8]:\,\hspace{\fboxrule}\hspace{\fboxsep}}\PYG{n}{filename}\PYG{o}{=}\PYG{l+s+s1}{\PYGZsq{}}\PYG{l+s+s1}{data/configuration2.dat}\PYG{l+s+s1}{\PYGZsq{}}
\PYG{k}{if} \PYG{n}{adm1fu}\PYG{o}{.}\PYG{n}{check\PYGZus{}filename}\PYG{p}{(}\PYG{n}{filename}\PYG{p}{)}\PYG{p}{:}
    \PYG{n}{ph\PYGZus{}test\PYGZus{}config2} \PYG{o}{=} \PYG{n}{np}\PYG{o}{.}\PYG{n}{loadtxt}\PYG{p}{(}\PYG{n}{filename}\PYG{p}{)}
\PYG{k}{else}\PYG{p}{:}
    \PYG{n}{ph\PYGZus{}test\PYGZus{}config2} \PYG{o}{=} \PYG{p}{[}\PYG{n}{adm1fu}\PYG{o}{.}\PYG{n}{ph}\PYG{p}{(}\PYG{n}{i}\PYG{p}{,} \PYG{n}{verbose}\PYG{o}{=}\PYG{l+s+s1}{\PYGZsq{}}\PYG{l+s+s1}{off}\PYG{l+s+s1}{\PYGZsq{}}\PYG{p}{,} \PYG{o}{*}\PYG{o}{*}\PYG{n}{config2}\PYG{p}{)}\PYG{p}{[}\PYG{l+m+mi}{0}\PYG{p}{]} \PYG{k}{for} \PYG{n}{i} \PYG{o+ow}{in} \PYG{n}{cat\PYGZus{}test}\PYG{p}{]}
    \PYG{n}{np}\PYG{o}{.}\PYG{n}{savetxt}\PYG{p}{(}\PYG{n}{filename}\PYG{p}{,} \PYG{n}{ph\PYGZus{}test\PYGZus{}config2}\PYG{p}{,} \PYG{n}{fmt}\PYG{o}{=}\PYG{l+s+s1}{\PYGZsq{}}\PYG{l+s+si}{\PYGZpc{}5.6f}\PYG{l+s+s1}{\PYGZsq{}}\PYG{p}{)}
\end{sphinxVerbatim}
}

{
\sphinxsetup{VerbatimColor={named}{nbsphinx-code-bg}}
\sphinxsetup{VerbatimBorderColor={named}{nbsphinx-code-border}}
\begin{sphinxVerbatim}[commandchars=\\\{\}]
\llap{\color{nbsphinxin}[9]:\,\hspace{\fboxrule}\hspace{\fboxsep}}\PYG{n}{plt}\PYG{o}{.}\PYG{n}{figure}\PYG{p}{(}\PYG{n}{figsize}\PYG{o}{=}\PYG{p}{(}\PYG{l+m+mi}{12}\PYG{p}{,}\PYG{l+m+mi}{5}\PYG{p}{)}\PYG{p}{)}
\PYG{n}{plt}\PYG{o}{.}\PYG{n}{scatter}\PYG{p}{(}\PYG{n}{cat\PYGZus{}test}\PYG{p}{,} \PYG{n}{ph\PYGZus{}test\PYGZus{}config2}\PYG{p}{)}
\PYG{n}{plt}\PYG{o}{.}\PYG{n}{xticks}\PYG{p}{(}\PYG{n}{fontsize}\PYG{o}{=}\PYG{l+m+mi}{15}\PYG{p}{)}
\PYG{n}{plt}\PYG{o}{.}\PYG{n}{yticks}\PYG{p}{(}\PYG{n}{fontsize}\PYG{o}{=}\PYG{l+m+mi}{15}\PYG{p}{)}
\PYG{n}{plt}\PYG{o}{.}\PYG{n}{ylabel}\PYG{p}{(}\PYG{l+s+s1}{\PYGZsq{}}\PYG{l+s+s1}{PH}\PYG{l+s+s1}{\PYGZsq{}}\PYG{p}{,}\PYG{n}{fontsize}\PYG{o}{=}\PYG{l+m+mi}{15}\PYG{p}{)}
\PYG{n}{plt}\PYG{o}{.}\PYG{n}{xlabel}\PYG{p}{(}\PYG{l+s+s1}{\PYGZsq{}}\PYG{l+s+s1}{Cation (kmole/m3)}\PYG{l+s+s1}{\PYGZsq{}}\PYG{p}{,}\PYG{n}{fontsize}\PYG{o}{=}\PYG{l+m+mi}{15}\PYG{p}{)}
\PYG{n}{plt}\PYG{o}{.}\PYG{n}{title}\PYG{p}{(}\PYG{l+s+s2}{\PYGZdq{}}\PYG{l+s+s2}{Phase 2 Configuration 2 with only cation being changed}\PYG{l+s+s2}{\PYGZdq{}}\PYG{p}{,}\PYG{n}{fontsize}\PYG{o}{=}\PYG{l+m+mi}{20}\PYG{p}{)}\PYG{p}{;}
\end{sphinxVerbatim}
}

\hrule height -\fboxrule\relax
\vspace{\nbsphinxcodecellspacing}

\makeatletter\setbox\nbsphinxpromptbox\box\voidb@x\makeatother

\begin{nbsphinxfancyoutput}

\noindent\sphinxincludegraphics[width=733\sphinxpxdimen,height=348\sphinxpxdimen]{{jupyter_notebook_ph-control_14_0}.png}

\end{nbsphinxfancyoutput}


\subsubsection{2. PH: one\sphinxhyphen{}phase reactor}
\label{\detokenize{jupyter_notebook/ph-control:2.-PH:-one-phase-reactor}}
{
\sphinxsetup{VerbatimColor={named}{nbsphinx-code-bg}}
\sphinxsetup{VerbatimBorderColor={named}{nbsphinx-code-border}}
\begin{sphinxVerbatim}[commandchars=\\\{\}]
\llap{\color{nbsphinxin}[10]:\,\hspace{\fboxrule}\hspace{\fboxsep}}\PYG{n}{old\PYGZus{}ph0} \PYG{o}{=} \PYG{n}{adm1fu}\PYG{o}{.}\PYG{n}{ph}\PYG{p}{(}\PYG{l+m+mi}{0}\PYG{p}{)}
\PYG{n+nb}{print}\PYG{p}{(}\PYG{l+s+s1}{\PYGZsq{}}\PYG{l+s+s1}{Predicted PH is }\PYG{l+s+si}{\PYGZob{}\PYGZcb{}}\PYG{l+s+s1}{ if using the original cation value }\PYG{l+s+si}{\PYGZob{}\PYGZcb{}}\PYG{l+s+s1}{\PYGZsq{}}\PYG{o}{.}\PYG{n}{format}\PYG{p}{(}\PYG{n}{old\PYGZus{}ph0}\PYG{p}{[}\PYG{l+m+mi}{0}\PYG{p}{]}\PYG{p}{,} \PYG{n}{old\PYGZus{}ph0}\PYG{p}{[}\PYG{l+m+mi}{1}\PYG{p}{]}\PYG{p}{)}\PYG{p}{)}
\end{sphinxVerbatim}
}

{

\kern-\sphinxverbatimsmallskipamount\kern-\baselineskip
\kern+\FrameHeightAdjust\kern-\fboxrule
\vspace{\nbsphinxcodecellspacing}

\sphinxsetup{VerbatimColor={named}{white}}
\sphinxsetup{VerbatimBorderColor={named}{nbsphinx-code-border}}
\begin{sphinxVerbatim}[commandchars=\\\{\}]
Reactor run, ph phase-one:
\$ADM1F\_EXE -steady -influent\_file influent\_cur.dat
Predicted PH is 6.91367 if using the original cation value 0
\end{sphinxVerbatim}
}

{
\sphinxsetup{VerbatimColor={named}{nbsphinx-code-bg}}
\sphinxsetup{VerbatimBorderColor={named}{nbsphinx-code-border}}
\begin{sphinxVerbatim}[commandchars=\\\{\}]
\llap{\color{nbsphinxin}[11]:\,\hspace{\fboxrule}\hspace{\fboxsep}}\PYG{n}{old\PYGZus{}ph1} \PYG{o}{=} \PYG{n}{adm1fu}\PYG{o}{.}\PYG{n}{ph}\PYG{p}{(}\PYG{l+m+mi}{0}\PYG{p}{,} \PYG{o}{*}\PYG{o}{*}\PYG{n}{config1}\PYG{p}{)}
\PYG{n+nb}{print}\PYG{p}{(}\PYG{l+s+s1}{\PYGZsq{}}\PYG{l+s+s1}{Predicted PH is }\PYG{l+s+si}{\PYGZob{}\PYGZcb{}}\PYG{l+s+s1}{ if using the original cation value }\PYG{l+s+si}{\PYGZob{}\PYGZcb{}}\PYG{l+s+s1}{\PYGZsq{}}\PYG{o}{.}\PYG{n}{format}\PYG{p}{(}\PYG{n}{old\PYGZus{}ph1}\PYG{p}{[}\PYG{l+m+mi}{0}\PYG{p}{]}\PYG{p}{,} \PYG{n}{old\PYGZus{}ph1}\PYG{p}{[}\PYG{l+m+mi}{1}\PYG{p}{]}\PYG{p}{)}\PYG{p}{)}
\end{sphinxVerbatim}
}

{

\kern-\sphinxverbatimsmallskipamount\kern-\baselineskip
\kern+\FrameHeightAdjust\kern-\fboxrule
\vspace{\nbsphinxcodecellspacing}

\sphinxsetup{VerbatimColor={named}{white}}
\sphinxsetup{VerbatimBorderColor={named}{nbsphinx-code-border}}
\begin{sphinxVerbatim}[commandchars=\\\{\}]
Reactor run, ph phase-one:
\$ADM1F\_EXE -steady -influent\_file influent\_cur.dat -Vliq 340 -t\_resx 1.5
Predicted PH is 4.57781 if using the original cation value 0
\end{sphinxVerbatim}
}

{
\sphinxsetup{VerbatimColor={named}{nbsphinx-code-bg}}
\sphinxsetup{VerbatimBorderColor={named}{nbsphinx-code-border}}
\begin{sphinxVerbatim}[commandchars=\\\{\}]
\llap{\color{nbsphinxin}[12]:\,\hspace{\fboxrule}\hspace{\fboxsep}}\PYG{n}{old\PYGZus{}ph2} \PYG{o}{=} \PYG{n}{adm1fu}\PYG{o}{.}\PYG{n}{ph}\PYG{p}{(}\PYG{l+m+mi}{0}\PYG{p}{,} \PYG{o}{*}\PYG{o}{*}\PYG{n}{config2}\PYG{p}{)}
\PYG{n+nb}{print}\PYG{p}{(}\PYG{l+s+s1}{\PYGZsq{}}\PYG{l+s+s1}{Predicted PH is }\PYG{l+s+si}{\PYGZob{}\PYGZcb{}}\PYG{l+s+s1}{ if using the original cation value }\PYG{l+s+si}{\PYGZob{}\PYGZcb{}}\PYG{l+s+s1}{\PYGZsq{}}\PYG{o}{.}\PYG{n}{format}\PYG{p}{(}\PYG{n}{old\PYGZus{}ph2}\PYG{p}{[}\PYG{l+m+mi}{0}\PYG{p}{]}\PYG{p}{,} \PYG{n}{old\PYGZus{}ph2}\PYG{p}{[}\PYG{l+m+mi}{1}\PYG{p}{]}\PYG{p}{)}\PYG{p}{)}
\end{sphinxVerbatim}
}

{

\kern-\sphinxverbatimsmallskipamount\kern-\baselineskip
\kern+\FrameHeightAdjust\kern-\fboxrule
\vspace{\nbsphinxcodecellspacing}

\sphinxsetup{VerbatimColor={named}{white}}
\sphinxsetup{VerbatimBorderColor={named}{nbsphinx-code-border}}
\begin{sphinxVerbatim}[commandchars=\\\{\}]
Reactor run, ph phase-one:
\$ADM1F\_EXE -steady -influent\_file influent\_cur.dat -Vliq 3400 -t\_resx 700
Predicted PH is 7.05983 if using the original cation value 0
\end{sphinxVerbatim}
}

\sphinxAtStartPar
\sphinxstylestrong{Set Target:} Let’s calculate the amount of cation needed by the one\sphinxhyphen{}phase reactor to match required \sphinxcode{\sphinxupquote{ph}} targets using Data Consistent Inversion method.

{
\sphinxsetup{VerbatimColor={named}{nbsphinx-code-bg}}
\sphinxsetup{VerbatimBorderColor={named}{nbsphinx-code-border}}
\begin{sphinxVerbatim}[commandchars=\\\{\}]
\llap{\color{nbsphinxin}[13]:\,\hspace{\fboxrule}\hspace{\fboxsep}}\PYG{k}{class} \PYG{n+nc}{target}\PYG{p}{:}
    \PYG{k}{def} \PYG{n+nf+fm}{\PYGZus{}\PYGZus{}init\PYGZus{}\PYGZus{}}\PYG{p}{(}\PYG{n+nb+bp}{self}\PYG{p}{,}\PYG{n}{ph}\PYG{p}{,}\PYG{n}{sig}\PYG{p}{)}\PYG{p}{:}
        \PYG{n+nb+bp}{self}\PYG{o}{.}\PYG{n}{ph} \PYG{o}{=} \PYG{n}{ph}
        \PYG{n+nb+bp}{self}\PYG{o}{.}\PYG{n}{sig} \PYG{o}{=} \PYG{n}{sig}
    \PYG{k}{def} \PYG{n+nf}{pdf}\PYG{p}{(}\PYG{n+nb+bp}{self}\PYG{p}{,}\PYG{n}{x}\PYG{p}{)}\PYG{p}{:}
        \PYG{k}{return} \PYG{n}{norm}\PYG{o}{.}\PYG{n}{pdf}\PYG{p}{(}\PYG{n}{x}\PYG{p}{,}\PYG{n+nb+bp}{self}\PYG{o}{.}\PYG{n}{ph}\PYG{p}{,}\PYG{n+nb+bp}{self}\PYG{o}{.}\PYG{n}{sig}\PYG{p}{)}
\end{sphinxVerbatim}
}

{
\sphinxsetup{VerbatimColor={named}{nbsphinx-code-bg}}
\sphinxsetup{VerbatimBorderColor={named}{nbsphinx-code-border}}
\begin{sphinxVerbatim}[commandchars=\\\{\}]
\llap{\color{nbsphinxin}[14]:\,\hspace{\fboxrule}\hspace{\fboxsep}}\PYG{c+c1}{\PYGZsh{} Give the necessary information}
\PYG{c+c1}{\PYGZsh{} target\PYGZus{}ph = 6.5       \PYGZsh{} target\PYGZus{}ph: target PH value, target\PYGZus{}sig: allow some variations around target PH}
\PYG{n}{target\PYGZus{}sig} \PYG{o}{=} \PYG{l+m+mf}{0.01}     \PYG{c+c1}{\PYGZsh{} The smaller this value is, the more accurate we will get in the end}
\PYG{n}{sample\PYGZus{}size} \PYG{o}{=} \PYG{l+m+mi}{100}

\PYG{n}{infl\PYGZus{}path} \PYG{o}{=} \PYG{l+s+s1}{\PYGZsq{}}\PYG{l+s+s1}{influent.dat}\PYG{l+s+s1}{\PYGZsq{}}
\PYG{n}{params\PYGZus{}path} \PYG{o}{=} \PYG{l+s+s1}{\PYGZsq{}}\PYG{l+s+s1}{params.dat}\PYG{l+s+s1}{\PYGZsq{}}
\PYG{n}{ic\PYGZus{}path} \PYG{o}{=} \PYG{l+s+s1}{\PYGZsq{}}\PYG{l+s+s1}{ic.dat}\PYG{l+s+s1}{\PYGZsq{}}

\PYG{c+c1}{\PYGZsh{}\PYGZsh{} Use data consistent inversion method to return the needed cation to get the target PH}
\PYG{n}{init\PYGZus{}sample} \PYG{o}{=} \PYG{n}{np}\PYG{o}{.}\PYG{n}{random}\PYG{o}{.}\PYG{n}{uniform}\PYG{p}{(}\PYG{l+m+mi}{0}\PYG{p}{,}\PYG{l+m+mf}{0.2}\PYG{p}{,}\PYG{n}{sample\PYGZus{}size}\PYG{p}{)}   \PYG{c+c1}{\PYGZsh{}the more samples we generate, the more accurate we will get in the end}
\end{sphinxVerbatim}
}

{
\sphinxsetup{VerbatimColor={named}{nbsphinx-code-bg}}
\sphinxsetup{VerbatimBorderColor={named}{nbsphinx-code-border}}
\begin{sphinxVerbatim}[commandchars=\\\{\}]
\llap{\color{nbsphinxin}[15]:\,\hspace{\fboxrule}\hspace{\fboxsep}}\PYG{n}{target72} \PYG{o}{=} \PYG{n}{target}\PYG{p}{(}\PYG{l+m+mf}{7.2}\PYG{p}{,}\PYG{n}{target\PYGZus{}sig}\PYG{p}{)}
\PYG{n}{target73} \PYG{o}{=} \PYG{n}{target}\PYG{p}{(}\PYG{l+m+mf}{7.3}\PYG{p}{,}\PYG{n}{target\PYGZus{}sig}\PYG{p}{)}
\PYG{n}{target75} \PYG{o}{=} \PYG{n}{target}\PYG{p}{(}\PYG{l+m+mf}{7.5}\PYG{p}{,}\PYG{n}{target\PYGZus{}sig}\PYG{p}{)}
\end{sphinxVerbatim}
}

\sphinxAtStartPar
\sphinxstylestrong{Target 1: target\_ph=7.2 with Default configuration}

{
\sphinxsetup{VerbatimColor={named}{nbsphinx-code-bg}}
\sphinxsetup{VerbatimBorderColor={named}{nbsphinx-code-border}}
\begin{sphinxVerbatim}[commandchars=\\\{\}]
\llap{\color{nbsphinxin}[16]:\,\hspace{\fboxrule}\hspace{\fboxsep}}\PYG{c+c1}{\PYGZsh{}\PYGZsh{} ph\PYGZus{}control accepts target, initial sample, number of cation values and file path of each input file}
\PYG{c+c1}{\PYGZsh{}\PYGZsh{} ph\PYGZus{}control return the needed cation to get the target PH}
\PYG{n}{cat\PYGZus{}tar72\PYGZus{}dc} \PYG{o}{=} \PYG{n}{adm1fu}\PYG{o}{.}\PYG{n}{ph\PYGZus{}control}\PYG{p}{(}\PYG{n}{target72}\PYG{p}{,}\PYG{n}{init\PYGZus{}sample}\PYG{p}{,}\PYG{l+m+mi}{1}\PYG{p}{,}\PYG{n}{infl\PYGZus{}path}\PYG{p}{,}\PYG{n}{params\PYGZus{}path}\PYG{p}{,}\PYG{n}{ic\PYGZus{}path}\PYG{p}{,}\PYG{n}{verbose}\PYG{o}{=}\PYG{l+s+s1}{\PYGZsq{}}\PYG{l+s+s1}{off}\PYG{l+s+s1}{\PYGZsq{}}\PYG{p}{,} \PYG{o}{*}\PYG{o}{*}\PYG{n}{config\PYGZus{}default}\PYG{p}{)}

\PYG{c+c1}{\PYGZsh{} Print out the Needed Cation value!!}
\PYG{n+nb}{print}\PYG{p}{(}\PYG{l+s+s1}{\PYGZsq{}}\PYG{l+s+s1}{The amount of cation in the reactor should be:}\PYG{l+s+s1}{\PYGZsq{}}\PYG{p}{,} \PYG{n}{cat\PYGZus{}tar72\PYGZus{}dc}\PYG{p}{[}\PYG{l+m+mi}{0}\PYG{p}{]}\PYG{p}{,} \PYG{l+s+s1}{\PYGZsq{}}\PYG{l+s+s1}{kmole/m3}\PYG{l+s+s1}{\PYGZsq{}}\PYG{p}{)}
\end{sphinxVerbatim}
}

{

\kern-\sphinxverbatimsmallskipamount\kern-\baselineskip
\kern+\FrameHeightAdjust\kern-\fboxrule
\vspace{\nbsphinxcodecellspacing}

\sphinxsetup{VerbatimColor={named}{white}}
\sphinxsetup{VerbatimBorderColor={named}{nbsphinx-code-border}}
\begin{sphinxVerbatim}[commandchars=\\\{\}]
Predicted PH is 6.91367
The amount of cation in the reactor should be: 0.07135066533871785 kmole/m3
\end{sphinxVerbatim}
}

{
\sphinxsetup{VerbatimColor={named}{nbsphinx-code-bg}}
\sphinxsetup{VerbatimBorderColor={named}{nbsphinx-code-border}}
\begin{sphinxVerbatim}[commandchars=\\\{\}]
\llap{\color{nbsphinxin}[17]:\,\hspace{\fboxrule}\hspace{\fboxsep}}\PYG{p}{[}\PYG{n}{adm1fu}\PYG{o}{.}\PYG{n}{ph}\PYG{p}{(}\PYG{n}{i}\PYG{p}{,} \PYG{o}{*}\PYG{o}{*}\PYG{n}{config\PYGZus{}default}\PYG{p}{)} \PYG{k}{for} \PYG{n}{i} \PYG{o+ow}{in} \PYG{n}{cat\PYGZus{}tar72\PYGZus{}dc}\PYG{p}{]}
\end{sphinxVerbatim}
}

{

\kern-\sphinxverbatimsmallskipamount\kern-\baselineskip
\kern+\FrameHeightAdjust\kern-\fboxrule
\vspace{\nbsphinxcodecellspacing}

\sphinxsetup{VerbatimColor={named}{white}}
\sphinxsetup{VerbatimBorderColor={named}{nbsphinx-code-border}}
\begin{sphinxVerbatim}[commandchars=\\\{\}]
Reactor run, ph phase-one:
\$ADM1F\_EXE -steady -influent\_file influent\_cur.dat -Vliq 3400 -t\_resx 0
\end{sphinxVerbatim}
}

{

\kern-\sphinxverbatimsmallskipamount\kern-\baselineskip
\kern+\FrameHeightAdjust\kern-\fboxrule
\vspace{\nbsphinxcodecellspacing}

\sphinxsetup{VerbatimColor={named}{white}}
\sphinxsetup{VerbatimBorderColor={named}{nbsphinx-code-border}}
\begin{sphinxVerbatim}[commandchars=\\\{\}]
\llap{\color{nbsphinxout}[17]:\,\hspace{\fboxrule}\hspace{\fboxsep}}[(7.18975, 0.07135066533871785)]
\end{sphinxVerbatim}
}

\sphinxAtStartPar
\sphinxstylestrong{Target 2: target\_ph=7.2 with configuration 1}

{
\sphinxsetup{VerbatimColor={named}{nbsphinx-code-bg}}
\sphinxsetup{VerbatimBorderColor={named}{nbsphinx-code-border}}
\begin{sphinxVerbatim}[commandchars=\\\{\}]
\llap{\color{nbsphinxin}[18]:\,\hspace{\fboxrule}\hspace{\fboxsep}}\PYG{c+c1}{\PYGZsh{}\PYGZsh{} ph\PYGZus{}control accepts target, initial sample, number of cation values and file path of each input file}
\PYG{c+c1}{\PYGZsh{}\PYGZsh{} pph\PYGZus{}control return the needed cation to get the target PH}
\PYG{n}{cat\PYGZus{}tar72\PYGZus{}c1} \PYG{o}{=} \PYG{n}{adm1fu}\PYG{o}{.}\PYG{n}{ph\PYGZus{}control}\PYG{p}{(}\PYG{n}{target72}\PYG{p}{,}\PYG{n}{init\PYGZus{}sample}\PYG{p}{,}\PYG{l+m+mi}{1}\PYG{p}{,}\PYG{n}{infl\PYGZus{}path}\PYG{p}{,}\PYG{n}{params\PYGZus{}path}\PYG{p}{,}\PYG{n}{ic\PYGZus{}path}\PYG{p}{,}\PYG{n}{verbose}\PYG{o}{=}\PYG{l+s+s1}{\PYGZsq{}}\PYG{l+s+s1}{off}\PYG{l+s+s1}{\PYGZsq{}}\PYG{p}{,} \PYG{o}{*}\PYG{o}{*}\PYG{n}{config1}\PYG{p}{)}

\PYG{c+c1}{\PYGZsh{} Print out the Needed Cation value!!}
\PYG{n+nb}{print}\PYG{p}{(}\PYG{l+s+s1}{\PYGZsq{}}\PYG{l+s+s1}{The amount of cation in the reactor should be:}\PYG{l+s+s1}{\PYGZsq{}}\PYG{p}{,} \PYG{n}{cat\PYGZus{}tar72\PYGZus{}c1}\PYG{p}{[}\PYG{l+m+mi}{0}\PYG{p}{]}\PYG{p}{,} \PYG{l+s+s1}{\PYGZsq{}}\PYG{l+s+s1}{kmole/m3}\PYG{l+s+s1}{\PYGZsq{}}\PYG{p}{)}
\end{sphinxVerbatim}
}

{

\kern-\sphinxverbatimsmallskipamount\kern-\baselineskip
\kern+\FrameHeightAdjust\kern-\fboxrule
\vspace{\nbsphinxcodecellspacing}

\sphinxsetup{VerbatimColor={named}{white}}
\sphinxsetup{VerbatimBorderColor={named}{nbsphinx-code-border}}
\begin{sphinxVerbatim}[commandchars=\\\{\}]
Predicted PH is 4.57781
The amount of cation in the reactor should be: 0.19737738732010346 kmole/m3
\end{sphinxVerbatim}
}

{
\sphinxsetup{VerbatimColor={named}{nbsphinx-code-bg}}
\sphinxsetup{VerbatimBorderColor={named}{nbsphinx-code-border}}
\begin{sphinxVerbatim}[commandchars=\\\{\}]
\llap{\color{nbsphinxin}[19]:\,\hspace{\fboxrule}\hspace{\fboxsep}}\PYG{p}{[}\PYG{n}{adm1fu}\PYG{o}{.}\PYG{n}{ph}\PYG{p}{(}\PYG{n}{i}\PYG{p}{,} \PYG{o}{*}\PYG{o}{*}\PYG{n}{config1}\PYG{p}{)} \PYG{k}{for} \PYG{n}{i} \PYG{o+ow}{in} \PYG{n}{cat\PYGZus{}tar72\PYGZus{}c1}\PYG{p}{]}
\end{sphinxVerbatim}
}

{

\kern-\sphinxverbatimsmallskipamount\kern-\baselineskip
\kern+\FrameHeightAdjust\kern-\fboxrule
\vspace{\nbsphinxcodecellspacing}

\sphinxsetup{VerbatimColor={named}{white}}
\sphinxsetup{VerbatimBorderColor={named}{nbsphinx-code-border}}
\begin{sphinxVerbatim}[commandchars=\\\{\}]
Reactor run, ph phase-one:
\$ADM1F\_EXE -steady -influent\_file influent\_cur.dat -Vliq 340 -t\_resx 1.5
\end{sphinxVerbatim}
}

{

\kern-\sphinxverbatimsmallskipamount\kern-\baselineskip
\kern+\FrameHeightAdjust\kern-\fboxrule
\vspace{\nbsphinxcodecellspacing}

\sphinxsetup{VerbatimColor={named}{white}}
\sphinxsetup{VerbatimBorderColor={named}{nbsphinx-code-border}}
\begin{sphinxVerbatim}[commandchars=\\\{\}]
\llap{\color{nbsphinxout}[19]:\,\hspace{\fboxrule}\hspace{\fboxsep}}[(7.19608, 0.19737738732010346)]
\end{sphinxVerbatim}
}

\sphinxAtStartPar
\sphinxstylestrong{Target 3: target\_ph=7.2 with configuration 2}

{
\sphinxsetup{VerbatimColor={named}{nbsphinx-code-bg}}
\sphinxsetup{VerbatimBorderColor={named}{nbsphinx-code-border}}
\begin{sphinxVerbatim}[commandchars=\\\{\}]
\llap{\color{nbsphinxin}[20]:\,\hspace{\fboxrule}\hspace{\fboxsep}}\PYG{c+c1}{\PYGZsh{}\PYGZsh{} ph\PYGZus{}control accepts target, initial sample, number of cation values and file path of each input file}
\PYG{c+c1}{\PYGZsh{}\PYGZsh{} ph\PYGZus{}control return the needed cation to get the target PH}
\PYG{n}{cat\PYGZus{}tar72\PYGZus{}c2} \PYG{o}{=} \PYG{n}{adm1fu}\PYG{o}{.}\PYG{n}{ph\PYGZus{}control}\PYG{p}{(}\PYG{n}{target72}\PYG{p}{,}\PYG{n}{init\PYGZus{}sample}\PYG{p}{,}\PYG{l+m+mi}{1}\PYG{p}{,}\PYG{n}{infl\PYGZus{}path}\PYG{p}{,}\PYG{n}{params\PYGZus{}path}\PYG{p}{,}\PYG{n}{ic\PYGZus{}path}\PYG{p}{,}\PYG{n}{verbose}\PYG{o}{=}\PYG{l+s+s1}{\PYGZsq{}}\PYG{l+s+s1}{off}\PYG{l+s+s1}{\PYGZsq{}}\PYG{p}{,} \PYG{o}{*}\PYG{o}{*}\PYG{n}{config2}\PYG{p}{)}

\PYG{c+c1}{\PYGZsh{} Print out the Needed Cation value!!}
\PYG{n+nb}{print}\PYG{p}{(}\PYG{l+s+s1}{\PYGZsq{}}\PYG{l+s+s1}{The amount of cation in the reactor should be:}\PYG{l+s+s1}{\PYGZsq{}}\PYG{p}{,} \PYG{n}{cat\PYGZus{}tar72\PYGZus{}c2}\PYG{p}{[}\PYG{l+m+mi}{0}\PYG{p}{]}\PYG{p}{,} \PYG{l+s+s1}{\PYGZsq{}}\PYG{l+s+s1}{kmole/m3}\PYG{l+s+s1}{\PYGZsq{}}\PYG{p}{)}
\end{sphinxVerbatim}
}

{

\kern-\sphinxverbatimsmallskipamount\kern-\baselineskip
\kern+\FrameHeightAdjust\kern-\fboxrule
\vspace{\nbsphinxcodecellspacing}

\sphinxsetup{VerbatimColor={named}{white}}
\sphinxsetup{VerbatimBorderColor={named}{nbsphinx-code-border}}
\begin{sphinxVerbatim}[commandchars=\\\{\}]
Predicted PH is 7.05983
The amount of cation in the reactor should be: 0.031198904067240532 kmole/m3
\end{sphinxVerbatim}
}

{
\sphinxsetup{VerbatimColor={named}{nbsphinx-code-bg}}
\sphinxsetup{VerbatimBorderColor={named}{nbsphinx-code-border}}
\begin{sphinxVerbatim}[commandchars=\\\{\}]
\llap{\color{nbsphinxin}[21]:\,\hspace{\fboxrule}\hspace{\fboxsep}}\PYG{p}{[}\PYG{n}{adm1fu}\PYG{o}{.}\PYG{n}{ph}\PYG{p}{(}\PYG{n}{i}\PYG{p}{,} \PYG{o}{*}\PYG{o}{*}\PYG{n}{config2}\PYG{p}{)} \PYG{k}{for} \PYG{n}{i} \PYG{o+ow}{in} \PYG{n}{cat\PYGZus{}tar72\PYGZus{}c2}\PYG{p}{]}
\end{sphinxVerbatim}
}

{

\kern-\sphinxverbatimsmallskipamount\kern-\baselineskip
\kern+\FrameHeightAdjust\kern-\fboxrule
\vspace{\nbsphinxcodecellspacing}

\sphinxsetup{VerbatimColor={named}{white}}
\sphinxsetup{VerbatimBorderColor={named}{nbsphinx-code-border}}
\begin{sphinxVerbatim}[commandchars=\\\{\}]
Reactor run, ph phase-one:
\$ADM1F\_EXE -steady -influent\_file influent\_cur.dat -Vliq 3400 -t\_resx 700
\end{sphinxVerbatim}
}

{

\kern-\sphinxverbatimsmallskipamount\kern-\baselineskip
\kern+\FrameHeightAdjust\kern-\fboxrule
\vspace{\nbsphinxcodecellspacing}

\sphinxsetup{VerbatimColor={named}{white}}
\sphinxsetup{VerbatimBorderColor={named}{nbsphinx-code-border}}
\begin{sphinxVerbatim}[commandchars=\\\{\}]
\llap{\color{nbsphinxout}[21]:\,\hspace{\fboxrule}\hspace{\fboxsep}}[(7.20101, 0.031198904067240532)]
\end{sphinxVerbatim}
}


\subsubsection{3. PH: two\sphinxhyphen{}phase reactor}
\label{\detokenize{jupyter_notebook/ph-control:3.-PH:-two-phase-reactor}}
\sphinxAtStartPar
\sphinxstylestrong{PH control for both phase 1 and phase 2}

\sphinxAtStartPar
reactor\_cat(target\_1=target1, target\_2=target2, Q1=1, Vliq1=1, t\_resx1=1, Q2=1, Vliq2=1, t\_res2=1)

\sphinxAtStartPar
\sphinxstylestrong{PH control for just phase 1 in two\sphinxhyphen{}phase reactor}

\sphinxAtStartPar
reactor\_cat(target\_1=target1, Q1=1, Vliq1=1, t\_resx1=1, Q2=1, Vliq2=1, t\_res2=1)

{
\sphinxsetup{VerbatimColor={named}{nbsphinx-code-bg}}
\sphinxsetup{VerbatimBorderColor={named}{nbsphinx-code-border}}
\begin{sphinxVerbatim}[commandchars=\\\{\}]
\llap{\color{nbsphinxin}[22]:\,\hspace{\fboxrule}\hspace{\fboxsep}}\PYG{c+c1}{\PYGZsh{}\PYGZsh{} Configuration of two\PYGZhy{}phase reacotr}
\PYG{c+c1}{\PYGZsh{} config12 = \PYGZob{}\PYGZdq{}Vliq1\PYGZdq{}:340, \PYGZdq{}Vliq2\PYGZdq{}:3400, \PYGZdq{}t\PYGZus{}resx1\PYGZdq{}:1.5, \PYGZdq{}t\PYGZus{}resx2\PYGZdq{}:700, \PYGZdq{}Q1\PYGZdq{}:618, \PYGZdq{}Q2\PYGZdq{}:618\PYGZcb{}}
\PYG{n}{config12} \PYG{o}{=} \PYG{p}{\PYGZob{}}\PYG{l+s+s2}{\PYGZdq{}}\PYG{l+s+s2}{Vliq1}\PYG{l+s+s2}{\PYGZdq{}}\PYG{p}{:}\PYG{l+m+mi}{340}\PYG{p}{,} \PYG{l+s+s2}{\PYGZdq{}}\PYG{l+s+s2}{Vliq2}\PYG{l+s+s2}{\PYGZdq{}}\PYG{p}{:}\PYG{l+m+mi}{3400}\PYG{p}{,} \PYG{l+s+s2}{\PYGZdq{}}\PYG{l+s+s2}{t\PYGZus{}resx1}\PYG{l+s+s2}{\PYGZdq{}}\PYG{p}{:}\PYG{l+m+mf}{1.5}\PYG{p}{,} \PYG{l+s+s2}{\PYGZdq{}}\PYG{l+s+s2}{t\PYGZus{}resx2}\PYG{l+s+s2}{\PYGZdq{}}\PYG{p}{:}\PYG{l+m+mi}{700}\PYG{p}{,} \PYG{l+s+s2}{\PYGZdq{}}\PYG{l+s+s2}{Q1}\PYG{l+s+s2}{\PYGZdq{}}\PYG{p}{:}\PYG{l+m+mi}{618}\PYG{p}{\PYGZcb{}}
\end{sphinxVerbatim}
}

\sphinxAtStartPar
\sphinxstylestrong{target\_ph1=7.5, target\_ph2=7.2 with default configuration12}

{
\sphinxsetup{VerbatimColor={named}{nbsphinx-code-bg}}
\sphinxsetup{VerbatimBorderColor={named}{nbsphinx-code-border}}
\begin{sphinxVerbatim}[commandchars=\\\{\}]
\llap{\color{nbsphinxin}[23]:\,\hspace{\fboxrule}\hspace{\fboxsep}}\PYG{n}{config12} \PYG{o}{=} \PYG{p}{\PYGZob{}}\PYG{l+s+s2}{\PYGZdq{}}\PYG{l+s+s2}{Vliq1}\PYG{l+s+s2}{\PYGZdq{}}\PYG{p}{:}\PYG{l+m+mi}{340}\PYG{p}{,} \PYG{l+s+s2}{\PYGZdq{}}\PYG{l+s+s2}{Vliq2}\PYG{l+s+s2}{\PYGZdq{}}\PYG{p}{:}\PYG{l+m+mi}{3400}\PYG{p}{,} \PYG{l+s+s2}{\PYGZdq{}}\PYG{l+s+s2}{t\PYGZus{}resx1}\PYG{l+s+s2}{\PYGZdq{}}\PYG{p}{:}\PYG{l+m+mf}{1.5}\PYG{p}{,} \PYG{l+s+s2}{\PYGZdq{}}\PYG{l+s+s2}{t\PYGZus{}resx2}\PYG{l+s+s2}{\PYGZdq{}}\PYG{p}{:}\PYG{l+m+mi}{700}\PYG{p}{,} \PYG{l+s+s2}{\PYGZdq{}}\PYG{l+s+s2}{Q1}\PYG{l+s+s2}{\PYGZdq{}}\PYG{p}{:}\PYG{l+m+mi}{618}\PYG{p}{\PYGZcb{}}
\PYG{n}{adm1fu}\PYG{o}{.}\PYG{n}{reactor2\PYGZus{}cat}\PYG{p}{(}\PYG{n}{init\PYGZus{}sample}\PYG{p}{,}\PYG{n}{target\PYGZus{}1}\PYG{o}{=}\PYG{n}{target75}\PYG{p}{,}\PYG{n}{target\PYGZus{}2}\PYG{o}{=}\PYG{n}{target72}\PYG{p}{,}\PYG{n}{verbose}\PYG{o}{=}\PYG{l+s+s1}{\PYGZsq{}}\PYG{l+s+s1}{off}\PYG{l+s+s1}{\PYGZsq{}}\PYG{p}{,}\PYG{o}{*}\PYG{o}{*}\PYG{n}{config12}\PYG{p}{)}
\end{sphinxVerbatim}
}

{

\kern-\sphinxverbatimsmallskipamount\kern-\baselineskip
\kern+\FrameHeightAdjust\kern-\fboxrule
\vspace{\nbsphinxcodecellspacing}

\sphinxsetup{VerbatimColor={named}{white}}
\sphinxsetup{VerbatimBorderColor={named}{nbsphinx-code-border}}
\begin{sphinxVerbatim}[commandchars=\\\{\}]
verbose: off
Predicted PH is 4.57781
\$ADM1F\_EXE -steady -Vliq 340 -t\_resx 1.5 -influent\_file influent\_cur.dat
Phase 1, after changing cation to 0.197377 kmole/m3, new PH = 7.19608
Predicted PH is 7.10724

\$ADM1F\_EXE -steady -Vliq 3400 -t\_resx 700 -influent\_file influent\_cur.dat
Phase 2, after changing cation to 0.017699 kmole/m3, new PH = 7.18887
\end{sphinxVerbatim}
}

\sphinxAtStartPar
\sphinxstylestrong{target\_ph1=7.5, target\_ph2=None with default configuration12}

{
\sphinxsetup{VerbatimColor={named}{nbsphinx-code-bg}}
\sphinxsetup{VerbatimBorderColor={named}{nbsphinx-code-border}}
\begin{sphinxVerbatim}[commandchars=\\\{\}]
\llap{\color{nbsphinxin}[24]:\,\hspace{\fboxrule}\hspace{\fboxsep}}\PYG{n}{adm1fu}\PYG{o}{.}\PYG{n}{reactor2\PYGZus{}cat}\PYG{p}{(}\PYG{n}{init\PYGZus{}sample}\PYG{p}{,}\PYG{n}{target\PYGZus{}1}\PYG{o}{=}\PYG{n}{target75}\PYG{p}{,}\PYG{n}{verbose}\PYG{o}{=}\PYG{l+s+s1}{\PYGZsq{}}\PYG{l+s+s1}{off}\PYG{l+s+s1}{\PYGZsq{}}\PYG{p}{,}\PYG{o}{*}\PYG{o}{*}\PYG{n}{config12}\PYG{p}{)}
\end{sphinxVerbatim}
}

{

\kern-\sphinxverbatimsmallskipamount\kern-\baselineskip
\kern+\FrameHeightAdjust\kern-\fboxrule
\vspace{\nbsphinxcodecellspacing}

\sphinxsetup{VerbatimColor={named}{white}}
\sphinxsetup{VerbatimBorderColor={named}{nbsphinx-code-border}}
\begin{sphinxVerbatim}[commandchars=\\\{\}]
verbose: off
Predicted PH is 4.57781
\$ADM1F\_EXE -steady -Vliq 340 -t\_resx 1.5 -influent\_file influent\_cur.dat
Phase 1, after changing cation to 0.197377 kmole/m3, new PH = 7.19608

\$ADM1F\_EXE -steady -Vliq 3400 -t\_resx 700 -influent\_file influent\_cur.dat
Phase 2, without changing cation, predicted PH = 7.65471
\end{sphinxVerbatim}
}
\subsubsection*{References}
\subsubsection*{Footnote}


\chapter{Join Us on Slack}
\label{\detokenize{join:join-us-on-slack}}\label{\detokenize{join::doc}}
\sphinxAtStartPar
If you have any questions or would like to be a developer, feel free to join our \sphinxhref{http://fastadmmodeling.slack.com}{slack channel}.
\subsubsection*{References}
\subsubsection*{Acknowledgements}

\sphinxAtStartPar
The research was supported by the U.S. Department of Energy, Office of Energy Efficiency and Renewable Energy, Bioenergy Technologies Office, under contract DE\sphinxhyphen{}AC02\sphinxhyphen{}06CH11357.



\renewcommand{\indexname}{Index}
\printindex
\end{document}